\documentclass{article}
\usepackage[utf8]{inputenc}
\usepackage[T1]{fontenc}
\usepackage[utf8]{inputenc}
\usepackage{lmodern}
\usepackage{xhfill}
\usepackage[a4paper, margin=1in]{geometry}
\usepackage{parskip}
\usepackage{fancyhdr}
\usepackage{color,soul}
\usepackage{amssymb}
\usepackage{graphicx}
\usepackage{comment}
\usepackage{amsmath}
\usepackage{amsthm}
\usepackage{amsmath}
\usepackage{amssymb}
\usepackage{mathtools}
\usepackage{blindtext}
\usepackage{titlesec}
\usepackage{tikz}
\usepackage{ upgreek }
\usepackage{tcolorbox}
\usepackage{hyperref}

\newcommand\mc{\mathcal}


\usetikzlibrary{decorations.pathmorphing, topaths, positioning,shapes, quotes, calc, graphs,graphs.standard, fit}

\pagestyle{fancy}
\lhead{1241}
\chead{MATH 239: Graph Theory Review}
\rhead{Jaiden Ratti}

\usepackage{minted}
\large
\title{MATH239 Enumerative Combinatorics}
\begin{document}

\begin{center}
Note
\end{center}

These are \textit{most of} the definitions and theorems for the Graph Theory section of the course. They are from memory as practice for the exam (hence the most of). It's probably a good idea to make sure you know these before the exam.  


\section{Introduction To Graph Theory}

\textbf{Def:} Graphs are isomorphic if there exists a bijection $f: V(G_1) \to V(G_2)$ such that $f(u)$ and $f(v)$ are adjacent \textit{if and only if} $u$ and $v$ are adjacent in $G$. 

\textbf{Thm} $\sum_{v \in V(G)} \deg(v) = 2 |E(G)|$ (Handshake Lemma)

\textbf{Thm} Number of vertices of odd degree is even. 

\textbf{Thm} Average degree of a vertex is $\frac{2|E(G)|}{|V(G)|}$.

\textbf{Def} Complete Graph ($\mathcal{K}_n$) has all vertices adjacent to each other. 

\textbf{Def} Complete Bipartite Graph ($\mathcal{K}_{m,n}$) has all vertices in $A$ adjacent to all vertices in $B$ where $|A| = m, |B| = n$.

\textbf{Def} A subgraph of $G$ has a vertex set that is a subset $U$ of $V(G)$ and whose edge set is a subset of those edges of $G$ with both vertices in $U$. 

\textbf{Def} A walk is an alternating sequence of vertices and edges. The length of a walk is indicated by the number of edges. A walk is closed if $v_0 = v_n$

\textbf{Def} A path is a walk with all vertices unique (and thus distinct edges). 

\textbf{Def} A trail is a walk with no repeated edges 

\textbf{Thm} If there exists a walk from $x$ to $y$, then there is a path from $x$ to $y$. 

\textbf{Def} A cycle is a subgraph with $n$ distinct vertices and $n$ distinct edges (where $n \ge 1$). Or, also known as a connected regular graph of degree $2$. 

\textbf{Thm} If every vertex has degree at least $2$, then $G$ contains a cycle. 

\textbf{Def} Girth of a graph $G$ is the length of the smallest cycle. 

\textbf{Def} Spanning cycle is known as a Hamilton cycle (visits every vertex exactly once). 

\textbf{Def} A graph is connected if for each two vertices $x,y$, there is a path from $x$ to $y$. 

\textbf{Thm} If for each vertex $w$ in $G$ there is a path from $v$ to $w$ in $G$, then $G$ is connected. 

\textbf{Def} A component of $G$ is a subgraph $C$ of $G$ such that 
a) $C$ is connected
b) No subgraph that properly contains $X$ is connected.

\textbf{Def} A cut induced by $X$ is the set of edges that have exactly one end in $X$. 

\textbf{Thm} A graph is not connected \textit{if and only if} there exists a proper non-empty subset $X$ of $V(G)$ such that the cut induced by $X$ is empty. 

\textbf{Def} Eulerian circuit of a graph $G$ is a closed walk that contains every edge of $G$ once. 

\textbf{Thm} $G$ has an Eulerian circuit \textit{if and only if} every vertex has even degree. 

\textbf{Def} An edge $e$ is a bridge if $G - e$ has more components than $G$. 

\textbf{Thm} An edge $e$ is a bridge of a graph \textit{if and only if} it is not contained in a cycle of $G$. 

\textbf{Cor} If there are two distinct paths from $u$ to $v$, $G$ contains a cycle. 

\section{Trees}

\textbf{Def} A tree is a connected graph with no cycles.

\textbf{Def} A forest is a graph with no cycles. 

\textbf{Lemma} If $u$ and $v$ are in $T$, there exists a unique $uv$-path in $T$. 

\textbf{Lemma} Every edge in a tree is a bridge. 

\textbf{Thm} If $T$ is a tree, $|E(T)| = |V(T)|-1$

\textbf{Cor} If $G$ is a forest with $k$ components, $|E(G)| = |V(G)| - k$.

\textbf{Def} A leaf in a tree is a vertex of degree $1$. 

\textbf{Thm} A tree with at least $2$ vertices has at least two leaves. 

\textbf{Def} A spanning subgraph that is a tree is a spanning tree. They have the fewest edges while remaining connected. 

\textbf{Thm} A graph $G$ is connected \textit{if and only if} it has a spanning tree. 

\textbf{Cor} If $G$ is connected, with $p$ vertices and $q = p-1$ edges, then $G$ is a tree. 

\textbf{Thm} If $T$ is a spanning tree of $G$ and $e$ is an edge not in $T$, then $T + e$ contains exactly one cycle $C$. If $e'$ is any edge in $C$, then $T + e - e'$ is also a spanning tree of $G$. 

\textbf{Thm} If $T$ is a spanning tree of $G$ and $e$ is an edge in $T$, then $T - e$ has $2$ components. If $e'$ is in the cut induced by one of the components, then $T - e + e'$ is also a spanning tree of $G$. 

\textbf{Thm} An odd cycle is not bipartite. 

\textbf{Thm} A graph is bipartite \textit{if and only if} it has no odd cycles. 

\section{Planar Graphs}

\textbf{Def} A graph is planar if no edges intersect. 

\textbf{Thm} Given a planar embedding of a connected graph $G$ with faces $f_1, \ldots, f_s$, then $\sum_{i=1}^s\deg(f_i) = 2|E(G)|$ (Faceshaking Lemma). Note that, a bridge contributes $2$ to one face, but a non-bridge edge contributes $1$ to one face, and $1$ to another face.

\textbf{Cor} If a connected planar graph $G$ has $f$ faces, the average degree of a face is $\frac{2|E(G)|}{f}$

\textbf{Thm} Let $G$ is a connected graph with $p$ vertices and $q$ edges. If $G$ has a planar embedding with $f$ faces, then $p + f - q = 2$.

\textbf{Def} A platonic solid is a graph where all faces have the same degree and all vertices have the same degree. 

\textbf{Thm} There are exactly five platonic solids (See below Lemma).

\textbf{Lemma} $($vertex degree, face degree$) \to \{(3,3),(3,4),(4,3),(5,3),(3,5)\}$

\textbf{Thm} If $G$ contains a cycle, then in a planar embedding of $G$, the boundary of each face contains a cycle.

\textbf{Thm} Let $G$ be a planar embedding with $p$ vertices, $q$ edges. If each face has degree $\ge d^{*}$, then $(d^{*} - 2)q \le d^{*}(p-2)$.

\textbf{Thm} In a planar graph with $p \ge 3$ vertices and $q$ edges, $q \le 3p-6$.

\textbf{Cor} $\mathcal{K}_5$ is not planar. 

\textbf{Cor} A planar graph has a vertex of degree at most five. 

\textbf{Thm} In a bipartite planar graph with $p \ge 3$ vertices and $q$ edges, $q \le 2p - 4$. 

\textbf{Lemma} $\mathcal{K}_{3,3}$ is not planar. 

\textbf{Def} An edge subdivision takes an edge and replaces it with a path of length $1$ or more. 

\textbf{Thm} A graph is not planar \textit{if and only if} it has a subgraph that is an edge subdivision of $\mathcal{K}_5$ or $\mathcal{K}_{3,3}$, 

\textbf{Def} A graph with $k-$colouring has adjacent vertices as different colours (and can be done with a total of $k$ colours). 

\textbf{Thm} A graph is $2-$colourable \textit{if and only if} it is bipartite. 

\textbf{Thm} $\mathcal{K}_n$ is $n-$colourable, and not $k$-colourable for any $k < n$. 

\textbf{Thm} Every planar graph is $6-$colourable. 

\textbf{Thm} Every planar graph is $5-$colourable. 

\textbf{Thm} Every planar graph is $4-$colourable. 

\section{Matchings}

\textbf{Def} A matching in a graph is a set of $M$ edges such that no two edges have a common end. 

\textbf{Def} A vertex is saturated by $M$ if the vertex is incident with an edge in $M$. 

\textbf{Def} We are often interested in finding a maximum matching. 

\textbf{Def} A perfect matching has size $\frac{p}{2}$ since it saturates every vertex. Every perfect matching is a maximum matching, 

\textbf{Def} A path is an alternating path with respect to $M$ if it alternates between $M$ and not $M$. 

\textbf{Def} A path is augmented if it joins two distinct vertices that are not saturated by $M$. Augmented paths always have odd length. 

\textbf{Lemma} If $M$ has an augmenting path, it is not a maximum matching. 

\textbf{Def} A cover of graph $G$ is a set $C$ of vertices such that every edge of $G$ has at least one end in $C$. 

\textbf{Lemma} If $M$ is a matching of $G$ and $C$ is a cover of $G$, then $|M| \le |C|$. 

\textbf{Lemma} If $M$ is a matching and $C$ is a cover and $|M| = |C|$, then $M$ is a maximum matching and $C$ is a minimal cover. 

\textbf{Thm} In a bipartite graph, the maximum size of a matching is the minimum size of a cover. 

\underline{$XY$ Construction}

$G$ is bipartite ($A, B$), $M$ is a matching of $G$. 

$X_0 = $ set of vertices in $A$ not saturated by $M$. 
$Y_0 = $ set of vertices in $B$ unsaturated.
$Z = $ set of vertices in $G$ that are joined to a vertex in $X_0$ by an alternating path. 

$X = A \cap Z$
$Y = B \cap Z$

If $Y \cap Y_0 = \emptyset$, $M$ is a maximum matching and $C = Y \cup (A \setminus X)$

\begin{itemize}
    \item No edge of $G$ from $X$ to $B \setminus Y$.
    \item $C = Y \cup (A \setminus X)$ is a cover.
    \item No edge of $M$ from $Y$ to $A \setminus X$.
    \item $|M| = |C| - |U|$ where $U$ is the set of unsaturated vertices in $Y$
    \item Augmenting path to each vertex in $U$. 
\end{itemize}

(There's also the Bipartite Matching Algorithm. Pretty much the same thing, just faster). 

\textbf{Thm} A bipartite graph $G$ with bipartition $A, B$ has a matching saturating every vertex in $A$, \textit{if and only if} every subset of $D$ of $A$ satisfies $|N(D)| \ge |D|$

\textbf{Def} Let $D \le V(G)$. The neighbour set of $D$ is the set of all vertices adjacent to at least one vertex in $D$. $N(D) = \{v \in V(G): u \in D, uv \in E(G)\}$

\textbf{Cor} A bipartite graph $G$ with bipartition $A,B$ has a perfect matching \textit{if and only if} $|A| = |B|$ and every subset $D$ of $A$ satisfies $|N(D)| \ge |D|$. 

\textbf{Thm} If $G$ is a $k-$regular bipartite graph with $k \ge 1$, then $G$ has a perfect matching. 

\textbf{Def} A graph with an edge $k-$colouring has edges incident with a vertex assigned different colours. 

\textbf{Thm} A bipartite graph with maximum degree $\Delta$ has an edge $\Delta-$colouring.

\textbf{Lemma} Let $G$ be a bipartite graph having at least one edge. Then $G$ has a matching saturating each vertex of maximum degree. 





\begin{center}
Tips
\end{center}

\begin{itemize}
    \item $p \iff q$ statements are powerful. Definitely remember those. 
    \item Induction is useful and usually very straightforward (colourings, trees etc.)
    \item When in doubt, contradiction (usually longest path or odd number of odd degrees). 
\end{itemize}

\end{document}
