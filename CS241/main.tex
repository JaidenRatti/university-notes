
\documentclass{article}

\usepackage{amsmath, amsthm, amssymb, amsfonts}
\usepackage{thmtools}
\usepackage{graphicx}
\usepackage{setspace}
\usepackage{float}
\usepackage{hyperref}
\usepackage[utf8]{inputenc}
\usepackage[english]{babel}
\usepackage{framed}
\usepackage[dvipsnames]{xcolor}
\usepackage{environ}
\usepackage{tcolorbox}
\usepackage[utf8]{inputenc}
\usepackage[T1]{fontenc}
\usepackage[utf8]{inputenc}
\usepackage{lmodern}
\usepackage{xhfill}
\usepackage{makecell}
\usepackage{parskip}
\usepackage[a4paper, margin=1in]{geometry}
\usepackage{fancyhdr}
\usepackage{color,soul}
\usepackage{amssymb}
\usepackage{graphicx}
\usepackage{comment}
\usepackage{mathtools}
\usepackage{blindtext}
\usepackage{titlesec}
\usepackage{circuitikz}
\usepackage[linguistics]{forest}
\usepackage{amsmath,amssymb,graphicx}
\usepackage[]{algorithm2e}
\usepackage{tikz}
\usepackage{ stmaryrd }
\usepackage{ listings }
\usepackage{tcolorbox}
\usepackage{tikz}
\usetikzlibrary{shapes.geometric, arrows}

\tikzstyle{startstop} = [rectangle, rounded corners, minimum width=3cm, minimum height=1cm,text centered, draw=black, fill=red!30]
\tikzstyle{io} = [trapezium, trapezium left angle=70, trapezium right angle=110, minimum width=3cm, minimum height=1cm, text centered, draw=black, fill=blue!30]
\tikzstyle{process} = [rectangle, minimum width=3cm, minimum height=1cm, text centered, draw=black, fill=orange!30]
\tikzstyle{decision} = [diamond, minimum width=3cm, minimum height=1cm, text centered, draw=black, fill=green!30]
\tikzstyle{arrow} = [thick,->,>=stealth]
\tcbuselibrary{theorems,skins,breakable}

\usetikzlibrary{automata,positioning}
% Theorem System
% The following boxes are provided:
%   Definition:     \defn 
%   Theorem:        \thm 
%   Lemma:          \lem
%   Corollary:      \cor
%   Proposition:    \prop   
%   question:          \qn
%   Fact:           \fact
%   Proof:          \pf
%   Example:        \ex
%   Remark:         \rmk (sentence), \rmkb (block)
% Suffix
%   r:              Allow Theorem/Definition to be referenced, e.g. thmr
%   p:              Add a short proof block for Lemma, Corollary, Proposition or question, e.g. lemp
%                   For theorems, use \pf for proof blocks

% Definition
\newtcbtheorem[number within=section]{mydefinition}{Definition}
{
    enhanced,
    frame hidden,
    titlerule=0mm,
    toptitle=1mm,
    bottomtitle=1mm,
    fonttitle=\bfseries\large,
    coltitle=black,
    colbacktitle=cyan!20!white,
    colback=cyan!10!white,
}{defn}

\NewDocumentCommand{\defn}{m+m}{
    \begin{mydefinition}{#1}{}
        #2
    \end{mydefinition}
}

\NewDocumentCommand{\defnr}{mm+m}{
    \begin{mydefinition}{#1}{#2}
        #3
    \end{mydefinition}
}

% Theorem
\newtcbtheorem[use counter from=mydefinition]{mytheorem}{Theorem}
{
    enhanced,
    frame hidden,
    titlerule=0mm,
    toptitle=1mm,
    bottomtitle=1mm,
    fonttitle=\bfseries\large,
    coltitle=black,
    colbacktitle=green!20!white,
    colback=green!10!white
}{thm}

\NewDocumentCommand{\thm}{m+m}{
    \begin{mytheorem}{#1}{}
        #2
    \end{mytheorem}
}

\NewDocumentCommand{\thmr}{mm+m}{
    \begin{mytheorem}{#1}{#2}
        #3
    \end{mytheorem}
}

% Lemma
\newtcbtheorem[use counter from=mydefinition]{mylemma}{Lemma}
{
    enhanced,
    frame hidden,
    titlerule=0mm,
    toptitle=1mm,
    bottomtitle=1mm,
    fonttitle=\bfseries\large,
    coltitle=black,
    colbacktitle=violet!20!white,
    colback=violet!10!white,
}{lem}

\NewDocumentCommand{\lem}{m+m}{
    \begin{mylemma}{#1}{}
        #2
    \end{mylemma}
}

\newenvironment{lempf}{
	{\noindent{\it \textbf{Proof for Lemma}}}
	\tcolorbox[blanker,breakable,left=5mm,parbox=false,
    before upper={\parindent15pt},
    after skip=10pt,
	borderline west={1mm}{0pt}{violet!20!white}]
}{
    \textcolor{violet!20!white}{\hbox{}\nobreak\hfill$\blacksquare$} 
    \endtcolorbox
}

\NewDocumentCommand{\lemp}{m+m+m}{
    \begin{mylemma}{#1}{}
        #2
    \end{mylemma}

    \begin{lempf}
        #3
    \end{lempf}
}

% Corollary
\newtcbtheorem[use counter from=mydefinition]{mycorollary}{Corollary}
{
    enhanced,
    frame hidden,
    titlerule=0mm,
    toptitle=1mm,
    bottomtitle=1mm,
    fonttitle=\bfseries\large,
    coltitle=black,
    colbacktitle=orange!20!white,
    colback=orange!10!white,
}{cor}

\NewDocumentCommand{\cor}{+m}{
    \begin{mycorollary}{}{}
        #1
    \end{mycorollary}
}

\newenvironment{corpf}{
	{\noindent{\it \textbf{Proof for Corollary.}}}
	\tcolorbox[blanker,breakable,left=5mm,parbox=false,
    before upper={\parindent15pt},
    after skip=10pt,
	borderline west={1mm}{0pt}{orange!20!white}]
}{
    \textcolor{orange!20!white}{\hbox{}\nobreak\hfill$\blacksquare$} 
    \endtcolorbox
}

\NewDocumentCommand{\corp}{m+m+m}{
    \begin{mycorollary}{}{}
        #1
    \end{mycorollary}

    \begin{corpf}
        #2
    \end{corpf}
}

% Proposition
\newtcbtheorem[use counter from=mydefinition]{myproposition}{Proposition}
{
    enhanced,
    frame hidden,
    titlerule=0mm,
    toptitle=1mm,
    bottomtitle=1mm,
    fonttitle=\bfseries\large,
    coltitle=black,
    colbacktitle=yellow!30!white,
    colback=yellow!20!white,
}{prop}

\NewDocumentCommand{\prop}{+m}{
    \begin{myproposition}{}{}
        #1
    \end{myproposition}
}

\newenvironment{proppf}{
	{\noindent{\it \textbf{Proof for Proposition.}}}
	\tcolorbox[blanker,breakable,left=5mm,parbox=false,
    before upper={\parindent15pt},
    after skip=10pt,
	borderline west={1mm}{0pt}{yellow!30!white}]
}{
    \textcolor{yellow!30!white}{\hbox{}\nobreak\hfill$\blacksquare$} 
    \endtcolorbox
}

\NewDocumentCommand{\propp}{+m+m}{
    \begin{myproposition}{}{}
        #1
    \end{myproposition}

    \begin{proppf}
        #2
    \end{proppf}
}

% question
\newtcbtheorem[use counter from=mydefinition]{myquestion}{Question}
{
    enhanced,
    frame hidden,
    titlerule=0mm,
    toptitle=1mm,
    bottomtitle=1mm,
    fonttitle=\bfseries\large,
    coltitle=black,
    colbacktitle=pink!30!white,
    colback=pink!20!white,
}{qn}


\NewDocumentCommand{\qn}{m+m}{
    \begin{myquestion*}{#1}{}
        #2
    \end{myquestion*}
}

\newenvironment{qnpf}{
	{\noindent{\it \textbf{Proof for question.}}}
	\tcolorbox[blanker,breakable,left=5mm,parbox=false,
    before upper={\parindent15pt},
    after skip=10pt,
	borderline west={1mm}{0pt}{pink!30!white}]
}{
    \textcolor{pink!30!white}{\hbox{}\nobreak\hfill$\blacksquare$} 
    \endtcolorbox
}

\NewDocumentCommand{\qnp}{m+m+m}{
    \begin{myquestion*}{#1}{}
        #2
    \end{myquestion*}

    \begin{qnpf}
        #3
    \end{qnpf}
}

% Fact
\newtcbtheorem[use counter from=mydefinition]{myfact}{Fact}
{
    enhanced,
    frame hidden,
    titlerule=0mm,
    toptitle=1mm,
    bottomtitle=1mm,
    fonttitle=\bfseries\large,
    coltitle=black,
    colbacktitle=purple!20!white,
    colback=purple!10!white,
}{fact}

\NewDocumentCommand{\fact}{+m}{
    \begin{myfact}{}{}
        #1
    \end{myfact}
}


% Proof
\NewDocumentCommand{\pf}{+m}{
    \begin{proof}
        [\noindent\textbf{Proof.}]
        #1
    \end{proof}
}

% Example
\newenvironment{example}{%
    \par
    \vspace{5pt}
	\begin{minipage}{\textwidth}
		\noindent\textbf{Example.}
		\tcolorbox[blanker,breakable,left=5mm,parbox=false,
	    before upper={\parindent15pt},
	    after skip=10pt,
		borderline west={1mm}{0pt}{cyan!10!white}]
}{%
		\endtcolorbox
	\end{minipage}
    \vspace{5pt}
}

\NewDocumentCommand{\ex}{+m}{
    \begin{example}
        #1
    \end{example}
}


% Remark
\NewDocumentCommand{\rmk}{+m}{
    {\it \color{blue!50!white}#1}
}

\newenvironment{remark}{
    \par
    \vspace{5pt}
    \begin{minipage}{\textwidth}
        {\par\noindent{\textbf{Remark.}}}
        \tcolorbox[blanker,breakable,left=5mm,
        before skip=10pt,after skip=10pt,
        borderline west={1mm}{0pt}{cyan!10!white}]
}{
        \endtcolorbox
    \end{minipage}
    \vspace{5pt}
}

\NewDocumentCommand{\rmkb}{+m}{
    \begin{remark}
        #1
    \end{remark}
}


\newcommand{\lcm}{\operatorname{lcm}}



\newcommand{\Dashv}{%
  \mathrel{\text{\reflectbox{$\vDash$}}}%
}
\newcommand{\vDashv}{%
  \mathrel{%
    \text{%
      \ooalign{$\vDash$\cr\reflectbox{$\vDash$}\cr}%
    }%
  }%
}

\newcommand{\vdashv}{%
    \mathrel{%
        \text{%
            \ooalign{$\vdash$ \cr $\dashv$ \cr}%    
        }
    }
}


\newcommand\vv{\vec}


\pagestyle{fancy}
\lhead{1249}
\chead{CS241: Foundations of Sequential Programs}
\rhead{Jaiden Ratti}

\usepackage{minted}
\large
\title{CS241: Foundations of Sequential Programs}
\begin{document}
\begin{titlepage}
	\begin{center}
    \line(1,0){300}\\
    [0.65cm]
	\huge{\bfseries Foundations of Sequential Programs}\\
	\line(1,0){300}\\
	\textsc{\Large CS241}\\
	\textsc{\Large  Jaiden Ratti}\\
        \textsc{\Large Prof. Chengnian Sun}\\
	[5.5cm]
	\end{center}
\end{titlepage}




\tableofcontents

\pagebreak


\section{Lecture 1}\label{lecture-1}

\qn{}{
What is a sequential program?
}

\qn{}{
What really happens when I compile and run a
program?
}

\qn{}{
How does a computer take code and turn it into
something it can utilize?
}

By the end of the course, there should be very little mystery left about
computers or computer programs.

High level overview of compilers

Compiler: Scans (turns into substrings) \(\to\) Parses \(\to\) Optimizes \(\to\) Codegen \{backend\}



\defn{}{
A bit is a binary digit (0 or 1)
}

\defn{}{
A nibble is 4 bits (1001)
}

\defn{}{
A byte is 8 bits (10011101)
}

\defn{} {
A word is a machine-specific grouping of bytes.
For us, a word will be 4 bytes (32-bit architecture) though 8-byte
(64-bit architecture) words are more common now.
}

\defn{}{
The base-16 representation is called the
hexadecimal system. It consists of the numbers from 0-9 and the letters
a-f (convert the number from 10 to 15 in decimal).
}

The binary number \(10011101\) will convert to \(9d\) in hexadecimal.
Break it into \(2\) nibbles.

\(\underbrace{1001}_{9}\) and \(\underbrace{1101}_{d}\) \(= 9d\)

\(0x9d\), the \(0x\) denotes a hexadecimal representation (or can do
\(9d_{16}\)). A conversion table will be provided in the exam.

Bytes as binary numbers
\begin{itemize}
    \item Unsigned (non-negative integers)
    \item Signed integers
\end{itemize}

Unsigned

The value of a number stored in this system is the binary sum, that is

\(b_72^7 + b_62^6 + b_52^5 + b_42^4 + b_32^3 + b_22^2 + b_12^1 + b_0\)
where \(b_n\) is either \(0\) or \(1\)

For example,

\(01010101_2 = 2^6 + 2^4 + 2^2 + 2^0 = 64 + 16 + 4 + 1 = 85_{10}\)

\(11111111_2 = 255_{10}\)

A byte (unsigned) can represent \(256\) numbers

Converting from decimal to binary:

Approach 1: Take the largest power of \(2\) less than \(n\), subtract
and repeat.

Approach 2: Repeatedly divide by \(2\)

The remainder (from bottom to top) is the binary representation

\(\frac{N - b_0}{2} = b_1+ 2b_2 +2^2b_3 = 19\) (when \(N = 38\)).
Remainder \(b_0 = 0\)

Signed Integers

\qn{}{
How do we represent negative integers?
}

Attempt 1: Make the first bit a signed bit. This is called the
``sign-magnitude'' representation.

\(0\) represents \(+\) and \(1\) represents \(-\), use the rest of the
bits to create the number.

We get positive and negative zero.

\((+1) \quad 0000 0001 + (-1) \quad 1000 0001 =  (2) \quad 1000 0010\)

Attempt 2: Two's complement form

Similar to sign-magnitude in spirit. First bit is \(0\) if non-negative,
\(1\) if negative (MSB is either \(-b_n2^n\))

Negate value by just subtracting from zero and letting it overflow.

A trick to get the same thing:

\begin{itemize}
\item
  Take the complement of all bits and then add 1.
\item
  Or, locate the rightmost \(1\) bit and flip all the bits to the left
  of it.
\end{itemize}

Decimal to Two's Complement

Compute \(-38_{10}\) using one byte of space. First write \(38\) in
binary.

\(0010 \quad 0110_{2}\)

Negate this number

\(1101 \quad 1010_{2}\) in 2's complement.

Two's Complement to Decimal

To convert \(11011010_2\) to decimal, one method is to flip the bits and
add \(1\) (or do the shortcut). Then compute and convert positive to
negative number.

Another way is to treat as unsigned and subtract

\(11011010_2 = -2^8 + 2^7 + 2^6 + 2^4 + 2^3 + 2^1 = 218 - 256 = -38\)

\section{Lecture 2}\label{lecture-2}

Warmup

\(-15\) to twos complement

\(15 \rightarrow 0000 1111\) (absolute value)

Convert to twos complement (shortcut)

\(-15 \rightarrow 11110001\)

\(-1\) to twos complement

\(1 \rightarrow 0000 0001\)

Convert to twos complement (shortcut)

\(-1 \rightarrow 11111111\)

\(128\) to twos complement

\(128 \rightarrow 1000 0000\)

\(-128 \rightarrow 01111111\)

Convert to absolute value using shortcut

\(128 \rightarrow 1000 0000\)

This is wrong. Cannot represent \(128\) this way (but can represent
\(-128\))
\begin{minted}[frame=lines, linenos, fontsize=\large]
{c}
int abs(int a);
\end{minted}

\begin{minted}[frame=lines, linenos, fontsize=\large]
{cpp}
if (a >= 0) return a;
else return -a;
\end{minted}

Bytes as Characters

ASCII uses \(7\) bits to represent characters.

Note that `a' is different than 0xa. Former is the decimal number \(97\)
in ACII, latter is the number \(10\) in decimal.

\begin{minted}[frame=lines, linenos, fontsize=\large]
{cpp}
int main() {
    printf("%c", 48);
    return 0;
}
\end{minted}

\texttt{Prints\ \$0\$.}

\subparagraph{Bit-Wise Operators}\label{bit-wise-operators}

Suppose we have

\texttt{unsigned\ char\ a\ =\ 5,\ b\ =\ 3}

\(a = 5 = 0000 \quad 0101\)

\(b = 3 = 0000 \quad 0011\)

Bitwise not \textasciitilde: negate bits (unary operator)

\(c = \sim a = 1111 \quad 1010\)

Bitwise and \&: (binary operator)

\(c = a \& b = 0000 \quad 0 0 0 1\)

Bitwise or \(|\): (binary operator)

\(c = a | b = 0000 \quad 0111\)

Bitwise exclusive or \^{}: (binary operator)

\(c = a\)\^{}\(b = 0000 \quad 0110\)

Returns 1 \(\iff\) bits are different

Bitwise shift right or left \(>>\) and \(<<\) (still dealing with
unsigned)

\(c = a >> 2 = 0000 0001\)

Discard rightmost \(n\) bits and fill left \(n\) \(0's\)

\(\frac{a}{2^n}\) where \(n\) is the number of bits shifted

\(c = a << 3 = 00101000\)

Discard leftmost \(n\) bits and fill right \(n 0's\)

\(2^n \cdot a\) where \(n\) is the number of bits shifted


\qn{}{
What is a Computer Program?
}

Programs operate on data.

Programs are data. This is a von Neumann architecture: Programs live in
the same memory space as the data they operate on.

Programs can manipulate other programs.

We will use 32-bit MIPS in this course

CPU
\begin{itemize}
\item 32 general purpose registers 
\item Control Unit
\item Memory
\begin{itemize}
    \item Registers
    \item L1 cache
    \item L2 cache
    \item RAM
    \item Disk
    \item Network memory
\end{itemize}

\item ALU
\end{itemize}

Registers are very fast memory.

Some general-purpose registers are special:
\begin{itemize}
\item \texttt{\$0} is always
\(0\)
\item \texttt{\$31} is for return address
\item \texttt{\$30} is our stack
pointer
\item \texttt{\$29} is our frame pointer
\end{itemize}

Code is just data: it is stored in RAM.

MIPS takes instructions from RAM and attempts to execute them.

Recall Problem: We only know from context what bits have what meaning,
and in particular, which are instructions.

Solution: Convention is to set memory address \(0\) in RAM to be an
instruction.

\qn{} { Problem: How does MIPS know what to do next?}

Solution: Have a special register called the Program Counter (PC) to
tell us what instruction to do next.

\qn{} { Problem: How do we put our program into RAM?}

Solution: A program called a loader puts our program into memory and
sets the PC to be the first address.

Fetch-Execute Cycle

\begin{tcolorbox}
\begin{verbatim}
PC = 0
while true do
    IR = MEM[PC] // loading the first word/instruction into IR
    PC += 4 // update PC to the next instruction 
    Decode and execute instruction in IR // note we increment PC first
end while
\end{verbatim}
\end{tcolorbox}

This is the only program a machine really runs. This is the essence of
the CPU cycle.

First Example: Addition

Write a program in MIPS that adds the values in registers \texttt{\$8}
and \texttt{\$9} and stores the result in register \texttt{\$3}

\texttt{add:}Add

\texttt{000000ssssstttttddddd00000100000}

\texttt{\$s\ =\ source}, \texttt{\$t\ =\ second\ source},
\texttt{\$d\ =\ destination}

\(000000|01000|01001|00011|00000100000\)

Putting values in Registers

\texttt{lis}: Load Immediate \& Skip

\texttt{0000000000000000ddddd00000010100}

Load value \(1\) into register \texttt{\$3}

\texttt{lis:\ (000\ ...\ 00011\ ...)} load and escape to the next
instruction \texttt{value:\ 000\ ...\ 1} (which is \(1\))

\texttt{\$d\ =\ MEM\ {[}PC{]};\ PC\ +=\ 4}

\section{Lecture 3}\label{lecture-3}

\subparagraph{Recall}\label{recall}

Fetch-Execute Cycle

\begin{tcolorbox}
\begin{verbatim}
PC = 0 // mem address of next instruction
while true {
   IR = MEM[PC]
   PC = PC+4
   Decode IR then execute
}
\end{verbatim}
\end{tcolorbox}

\texttt{add}: Add

\(000000 \quad \underbrace{sssss}_{\text{operand reg}} \quad \underbrace{ttttt}_{\text{operand reg}} \quad \underbrace{ddddd}_{\text{result}} \quad 00000 \quad 10000\)

\texttt{s\ =\ source}, \texttt{t\ =\ second\ source},
\texttt{d\ =\ destination}

\texttt{lis}: Load Immediate \& Skip, for putting values directly into
registers.

\(00 \ldots 0 \quad ddddd \quad 00 \ldots 1..0\)

\begin{verbatim}
$3 = 1
0x0: 000... 00 00011 0....1..0 (d = 3)
0x4: 0000000000000000000000000000001 (value 1)
\end{verbatim}

\texttt{\$d\ =\ MEM{[}PC{]};\ PC\ +\ 4}

\begin{tcolorbox}
\begin{verbatim}
$d = MEM[PC] // load
$3 = MEM[0x4]
PC becomes 0x8
\end{verbatim}
\end{tcolorbox}

Add values \(11\) and \(13\) and store the result in register \(3\).

\begin{tcolorbox}
\begin{verbatim}
0x0: lis 01000 // 
0x4: 11 as a 32 bit word
0x8: lis 01001
0xc: 13 as a 32 bit word
0x10: add 01000 and 01001 and store in 00011
\end{verbatim}
\end{tcolorbox}

\qn{}{ How do we stop?}

Operating system provides a return address in register \texttt{\$31}.

We get to that return address via \texttt{jr} (Jump Register).

Multiplying two words together might give a word that requires twice as
much space.

To deal with this, we use two registers: \texttt{hi} and \texttt{lo}.

\texttt{hi} has the leftmost 32 bits, and \texttt{lo} has the rightmost
32 bit.

Division performs integer division and stores the quotient in
\texttt{lo} and remainder in \texttt{hi}.

\texttt{hi} and \texttt{lo} are special purpose registers: they don't
have register numbers.

\texttt{mfhi} and \texttt{mflo} are general purpose registers.

\texttt{mfhi:\ \$3} \texttt{11\ *\ \$3\ =\ hi}

Larger amount of memory is stored off the CPU.

RAM access is slower than register access (but is larger).

Data travels between RAM and CPU via the bus.

Words occur every 4 bytes, starting with byte 0. Indexed by
\(0, 4, 8, \ldots n-4\).

Cannot directly use the data in the RAM. Must transfer first to the
registers.

Operations on RAM.

Load word takes a word from RAM and places it into a register.
Specifically, load the word in \texttt{MEM{[}\$s\ +\ i{]}} and store in
\texttt{\$t}.

\texttt{load\ \$t\ \$s(i)} where
\texttt{\$t:\ result\ reg;\ \$s:\ mem\ address;\ i:\ 16-54\ bit\ signed\ offset\ (number)}

\begin{tcolorbox}
\begin{verbatim}
0x4444
lis $3 <- mem address
binary word for 4
load $3, $3(0)
// register 3 has the value in memory address 4
\end{verbatim}
\end{tcolorbox}

Load \(\equiv\) Read and Store \(\equiv\) Write

Store word takes a word from a register and stores it into RAM.
Specifically, load the word in \texttt{\$t} and store it in
\texttt{MEM{[}\$s\ +\ i{]}}.

\begin{tcolorbox}
\begin{verbatim}
store $0, $3(0)$
MEM[$3 + 0] = 0
\end{verbatim}
\end{tcolorbox}

Machine code is difficult to memorize and edit.

Assembly language is a text language which there is a 1-to-1
correspondence between assembly instructions and machine code
instructions. This is more human-readable. Higher-level languages have a
more complex mapping to machine code.

\section{Lecture 4}\label{lecture-4}

A string like \texttt{add\ \$3,\ \$2,\ \$1} is stored as a sequence of
characters.

If we can break this down to meaningful chunks of information, it would
be easier to translate.

\texttt{{[}add{]}{[}\$3{]}{[}\$2{]}{[}\$1{]}}

A scanner breaks strings down into tokens. Not as simple as looking for
spaces, the words need to make sense.

Scanner extracts the Kind and Lexeme

\texttt{Indentifier:\ "sub"}


\begin{tcolorbox}
    

\begin{verbatim}
int a = 241;
INT: int
Identifier: "a"
EQ: =
NUM: 241
SEMI: ";"
\end{verbatim}
\end{tcolorbox}


The string ``241'' has four ways to be split up.

In the C statement \texttt{int\ x\ =\ 241;} we want to interpret ``241''
as a single number.

\defn{}{ Maximal munch: Always choose the longest
meaningful token while breaking up the string
}

\qn{}{
Then how do we know when to stop? There's no
limit to how long a number can be.
}

We need a way to wrangle infinitely many possible strings.

\defn{} {An alphabet is a non-empty, finite set of
symbols, often denoted by \(\Sigma\)
}

\defn{}{ A string (or word) \(w\) is a finite sequence of
symbols chosen from \(\Sigma\).}

\defn{}{The set of all strings over an alphabet
\(\Sigma\) is denoted by \(\Sigma *\)
}

\defn{}{
A language is a set of strings
}

\defn{}{
The length of a string \(w\) is denoted by
\(|w|\)
}

Alphabets:

\(\Sigma = \{a,b,c,\ldots,z\}\) the Latin alphabet

\(\Sigma = \{0,1\}\) the alphabet of binary digits

\(\Sigma = \{0,1,2,\ldots,9\}\) the alphabet of base 10 digits

\(\Sigma = \{0,1,2,\ldots, 9, a, b, c, d, e, f\}\) hexadecimal

Strings:

\(\varepsilon\) is the empty string. It is in \(\Sigma*\) for any
\(\Sigma\). \(|\varepsilon| = 0\)

For \(\Sigma = \{0,1\}\), strings include \(w = 011101\) or
\(x = 1111\). Note \(|w| = 6\) and \(|x| = 4\).

Languages:

\(L = \emptyset\) or \(\{\}\), the empty language

\(L = \{\varepsilon\}\), the language consisting of (only) the empty
string

\(L = \{ab^na: n \in \mathbb{N}\}\), the set of strings over the
alphabet \(\Sigma = \{a,b\}\) consisting of an \(a\) followed by \(0\)
or more \(b\) characters followed by an \(a\).

The objective of our scanner is to break a string into words in a given
language.

Simpler objective: Given a language, determine if a string belongs to
the language.

How hard is this? Depends on the language.

\(L =\) any dictionary: Trivial

\(L = \{ab^na: n \in \mathbb{N}\}\): Very easy

\(L = \{\)Valid MIPS assembly programs \(\}\): Easy

\(L = \{\)Valid Java/C/C++ programs \(\}\): Harder

\(L = \{\)Set of programs that halt \(\}\): Impossible

Memberships in Languages

In order of relative difficulty:
\begin{itemize}
    \item Finite
    \item Regular
    \item Context-free
    \item Context-sensitive
    \item Recursive
    \item Impossible languages
\end{itemize}

Consider the language \{bag, bat, bit\}.

We can just check whether a string is in the list.

\qn{}{ But can we do this in a more efficient way?
}

Hash Map/Set

\begin{tcolorbox}
\begin{verbatim}
for each w in L {
    if w' = w
        Accept
}
\end{verbatim}
\end{tcolorbox}

Still not the most efficient

Most efficient is to check each letter at a time and reject if it can't
be possible.

In our example, all our words start with b. If our first symbol for
input is not b, we can instantly reject. If the first symbol is b and
our second is a, we can reject ``bit'' and keep going. Similar to prefix
tree.

Important Features of Diagram
\begin{itemize}
    \item An arrow into the initial start state
    \item Accepting states are two circles 
    \item Arrows from state to state are labelled
    \item Error state(s) are implicit
\end{itemize}

\defn{}{ A regular language over an alphabet \(\Sigma\)
consists of one of the following:
\begin{enumerate}
    \item The empty language and the language
consisting of the empty word are regular
    \item All languages \(\{a\}\) for
all \(a \in \Sigma\) are regular.
    \item The union, concatenation or Kleene
star of any two regular languages are regular. 
    \item Nothing else
\end{enumerate}
}

Let \(L, L_1, L_2\) be three regular languages. Then the following are
regular languages

Union: \(L_1 \cup L_2 = \{x: x \in L_1 \text{ or } x \in L_2\}\)

Concatenation:
\(L_1 \cdot L_2 = L_1 L_2 = \{xy: x \in L_1, y \in L_2\}\)

Kleene star:
\(L^* = \{\varepsilon\} \cup \{xy: x \in L^*, y \in L\} = \cup_{n=0}^{\infty} L^n\)

Suppose that \(L_1 =\) \{up, down\}, \(L_2 =\) \{hill, load\} and
\(L =\{a,b\}\) over appropriate alphabets. Then

\begin{itemize}
\item
  \(L_1 \cup L_2 =\) \{up, down, hill load\}
\item
  \(L_1 \cdot L_2\) = \{uphill, upload, downhill, download\}
\item
  \(L^* = \{\varepsilon, a, b, aa, ab, ba, bb, aaa, aab, aba, abb, baa, bab, \ldots\}\)
\end{itemize}

Let \(\Sigma = \{a,b\}\). Explain why the language
\(L = \{ab^na: n \in \mathbb{N}\}\) is regular.

Solution: \(\{a\}\) and \(\{b\}\) are finite, and so regular.
\(\{b\}^*\) is also regular, regular languages are closed under Kleene
star. Then, the concatenation \(\{a\} \cdot \{b\}^* \cdot \{a\}\) must
also be regular.

In tools like \texttt{egrep}, regular expressions are often used to help
find patterns of text. Regular expressions are just a way of expressing
regular languages.

The notation is very similar, except we drop the set notation. As
examples

\begin{itemize}
    \item \(\{\varepsilon\}\) becomes \(\varepsilon\)
    \item \(L_1 \cup L_2\) becomes \(L_1 | L_2\)
    \item Concatenation is never written
with the explicit \(\cdot\)
    \item Order of operations: \(^*, \cdot, |\)
    \item \(?\) means optional
\end{itemize}


Extending the Finite Languages Diagram

We can allow our picture to have loops.

\defn{}{ Deterministic Finite Automata (DFA). A DFA is a
\(5\)-tuple (\(\Sigma, Q, q_0, A, \delta\))
\begin{itemize}
    \item \(\Sigma\) is a finite non-empty set (alphabet)
    \item \(Q\) is a finite non-empty set of states
    \item \(q_0 \in Q\) is a start state 
    \item \(A \subseteq Q\) is a set of accepting states 
    \item \(\delta: (Q \times E) \to Q\) is our total transition function (given a state and a symbol of our alphabet, what state should we go to?).    
\end{itemize}
}

\section{Lecture 5}\label{lecture-5}

Creating Binary in C++

How do we write the binary output

\texttt{0001\ 0100\ 0100\ 0000\ 1111\ 1111\ 1111\ 1101}

\texttt{bne\ \$2,\ \$0,\ -1}

Convert the registers to binary bits

We can use bit shifting to put the information into the correct
position.

\texttt{bne} opcode is 5

\texttt{int\ instr\ =\ (5\ \textless{}\textless{}\ 26)\ \textbar{}\ (2\ \textless{}\textless{}\ 21)\ \textbar{}\ (0\ \textless{}\textless{}\ 16)\ \textbar{}\ offset}

Shift opcode 26 left, shift binary representation of register
\texttt{\$s} left 21 bits, shift binary representation of register
\texttt{\$t} by 16 bits.

We need to be careful with the offset.

Recall in C++, ints are 4 bytes. We only want the last two bytes. First
we need to apply a ``mask'' to only get the last 16 bits.

\texttt{int\ offset\ =\ -1}

32-bit \texttt{000....1}

32-bit \texttt{111....0} bitwise or

32-bit \texttt{111....1}

This does not work since we are changing all 32 bits, we only need to
change the last 16 bits.

We discard the leftmost 16 bits above.

\texttt{int\ offset\ =\ -1\ \&\ 0xffff}

Can declare the offset to be \texttt{int16\_t} (still need to do the bit
masking).

Just need to \texttt{cout\ \textless{}\textless{}\ instr} right? Wrong.

This output would be 9 bytes, corresponding to the ASCII code for each
digit of the instruction as interpreted in decimal. We want to put the
four bytes that correspond to this number.

Printing Bytes in C++

\begin{minted}[frame=lines, linenos, fontsize=\large]
{cpp}
int instr = (5 << 26) | (2 << 21) | (0 << 16) | (-1 & 0xffff);
unsigned char c = instr >> 24;
cout << c;
c = instr >> 16; cout << c;
c = instr >> 8; cout << c;
c = instr; cout << c;
\end{minted}


You can also mask here to get the ``last byte'' by doing \& \texttt{0xff} if
worried about which byte will get copied over.


Rules for DFA

States can have labels inside the bubble, this is how we refer to the
states in \(Q\).

For each character, follow the transition. If there is none, go to the
implicit error state.

Once the input is exhausted, check if the final state is accepting. If
so, accept. Otherwise, reject.

Warm-up Problem

Write a DFA over \(\Sigma = \{a,b\}\) that
\begin{enumerate}
   \item Accepts only words with an even number of \(a\)s 
   \item Accepts only words with an odd number of \(a\)s and an even number of \(b\)s 
   \item Accepts only words where the parity of the number of \(a\)s is equal to the parity of the number of \(b\)s 
   \item Write a DFA over \(\Sigma = \{a, b\}\) that accepts all words ending
with bba.
\end{enumerate}

Start

\begin{enumerate}
\item
Accepts only words with an even number of \(a\)s
\end{enumerate}

\(\Sigma = \{a,b\}\)

\(Q = \{q_0, q_1\}\)

\(q_0\) is our start state

\(A = \{q_0\}\)

\(q_0 \coloneqq\) even \(a\)s

\(q_1 \coloneqq\) odd \(a\)s

\(\delta\) is defined by
\begin{itemize}
    \item \(\delta(q_0, b) = q_0\)
    \item \(\delta(q_0, a) = q_1\)
    \item \(\delta(q_1, b) = q_1\)
    \item \(\delta(q_1, a) = q_0\)
\end{itemize}
 
\begin{enumerate}
\def\labelenumi{\arabic{enumi}.}
\setcounter{enumi}{1}
\item
  Accepts only words with an odd number of \(a\)s and an even number of
  \(b\)s
\end{enumerate}

\(\Sigma = \{a,b\}\)

\(Q = \{q_0, q_1, q_2, q_3\}\)

\(q_0\) is our start state

\(A = q_1\)

\(q_0 \coloneqq\) even \(a\)s, odd \(b\)s

\(q_1 \coloneqq\) odd \(a\)s, even \(b\)s

\(q_2 \coloneqq\) even \(a\)s, odd \(b\)s

\(q_3 \coloneqq\) odd \(a\)s, odd \(b\)s

\(\delta\) is defined by:
\begin{itemize}
    \item \(\delta(q_0, a) = q_1\)
    \item \(\delta(q_0, b) = q_2\)
    \item \(\delta(q_1, a) = q_0\)
    \item \(\delta(q_1, b) = q_3\)
    \item \(\delta(q_2, a) = q_3\)
    \item \(\delta(q_2, b) = q_0\)
    \item \(\delta(q_3, a) = q_2\)
    \item \(\delta(q_3, b) = q_1\)
\end{itemize}


\begin{enumerate}
\def\labelenumi{\arabic{enumi}.}
\setcounter{enumi}{2}
\item
  Accepts only words where the parity of the number of \(a\)s is equal
  to the parity of the number of \(b\)s
\end{enumerate}

\(\Sigma = \{a,b\}\)

\(Q = \{q_0, q_1, q_2, q_3\}\)

\(q_0\) is our start state

\(A = q_0, q_3\)

\(q_0 \coloneqq\) even \(a\)s, odd \(b\)s

\(q_1 \coloneqq\) odd \(a\)s, even \(b\)s

\(q_2 \coloneqq\) even \(a\)s, odd \(b\)s

\(q_3 \coloneqq\) odd \(a\)s, odd \(b\)s

\(\delta\) is defined by:
\begin{itemize}
    \item \(\delta(q_0, a) = q_1\)
    \item \(\delta(q_0, b) = q_2\)
    \item \(\delta(q_1, a) = q_0\)
    \item \(\delta(q_1, b) = q_3\)
    \item \(\delta(q_2, b) = q_0\)
    \item \(\delta(q_3, b) = q_1\)
    \item \(\delta(q_2, a) = q_3\)
    \item \(\delta(q_3, a) = q_2\)
\end{itemize}


This is not the optimal solution.

\(\Sigma = \{a,b\}\)

\(Q = \{q_0, q_1\}\)

\(q_0\) is our start state

\(A = q_0\)

\(q_0 \coloneqq\) same parity

\(q_1 \coloneqq\) different parity

\(\delta\) is defined by:
\begin{itemize}
    \item \(\delta(q_0, a) = q_1\)
    \item \(\delta(q_0, b) = q_1\)
    \item \(\delta(q_1, a) = q_0\)
    \item \(\delta(q_1, b) = q_0\)
\end{itemize}


\begin{enumerate}
\def\labelenumi{\arabic{enumi}.}
\setcounter{enumi}{3}

\item
  Write a DFA over \(\Sigma = \{a, b\}\) that accepts all words ending
  with bba.
\end{enumerate}

\(\Sigma = \{a,b\}\)

\(Q = \{q_0, q_1, q_2, q_3\}\)

\(q_0\) is our start state

\(A = q_3\)

\(\delta\) is defined by:
\begin{itemize}
    \item \(\delta(q_0, a) = q_0\) (self loop)
    \item \(\delta(q_0, b) = q_1\)
    \item \(\delta(q_1, b) = q_2\)
    \item \(\delta(q_2, a) = q_3\)
    \item \(\delta(q_1, a) = q_0\) (need to start over)
    \item \(\delta(q_2, b) = q_2\) (self loop)
    \item \(\delta(q_3, a) = q_0\) (start over)
    \item \(\delta(q_3, b) = q_1\) (start over but not at beginning)
\end{itemize}


\defn{}{ The language of a DFA \(M\) is the set of all
strings accepted by \(M\), that is:
\(L(M) = \{w : M \text{ accepts } w\}\)
}

\begin{tcolorbox}
\begin{verbatim}
w = a_1a_2...a_n
s = q_0
for i in 1 to n do
    s = \delta(s,ai)
end for
if s in A then
    Accept
else
    Reject
end if
\end{verbatim}
\end{tcolorbox}

You could also use a lookup table.

\thm{Kleene}{ \(L\) is regular if and only if
\(L = L(M)\) for some DFA \(M\). That is, regular languages are
precisely the languages accepted by DFAs.
}

\qn{}{ Is C a regular language?
}

The following are regular
\begin{itemize}
    \item C keywords
    \item C identifiers
    \item C literals
    \item C operators
    \item C comments
\end{itemize}


Sequences of these are also regular (Kleene star). Finite automata can
do our tokenization.

\qn{}{What about punctuation? Even simpler, set
\(\Sigma = \{(,)\}\) and \(L =\) \{strings with balanced parentheses\}.
Is \(L\) regular?
}

\begin{tcolorbox}
\begin{verbatim}
w = ()
w = (())
w = (((()))((())))
\end{verbatim}
\end{tcolorbox}

This language is not regular. It is a context-free language (more on
this later).

\qn{}{ How does our scanner work?
}

Our goal is: Given some text, break up the text into tokens.

Some tokens can be recognized in multiple different ways.

\texttt{w\ =\ 0x12cc}. This could be a single hex, or could be an int
followed by an (x) followed by another int (1) followed by another int
(2) and followed by an id (cc).

Formalization of this problem.

Given a regular language \(L\) (say, \(L\) is all valid MIPS or C
tokens), determine if a given word \(w\) is in \(LL*\) (or in other
words, is \(w \in L * \setminus \{\varepsilon\})\) (we don't consider
the empty program to be valid).

Consider the language \(L\) of just ID tokens in MIPS:

\(Q = \{q_0, q_1\}\)

\(q_0\) is our start state

\(A = \{q_1\}\)

\(\delta\) is defined by:
\begin{itemize}
    \item \(\delta(q_0, a-z, A-Z) = q_1\)
    \item \(\delta(q_1, a-z, A-Z, 0-9) = q_1\)
    \item \(\delta\left( q_1, {\varepsilon}\text{/output token} \right) = q_0\)
\end{itemize}

\texttt{w\ =\ abcde}

Initially in \(q_0 \to q_1\).

Then there are two options. Either stay in \(q_1\), or return the empty
token.

In this case, we would get \texttt{{[}ID,\ "a"{]}}

Go back to \(q_0\). We can do the same thing again after reaching
\(q_1\) with \texttt{b}.

Now, we also get \texttt{{[}ID,\ "b"{]}}

We can keep doing this

This time, we want to stay in \(q_1\). We will stay in \(q_1\) for
\texttt{d} and \texttt{e}. We have consumed all the symbols.

We would then output \texttt{{[}ID,\ "cde"{]}}. We now have a longer
token.

\section{Lecture 6}\label{lecture-6}

Given a regular language \(L\), determine if a word \(w\) is in \(LL*\).

Two algorithms:
\begin{itemize}
    \item Maximal munch
    \item Simplified maximal munch
\end{itemize}

Idea: Consume the largest possible token that makes sense. Produce the
token and then proceed.

Difference:
\begin{itemize}
\item Maximal Munch: Consume characters until you no longer have
a valid transition. If you have characters left to consume, backtrack to
the last valid accepting state and resume.
\item Simplified Maximal Munch:
Consume characters until you no longer have a valid transition. If you
are currently in an accepting state, produce the token and proceed.
Otherwise go to an error state.
\end{itemize}

DFA for now

\(\Sigma = \{a,b,c\}\)

\(L = \{a,b,abca\}\)

\(q_0\) is our start state

\(A = \{q_1, q_4, q_5\}\)

\(\delta\) is defined by:
\begin{itemize}
    \item \(\delta(q_0, a) = q_1\)
    \item \(\delta(q_0, b) = q_5\)
    \item \(\delta(q_1, b) = q_2\)
    \item \(\delta(q_2, c) = q_3\)
    \item \(\delta(q_3, a) = q_4\)
\end{itemize}


Note there is a \(\varepsilon\)/output token from the accepting states
to the start state.

Maximal Munch: \(\Sigma = \{a,b,c\}, L = \{a,b,abca\}, w = ababca\)
\begin{itemize}
    \item Algorithm consumes \(a\) and flags this state as its accepting state. Then, \(b\) tries to consume \(a\) but ends up in an error state.
    \item Algorithm then backtracks to the first \(a\) since that was the last accepting state. Token \(a\) is output.
    \item Algorithm then resumes consuming \(b\) and flags this state as accepting. Then, it tries to consume \(a\) but ends up in an error state.
    \item Algorithm then backtracks to the first \(b\) since that was the last accepting state. Token \(b\) is output.
    \item Algorithm then consumes the second \(a\), the second \(b\), the first \(c\), the third \(a\), and runs out of input. This last state is accepting, so it outputs the last token \(abca\) and accepts.
\end{itemize}


Simplified Maximal Munch:
\(\Sigma = \{a,b,c\}, L = \{a,b,abca\}, w = ababca\) \begin{itemize}
    \item Algorithm consumes \(a\), then \(b\) tries to consume \(a\) but ends up in an error state. Note there is no keeping track of the first accepting state.
    \item Algorithm then checks to see if \(ab\) is accepting. It is not (as \(ab \not \in L\)).
    \item Algorithm rejects \(ababca\).
    \item Note: This gave the wrong answer, but this algorithm is usually good enough and is used in practice.
\end{itemize}


Simplified demo

\(a\): \(q_0 \to q_1\) (accepting)

\(b: q_1 \to q_2\)

\(a: q_2 \to\) ERROR (give error)

Consider the following C++ line:

\begin{minted}[frame=none, fontsize=\large]
{cpp}
vector<pair<string, int>> v;
\end{minted}


Notice that at the end, there is the token \(>>!\). This, on its own is
a valid token. With either algorithm we would reject this declaration.
To do this declaration, you needed a space:

\begin{minted}[frame=none, fontsize=\large]
{cpp}
vector<pair<string, int> > v;
\end{minted}

\qn{}{ What was the point of scanning?
}

Machine language is hard to write: We want to use assembly language.

We need to scan assembly lines in order to compile assembly language.

All the machine language operations we've seen so far, as assembly.

\begin{tcolorbox}
\begin{verbatim}
add $d
lis $d
.word i
jr $s
mult $s, $t
dif $s, $t
mfhi $d
mflo $d
lw $t, i($s)
sw $t, i($s)
\end{verbatim}
\end{tcolorbox}

The order of \texttt{\$s}, \texttt{\$t}, and \texttt{\$d} are different
in assembly than machine code.

Suppose that \texttt{\$1} contains the address of an array and
\texttt{\$2} takes the number of elements in this array (assume small
enough that we don't have to worry about overflow). Place the number
\(7\) in the last possible spot in the array.


\begin{tcolorbox}
\begin{verbatim}
lis $8
.word 0x7 // store 7 in $8
lis $9
.word 4 // store 4 in $9
mult $2, $9 // num of elements * 4
mflo $3 // move above product to $3
add $3, $3, $1 // add this offset to address
sw $8 {-}4($3) // store the value at the end
jr $31}
\end{verbatim}
\end{tcolorbox}

\qn{}{ Write an assembly language MIPS program that takes a
value in register \texttt{\$1} and stores the value of its last base-10
digit in register \texttt{\$2}
}

\begin{tcolorbox}
\begin{verbatim}
lis $10
.word 10
div $1 $10
mfhi $2 // hi = remainder, lo = quotient
jr $31
\end{verbatim}
\end{tcolorbox}

MIPS also comes equipped with control statements.

\texttt{beq\ \$s,\ \$t,\ i}

\begin{minted}[frame=lines, linenos, fontsize=\large]
{cpp}
if ($s == $t) {
    PC += i * 4;
}
\end{minted}

\texttt{beq\ \$s,\ \$t,\ i}

\begin{minted}[frame=lines, linenos, fontsize=\large]
{cpp}
if ($s != $t) {
    PC += i * 4;
}
\end{minted}

\begin{tcolorbox}
\begin{verbatim}
beq $0, $0, 1 //skip 1 instruction
add $1, $2, $3
jr $31
\end{verbatim}
\end{tcolorbox}

If \(i == 0\), doesn't skip next instruction. If \(i == -1\), infinite
loop.

\qn{}{Write an assembly language MIPS program that places the
value \(3\) in register \texttt{\$2} if the signed number in register
\texttt{\$1} is odd and places the value \(11\) in register \texttt{\$2}
if the number is even.
}

\begin{tcolorbox}
\begin{verbatim}
lis $8
.word 2 ; $8 == 2
lis $9
.word 3 ;$9 == 3
lis $2
.word 11 ; assume even
div $1 $8
mfhi $3
beq $3 $0 1
add $2, $9, $0
jr $31
\end{verbatim}
\end{tcolorbox}

Inequality Command

\texttt{slt\ \$d,\ \$s,\ \$t}

Set Less Than. Sets the value of register \texttt{\$d} to be \(1\)
provided the value in register \texttt{\$s} is less than the value in
register \texttt{\$t} and sets it to be \(0\) otherwise.

\begin{minted}[frame=lines, linenos, fontsize=\large]
{cpp}
if ($s < $t) {
    $d = 1
} else {
    $d = 0
}
\end{minted}

Note: There is also an unsigned version of this command.

\qn{}{Write an assembly language MIPS program that negates the
value in register \texttt{\$1} provided its positive.
}

\begin{tcolorbox}
\begin{verbatim}
slt $2, $1, $0 ; $1 < $0 -> $2 = 1. $1 > $0 -> $2 = 0.
bne $2, $0, 1 ; if $2 != 0, then $2 is negative
sub $1, $0, $1
jr $31
\end{verbatim}    
\end{tcolorbox}


\qn{}{ Write an assembly language MIPS program that places the
absolute value of register \texttt{\$1} in register \texttt{\$2}
}

\begin{tcolorbox}
\begin{verbatim}
add $2, $1, $0 ; assume $1 is positive
slt $3, $0, $1 ; 0 < $1 (if $1 > 0 -> $3 = 1) (else $3 = 0)
bne $3, $0, 1
sub $2, $0, $2 ; ($2 = -$2)
jr $31
\end{verbatim}
\end{tcolorbox}

With branching we can even do looping.

\qn{}{Write an assembly language MIPS program that adds
together all even numbers from \(1\) to \(20\) inclusive. Store the
answer in register \texttt{\$3}.
}

\begin{tcolorbox}
\begin{verbatim}
lis $2
.word 20
lis $1
.word 2
add $3, $3, $0
add $3, $3, $2 ; $3 = $3 + $2
sub $2, $2, $1 ; $2 = $2 - 2
bne $2, $0, -3 ; (if not == 0, will go back to first add)
jr $31
\end{verbatim}    
\end{tcolorbox}

Hard coding the \(-3\) above isn't good for the long run. We fix this by
using a label.

\texttt{label:\ operation\ commands}

Explicit Example

\begin{tcolorbox}
\begin{verbatim}
0x0: sub $3, $0, $0
0x4: sample:
0x4: add $1, $0, $0
\end{verbatim}    
\end{tcolorbox}

We can thus do
\begin{tcolorbox}
\begin{verbatim}
lis $2
.word 20
lis $1
.word 2
add $3, $0, $0
top:
    add $3, $3, $2
    sub $2, $2, $1
    bne $2, $0, top
jr $31
\end{verbatim}
\end{tcolorbox}

Assembler computes the difference between the program counter and
\texttt{top}. PC is the line number after the current line.

\section{Lecture 7}\label{lecture-7}

What if a procedure wants to use registers that have data already.

We could preserve registers: save and restore them.

We have lots of memory in RAM we can use. We don't want our procedures
to use the same RAM.

Used RAM goes after Free RAM.

Calling procedures pushes more registers onto the stack and returning
pops them off.

We call \texttt{\$30} our stack pointer.

Template for Procedures

\(f:\) procedure modifies \texttt{\$1} and \texttt{\$2}.

Entry: preserve \texttt{\$1} and \texttt{\$2}

Exit: restore \texttt{\$1} and \texttt{\$2}

\begin{tcolorbox}
\begin{verbatim}
f:
    sw $1, -4($30) ; Push registers modified
    sw $2, -8($30)
    lis $2 ; Decrement stack pointer
    .word 8
    sub $30, $30, $2
    ; Code
    add $30, $30, $2 ; Assuming $2 is still 8
    lw $2, -8($30)
    lw $1, -4($30)
\end{verbatim}
\end{tcolorbox}

There is a problem with returning:

\begin{tcolorbox}

\begin{verbatim}
main:
lis $8
.word f ; Recall f is an address
jr $8 ; Jump to the first line of f
\end{verbatim}
\end{tcolorbox}

We get a new command \texttt{jalr\ \$s}.

Jump and Link Register. Sets \texttt{\$31} to be the PC and then sets
the PC to be \texttt{\$s}. Accomplished by \texttt{temp\ =\ \$s} then
\texttt{\$31\ =\ PC} then \texttt{PC\ =\ temp}.

\texttt{jalr} will overwrite register \texttt{\$31}. How do we return to
the loader from main after using \texttt{jalr}? What if procedures call
each other?

We need to save this register first.

\qn{}{How do we pass arguments?
}

Typically, we'll just use registers. If we have too many, we could push
parameters to the stack and then pop them.

Sum Evens \(1\) to \(N\) (assume \(N > 1\))

\begin{tcolorbox}
\begin{verbatim}
; sumEvens1toN adds all even numbers from 1 to N
; Registers:
; $1 Temp Register (Should save)
; $2 Input Register (Should save)
; $3 Output Register (Do not save)

sumEvens1ToN:
    sw $1, -4($30) ; Save $1 and $2
    sw $2, -8($30)
    lis $1
    .word 8
    sub $30, $30, $1 ; Decrement stack pointer
    add $3, $0, $0 ; Initialize $3
    lis $1
    .word 2
    div $2, $1 ; is N even?
    mfhi $1
    sub $2, $2, $1 ; Sub 1 if not
    lis $1
    .word 2 ; Restore 2
    top:
    add $3, $3, $2
    sub $2, $2, $1
    bne $2, $0, top
    lis $1
    .word 8
    add $30, $30, $1 ; Restore stack pointer
    lw $2, -8($30)
    lw $1, -4($30) ; Reload $1 and $2
    jr $31 ; Back to caller
\end{verbatim}
\end{tcolorbox}

\qn{}{ How do we print to the screen or read input?
}

We do this one byte at a time.

Output: Use \texttt{sw} to store words in location \texttt{0xffff000c}.
Least significant byte will be printed.

Input: Use \texttt{lw} to load words in location \texttt{0xffff0004}.
Least significant byte will be the next character from \texttt{stdin}.

Print \texttt{cs} to the screen followed by a newline character.

\begin{tcolorbox}
\begin{verbatim}
lis $1
.word 0xffff000c
lis $2
.word 67 ; c
sw $2 0($1)
lis $2
.word 83 ; s
sw $2, 0($1)
lis $2
.word 10 ; \n
sw $2, 0($1)
jr $31
\end{verbatim}
\end{tcolorbox}


Let's finish up the assembler. Language translation involves two phases:
Analysis and Synthesis.

Analysis: Understand what is meant by the input source. Use DFAs and
maximal munch to break into tokens. But there's more; valid ranges,
labels.

Synthesis: Output the equivalent target code in the new format

The Biggest Analysis Problem

How do we assemble this code:
\begin{tcolorbox}
\begin{verbatim}
beq $0, $1, myLabel
myLabel:
add $1, $1, $1
\end{verbatim}
\end{tcolorbox}


The problem is that \texttt{myLabel} is used before it's defined: we
don't know the address when it's used.

The best fix to this is to perform two passes.

Pass 1: Group tokens into instructions and record addresses of labels.
(Note: multiple labels are possible for the same line).

Pass 2: Translate each instruction into machine code. If it refers to a
label, look up the associated address and compute the value.

\section{Lecture 8}\label{lecture-8}

Recall the definition of DFA

We can extend the definition of \(\delta: (Q \times \Sigma^*) \to Q\) to
a function defined over \((Q \times \Sigma^*)\) via:
\begin{align*}
\delta^* : ( Q \times \Sigma^*) \to Q \\
(q, \varepsilon) \to q \\
(q, aw) \to \delta^*(\delta(q,a),w)
\end{align*}
where \(a \in \Sigma\) and \(w \in \Sigma^*\). If processing a
string, process a letter first then process the rest of the string.

\defn{}{ A DFA given by \(M = (\Sigma, Q, q_0, A, \delta)\) accepts a
string \(w\) if and only if \(\delta^*(q_0, w) \in A\)
}

What if we allowed more than one transition from a state with the same
symbol?

To make the right choice, we would need an oracle that can predict the
future.

This is called non-determinism. We then say that a machine accepts a
word \(w\) if and only if there exists some path that leads to an
accepting state.

We can simplify the ``ends with bba'' example from previous lecture to
an NFA.

\(L = \{w : w \text{ ends with bba} \}\)

\(Q = \{q_0, q_1, q_2, q_3\}\)

\(q_0\) is our start state

\(A = \{q_3\}\)

Transitions
\begin{itemize}
    \item \(\delta(q_0, a) \to q_0\)
    \item \(\delta(q_0, b) \to q_0\)
    \item \(\delta(q_0, b) \to q_1\)
    \item \(\delta(q_1, b) \to q_2\)
    \item \(\delta(q_2, a) \to q_3\)
\end{itemize}


\defn{}{ Definition Let \(M\) be an NFA. We say that \(M\) accepts
\(w\) if and only if there exists some path through \(M\) that leads to
an accepting states.
}

The language of an NFA \(M\) is the set of all strings accepted by
\(M\), that is: \(L(M) = \{w: M \text{ accepts } w\}\)

\defn{}{Definition An NFA is a \(5\)-tuple
(\(\Sigma, Q, q_0, A, \delta):\)
\begin{itemize}
    \item \(\Sigma\) is a finite non-empty set
    \item \(Q\) is a finite non-empty set of states
    \item \(q_0 \in Q\) is a start state
    \item \(A \subseteq Q\) is a set of accepting states
    \item \(\delta: (Q \times \Sigma) \to 2^Q\) is our total transition function. Note that \(2^Q\) denotes the power set of \(Q\), that is, the set of all subsets of \(Q\). This allows us to go to multiple states at once
\end{itemize}

}

We can extend the definition of \(\delta: (Q \times \Sigma) \to 2^Q\)
via:


\begin{align*}
\delta^*: (2^Q \times \Sigma^*) \to 2^Q \\
(S, \varepsilon) \mapsto S \\
(S, aw) \mapsto \delta^* \left ( \bigcup_{q \in S} \delta(q,a), w \right )
\end{align*}
 where \(a \in \Sigma\). We also have:

\defn{}{ Definition An NFA given by
\(M = (\Sigma, Q, q_0, A, \delta)\) accepts a string \(w\) if and only
if \(\delta^*(\{q_0\}, w) \cap A \ne \emptyset\)
}

Using the NFA defined earlier (bba) process \texttt{abbba}

\(w =\) \texttt{abbba}

\(S = \{q_0\}\)

Process \texttt{a}

\(S = \bigcup_{q \in \{q_0\}} \delta(q, a)\)

\(= \delta(q_0, a) = \{q_0\}\) (self-loop is the only option) 
\(S = \{q_0\}\)

Process \texttt{b}

\(S = \bigcup_{q \in \{q_0\}} \delta(q, b)\)

\(= \delta(q_0, b) = \{q_0, q_1\}\) (two options)
\(S = \{q_0, q_1\}\)

Process \texttt{b}

\(S = \bigcup_{q \in \{q_0, q_1\}} \delta(q,b) = \delta(q_0, b) \cup \delta(q_1, b)\)

\(= \{q_0, q_1\} \cup \{q_2\}\)

\(= \{q_0, q_1, q_2\}\) \(S = \{q_0, q_1, q_2\}\)

Process \texttt{b}

\(S = \bigcup_{q \in \{q_0, q_1, q_2\}} \delta(q_0, b) \cup \delta(q_1, b) \cup \delta(q_2, b)\)

\(= \{q_0, q_1\} \cup \{q_2\} \cup \emptyset\)

\(= \{q_0, q_1, q_2\}\)  \(S = \{q_0, q_1, q_2\}\)

Process \texttt{a}

\(S = \bigcup_{q \in \{q_0, q_1, q_2\}} \delta(q_0, a) \cup \delta(q_1, a) \cup \delta(q_2, a)\)

\(= \{q_0\} \cup \emptyset \cup \{q_3\}\)

\(= \{q_0, q_3\}\) 
\(S \cap A = S \cap \{q_3\} = \{q_0, q_3\} =  \{q_3\}\)

Thus the input string is accepted by the NFA.

\qn{}{ Why are NFAs not more powerful than DFAs?
}

Even the power-set of a set of states is still finite. We can represent
the set of states in the NFA as single states in the DFA.

Algorithm to convert from NFA to DFA.

\begin{itemize}

\item
  Start with the state \(S = \{q_0\}\)
\item
  From this state, go to the NFA and determine what happens on each
  \(a \in \Sigma\) for each \(q \in S\). The set of resulting states
  should become its own state in your DFA.
\item
  Repeat the previous step for each new state created until you have
  exhausted every possibility.
\item
  Accepting states are any states that included an accepting state of
  the original NFA.
\end{itemize}

Previous NFA as a DFA.

\(\Sigma = \{a,b\}\)

\(S_0 = \{q_0\}\)

\(A = \{q_3\}\)

Transitions (for DFA) 
\begin{itemize}
    \item \(\delta_D(S_0, a) = \{q_0\} = S_0\)
    \item \(\delta_D(S_0, b) = \{q_0, q_1\} = S_1\)
    \item \(\delta_D(S_1, a) = S_0\)
    \item \(\delta_D(S_1, b) = \delta(\{q_0\}, b) \cup \delta(\{q_1\}, b) = \{q_0, q_1\} \cup \{q_2\} = \{q_0, q_1, q_2\} = S_2\)
    \item \(\delta_D(S_2, a) = \{q_0, q_3\} = S_3\)
    \item \(\delta_D(S_2, b) = S_2\)
    \item \(\delta_D(S_3, a) = S_0\)
    \item \(\delta_D(S_3, b) = S_1\)
\end{itemize}


States in our DFA \(S_0 = \{q_0\}\) \(S_1 = \{q_0, q_1\}\)
\(S_2 = \{q_0, q_1, q_2\}\) \(S_3 = \{q_0, q_3\}\)

\ex{Let \(\Sigma = \{a,b,c\}\) Write an NFA such that
\(L = \{w: w \text { does not contain ac\}}\)}{ 
}

\(\Sigma = \{a,b,c\}\)

\texttt{not\ ending\ with\ a} \(= q_0\) is our start state

\(Q = \{q_0, q_1\}\)

\texttt{ending\ with\ a} \(= q_1\)

\(A = \{q_0, q_1\}\)

Transitions: 
\begin{itemize}
    \item \(\delta(q_0, b) \to q_0\)
    \item \(\delta(q_0, c) \to q_0\)
    \item \(\delta(q_0, a) \to q_1\)
    \item \(\delta(q_1, a) \to q_1\)
    \item \(\delta(q_1, b) \to q_0\)
\end{itemize}


\ex{ Let \(\Sigma = \{a,b,c\}\) Write an NFA such that
\(L = \{abc\} \cup \{w: w \text{ ends with cc\}}\)
}{}

\(\Sigma = \{a,b,c\}\)

First element of union

\(A = \{q_3\}\)

\(Q = \{q_0, q_1, q_2, q_3\}\)

\(q_0\) is our start state

Transitions 
\begin{itemize}
    \item \(\delta(q_0, a) \to q_1\)
    \item \(\delta(q_1, b) \to q_2\)
    \item \(\delta(q_2, c) \to q_3\)
    \item Second element of union
\end{itemize}


\(A = \{q_2\}\)

\(Q = \{q_0, q_1, q_2\}\)

\(q_0\) is our start state

Transitions
\begin{itemize}
    \item \(\delta(q_0, a) \to q_0\)
    \item \(\delta(q_0, b) \to q_0\)
    \item \(\delta(q_0, c) \to q_0\)
    \item \(\delta(q_0, c) \to q_1\)
    \item \(\delta(q_1, c) \to q_2\)
\end{itemize}


\qn{}{ How do we combine these?
}

\(Q = \{q_0, , q_1, q_2,q_3,q_4,q_5,q_6\}\)

\(A = \{q_3, q_6\}\)

Transitions 
\begin{itemize}
    \item \(\delta(q_0, a) \to q_1\)
    \item \(\delta(q_1, b) \to q_2\)
    \item \(\delta(q_2, c) \to q_3\)
    \item \(\delta(q_0, a,b,c) \to q_4\)
    \item \(\delta(q_4, a,b,c) \to q_4\)
    \item \(\delta(q_4, c) \to q_5\)
    \item \(\delta(q_0, c) \to q_5\)
    \item \(\delta(q_5, c) \to q_6\)
\end{itemize}


The DFA is complicated. Note: Combining two languages is non-obvious.

Summary: From Kleene's theorem, the set of languages accepted by a DFA
are the regular languages. The set of languages accepted by DFAs are the
same as those accepted by NFAs. Therefore, the set of languages accepted
by an NFA are precisely the regular languages.

\qn{}{What if we permitted state changes without reading a
character.
}

These are known as \(\varepsilon\) transitions.

\defn{}{
An \(\varepsilon\)-NFA is a
5-tuple(\(\Sigma, Q, q_0, A, \delta\))
\begin{itemize}
    \item \(\Sigma\) is a finite non-empty set that does not contain the symbol \(\varepsilon\)
    \item \(Q\) is a finite non-empty set of states
    \item \(q_0 \in Q\) is a start state
    \item \(A \subseteq Q\) is a set of accepting states
    \item \(\delta : (Q \times \Sigma \cup \{\varepsilon\}) \to 2^Q\) is our total transition function
\end{itemize}

}

These \(\varepsilon\)-transitions make it trivial to take the union of
two NFAs.

\(Q = \{q_0,q_1,q_2,q_3,q_4,q_5,q_6,q_7\}\)

\(A = \{q_4, q_7\}\)

Transitions:
\begin{itemize}
    \item \(\delta(q_0, \varepsilon) \to q_1\)
    \item \(\delta(q_0, \varepsilon) \to q_5\)
    \item \(\delta(q_1, a) \to q_2\)
    \item \(\delta(q_2, b) \to q_3\)
    \item \(\delta(q_3, c) \to q_4\)
    \item \(\delta(q_5, a,b,c) \to q_5\)
    \item \(\delta(q_5, c) \to q_6\)
    \item \(\delta(q_6, c) \to q_7\)
\end{itemize}


Extending \(\delta\) for an \(\varepsilon\)-NFA

Define \(E(S)\) to be the epsilon closure of the set of states \(S\),
that is, the set of all states reachable from \(S\) in 0 or more
\(\varepsilon\) transitions.

Note, this implies that \(S \subset E(S)\)

Again we can extend the definition of
\(\delta: (Q \times \Sigma \cup \{\varepsilon\}) \to 2^Q\) to a function
\(\delta^* : (2^Q \times \Sigma^*) \to 2^Q\) via:
\begin{align*}
\delta^*: (2^Q \times \Sigma^*) \to 2^Q \\
(S, \varepsilon) \mapsto E(S) \\
(S, aw) \mapsto \delta^* \left ( \bigcup_{q \in S} E(\delta(q,a)),w \right )
\end{align*}

where \(a \in \Sigma\). We also have

\defn{}{ Definition An \(\varepsilon\)-NFA given by
\(M = (\Sigma, Q, q_0, A, \delta)\) accepts a string \(w\) if and only
if \(\delta^*(E\{q_0\},w) \cap A \ne \emptyset\)
}

\section{Lecture 9}\label{lecture-9}

Recall from last lecture:

Extending \(\delta\) for an \(\varepsilon\)-NFA

\(S\) : a set of state \(S \subseteq Q\)

\(a \in \Sigma, w \in \Sigma^*\)

\(\epsilon\)-NFA:
\(S: (Q \times (\Sigma \cup \{\varepsilon\})) \to 2^Q\)

\(\delta^*(2^Q \times \Sigma^*) \to 2^Q\)

\(\delta^*(S, \epsilon) = E(S)\)

\(\delta^*(S,aw) = \delta^*(\bigcup_{q \in S} E(\delta(q,a)),w)\)

\begin{quote}
Simulating an \(\varepsilon\)-NFA

Let \(E(S)\) be the epsilon closure of a set of states \(S\). Recall
\(S \subset E(S)\)

\(w = a_1 a_2 \ldots a_n\)

\(S = E(\{q_0\})\)

\(S = \bigcup_{q \in S} \delta(q, a_i)\)

if \(S \cap A \ne \emptyset\) then Accept

Otherwise Reject
\end{quote}

\ex{

\(q_0\) is the start state

\(Q = \{q_0, q_1, q_2, q_3, q_4, q_5, q_6, q_7\}\)

\(A = \{q_4, q_7\}\)

Transitions: 
\begin{itemize}
    \item \(\delta(q_0, \varepsilon) \to q_1\)
    \item \(\delta(q_0, \varepsilon) \to q_5\)
    \item \(\delta(q_1, a) \to q_2\)
    \item \(\delta(q_2, b) \to q_3\)
    \item \(\delta(q_3, c) \to q_4\)
    \item \(\delta(q_5, a,b,c) \to q_5\)
    \item \(\delta(q_5, c) \to q_6\)
    \item \(\delta(q_6, c) \to q_7\)
\end{itemize}


Take the string \(w = abcaccc\)

\(S = E(\{q_0\} = \{q_0, q_1, q_5\}\)

\(S = E(\delta(q_0, a) = \emptyset \cup \delta(q_1, a) = \{q_2\} \cup \delta(q_5, a) = \{q_5\})\)

\(S = \{q_2, q_5\}\)}{}

\ex{

\(q_0\) is the start state

\(Q = \{q_0, q_1, q_2\}\)

\(A = \{q_2\}\)

Transitions: 
\begin{itemize}
    \item \(\delta(q_0, a) \to q_1\)
    \item \(\delta(q_1, a,b) \to q_1\)
    \item \(\delta(q_1, \varepsilon) \to q_2\)
    \item \(\delta(q_2, c) \to q_0\)
\end{itemize}


This machine represents the regular language \(a(a | b)^*\)

Take the string \(w = abca\)

\(S = E(\{q_0\}) = \{q_0\}\) ``a''

\(S = E( \delta(q_0, a)) = E(\{q_1\}) = \{q_1,q_2\}\)

Next symbol

\(S = \{q_1, q_2\}\) ``b''

\(S = E(\delta(q_1, b) \cup \delta(q_2, b)) = E(\{q_1\} \cup \emptyset) = \{q_1, q_2\}\)
since \(q_2\) is available form \(q_1\) via \(\varepsilon\) transitions.

Next symbol

\(S = \{q_1, q_2\}\) ``c''

\(S = E(\delta(q_1, c) \cup \delta(q_2, c)) = E(\emptyset \cup \{q_0\}) = \{q_0\}\)

Last symbol

\(S = \{q_0\}\) ``a''

\(S = E(\delta(q_0, a)) = E(\{q_1\}) = \{q_1, q_2\}\)

Since \(\{q_1, q_2\} \cap \{q_2\} \ne \emptyset\), accept.}{}

DFA \(\equiv\) NFA \(\equiv\) \(\varepsilon\)-NFA \(\equiv\) Regular
\(\equiv\) Regular Expression

If we can show that an \(\varepsilon\)-NFA exists for every regular
expression, then we have proved one direction of Kleene's. We can do
this by structural induction.

Regular Language

NFA that recognizes \(\emptyset\)

\(A = \{\}\)

\(q_0\) is our start state

\(Q = \{q_0\}\)

NFA that recognizes \(\{\varepsilon\}\)

\(A = \{q_0\}\)

\(q_0\) is our start state

\(Q = \{q_0\}\)

NFA that recognizes \(\{a\}\)

\(A = \{q_1\}\)

\(q_0\) is our start state

\(Q = \{q_0, q_1\}\)

Transitions:
\begin{itemize}
    \item \(\delta(q_0, a) \to q_1\)    
\end{itemize}

NFA that recognizes union \(L_1 \cup L_2\)

Connect start state \(q_0\) with first state of \(L_1\) and \(L_2\) via
\(\varepsilon\)

NFA that recognizes concatenation \(L_1 L_2\)

Connect start state \(q_0\) to start state of \(L_1\) with
\(\varepsilon\) transition. Connect the final state in \(L_1\) with the
first state in \(L_2\) via \(\varepsilon\) transition.

NFA that recognizes \(L^*\)

From the accepting state of \(L\), draw a \(\varepsilon\) transition
back to start state and (accepting state) \(q_0\). \(q_0\) has a
\(\varepsilon\) transition to the beginning of \(L\).

We have completed Scanning. Now onto syntax.

Motivating Example:

Consider \(\Sigma = \{(,)\}\) and
\(L = \{w : w \text{ is a balanced string of parentheses}\}\)

\qn{}{ Is this language regular? Can we build a DFA for L?
}

No.

Consider the regular attempt:

\((???) | \varepsilon\)

The \(???\) is the problem. What goes inside the parentheses is the
entire language of matched parentheses. What if we could recurse in
regular expressions?

\(L = (L) | \varepsilon\) (Note: this just covers \((((\ldots)))\), not
e.g., \(()()(())\))

In terms of power, context-free languages are exactly regular languages
plus recursion. In terms of expression, rather than extend regular
expressions, we have a different form called grammars.

\defn{}{ Definition Grammar is the language of languages.
}

Grammars help us describe what we ware allowed and not allowed to say.
Context-free grammars are a set of rewrite rules that we can use to
describe a language.

CFG for C++ has a lot of recursion.

\defn{}{ Definition A Context Free Grammar (CFG) is a 4 tuple
\((N, \Sigma, P, S)\) where 
\begin{itemize}
    \item \(N\) is a finite non-empty set of non-terminal symbols
    \item \(\Sigma\) is an alphabet; a set of non-empty terminal symbols.
    \item \(P\) is a finite set of productions, each of the form \(A \to \beta\) where \(A \in N\) and \(\beta \in (N \cup \Sigma)^*\)
    \item \(S \in N\) is a starting symbol
\end{itemize}

}

Note: We set \(V = N \cup \Sigma\) to denote the vocabulary, that is,
the set of all symbols in our language.

Conventions:

Lower case letters from the start of the alphabet, i.e., a, b, c,
\(\ldots\), are elements of \(\Sigma\).

Lower case letters from the end of the alphabet, i.e., w, x, y, z, are
elements of \(\Sigma^*\) (words)

Upper-case letters, i.e., A, B, C, \(\ldots\), are elements of \(N\)
(non-terminals)

\(S\) is always our start symbol.

Greek letters, i.e., \(\alpha, \beta, \gamma, \ldots\), are elements of
\(V^*\) (recall this is (\(N \cup \Sigma)^*\))

In most programming languages, the terminals of the context-free
languages are the tokens, which are the words in the regular language.

This is why scanners categorize tokens (e.g.~all infinity IDs are
``ID''): so that the CFL's alphabet is finite.

It is possible to define CFGs directly over the input characters: this
is called scannerless.

Let's revisit \(\Sigma = \{(,)\}\) and
\(L = \{w : w \text{ is a balance string of parentheses}\}\)

\(S \to \varepsilon, S \to (S), S \to SS\)

We can also write this using a shorthand: \(S \to \varepsilon|(S)| SS\)

Find a derivation of \((()())\). Recall our CFG above.

\defn{}{ Over a CFG \((N, \Sigma, P, S)\), we say
that\(\ldots\) 
\begin{itemize}
    \item \(A\) derives \(\gamma\) and we write \(A \implies \gamma\) if and only if there is a rule \(A \to \gamma\) in \(P\).
    \item \(\alpha A \beta \implies \alpha \gamma \beta\) if and only if there is a rule \(A \to \gamma\) in \(P\).
    \item \(\alpha \implies \gamma\) if and only if a derivation exists, that is, there exists \(\delta_i \in C^*\) for \(0 \le i \le k\) such that \(\alpha = \delta_0 \implies \delta_1 \implies \ldots \implies \delta_k = \gamma\). Note that \(k\) can be \(0\).
\end{itemize}

}

Solution:

\(S \implies (S) \implies (SS) \implies ((S)S) \implies (()S) \implies (()(S)) \implies (()())\)
Hence, \(S \implies (()())\)

\qn{}{ Why Context-Free?
}

Context-free languages actually add a sort of context to regular
languages, so why are they ``context free''?

They're free of a different sort of context. For instance, a
context-free language can't catch this:


\begin{minted}[frame=lines, linenos, fontsize=\large]
{cpp}
int a;
(*a) + 12;
\end{minted}

Need the context that \texttt{a} is an int to know that this isn't
allowed.

\defn{}{ Define the language of a CFG
(\(N, \Sigma, P, S\)) to be \(L(G) = \{w \in \Sigma^* : S \implies w\}\)
}

\defn{}{ A language is context-free if and only if there
exists a CFG \(G\) such that \(L = L(G)\)
}

Every regular language is context-free.

\begin{enumerate}
\def\labelenumi{\arabic{enumi}.}
\item
  \(\emptyset: (\{S\}, \{a\}, \emptyset, S)\)
\item
  \(\{\varepsilon\}: (\{S\}, \{a\}, S \to \varepsilon, S)\)
\item
  \(\{a\}: (\{S\}, \{a\}, S \to a, S)\)
\item
  Union: \(\{a\} \cup \{b\}: (\{S\}, \{a, b\}, S \to a | b, S)\)
\item
  Concatenation: \(\{ab\}: (\{S\}, \{a,b\}, S \to ab, S)\)
\item
  Kleene Star: \(\{a\}^*: (\{S\}, \{a\}, S \to Sa | \varepsilon, S)\)
\end{enumerate}

\section{Lecture 10}\label{lecture-10}

Practice

Let \(\Sigma = \{a,b\}\). Find a CFG for each of the following

\begin{itemize}

\item
  \(\{a^n b^n: n \in \text{natural numbers}\}\)
\end{itemize}

CFG: \(S \to \varepsilon | aSb\)

Derivation:

\(S \implies aSb \implies aaSbb \implies aaaSbbb \implies aaabbb\)

\begin{itemize}

\item
  Palindromes over \(\{a,b,c\}\)
\end{itemize}

\(S \to \varepsilon | a | b | aSa | bSb | cSc\)

\begin{itemize}

\item
  \(a(a | b) * b\)
\end{itemize}

\(S \to a T b\)

\(T \to MT | \varepsilon\)

\(M \to a | b\)

A fundamental example

Let's consider arithmetic operations over
\(\Sigma = \{a,b,c,+,-,*,/,(,)\}\) Find 
\begin{itemize}
    \item A CFG for \(L_1\): arithmetic expressions from \(\Sigma\) without parentheses, and a derivation for \(a - b\).
    \item A CFG for \(L_2\): Well-formed arithmetic expressions from \(\Sigma\) with balanced parentheses, and a derivation for \(((a) - b)\).
\end{itemize}


\(L_1\): Expression without parentheses

\(S \to a | b | c | SRS\)

\(R \to + | - | * | /\)

\(w = a - b\)

\(S \implies SRS \implies aRS \implies a - S \implies a - b\)

\(L_2\): Expression with parentheses

\(S \to a | b | c | (SRS)\)

\(R \to + | - | * | /\)

\(w = ((a))-b)\)

\(S \implies (SRS) \implies ((SRS)RS) \implies ((aRS)RS) \not\implies ((a) - b)\)
This does not work. Instead, we need

\(S \to a | b | c | SRS | (S)\)

\(R \to + | - | * | /\)

\(S \implies (S) \implies (SRS) \implies ((S)RS) \implies ((a)RS) \implies ((a)-S) \implies ((a)-b)\)
Notice, in these two derivations that we had a choice at each step which
element of \(N\) to replace.

\begin{itemize}
\item
  \(S \implies SRS \implies aRS \implies a-S \implies a-b\)
\item
  \(S \implies (S) \implies (SRb) \implies (S - b) \implies (a -b)\)
\end{itemize}

Leftmost derivation: In the first derivation, we chose to do a left
derivation, that is, one that always expands from the left first.

Rightmost derivation: In the second derivation, we chose to do a right
derivation, that is, one that always expands from the right first.

Parse Trees

\(w = aaabbb\)

\(S \to \varepsilon | aSb\)
\(S \implies aSb \implies aaSbb \implies aaaSbbb \implies aaa\varepsilon bbb \implies aaabbb\)


\qn{}{Is it possible for multiple leftmost derivations (or
multiple rightmost derivations) to describe the same string?
}

Consider two leftmost derivations for \(w = a - b * c\).

\(S \implies SRS \implies aRS \implies a - S \implies a - SRS \implies a - bRS \implies a - b * S \implies a - b * c\)


\(S \implies SRS \implies SRSRS \implies aRSRS \implies a - SRS \implies a - bRS \implies a - b * S \implies a - b * c\)


\defn{}{ A grammar for which some word has more than one
distinct leftmost derivation/rightmost derivation/parse tree is called
ambiguous.
}

\qn{}{ Why do we care about this?
}

As compiler writers, we care about where the derivation came from: parse
trees give meaning to the string with respect to the grammar.

Post-order traversal of trees pseudocode.

\begin{minted}[frame=lines, linenos, fontsize=\large]
{cpp}
t: TreeNode
int evaluate(t: Tree) {
    for each child in t {
        v_i = evaluate(child)
    } 
    values of children
    return compute(values)
}
\end{minted}

We use some sort of precedence heuristic to guide the derivation
process.

Or, make the grammar unambiguous. This is what we did without first
(incomplete) \(L_2\) (by adding parentheses).

\section{Lecture 11}\label{lecture-11}

There is a better way to eliminate ambiguity.

In a parse tree, we evaluate the expression in a depth-first, post-order
traversal.

We can make a grammar left/right associative by crafting the recursion
in the grammar.

Forcing Right Associative

\(S \to L R S | L\)

\(L \to a | b | c\)

\(R \to +| - | * | /\)

This forces a right-associative grammar (for \(a - b * c\), \(b * c\)
will evaluate first)

Forcing Left Associative

\(S \to SRL | L\)

\(L \to a | b | c\)

\(R \to + | - | * | /\)

This forces a left-associative grammar (recurses to the left).

We can use this to create a grammar that follows BEDMAS (by making
\(*,/\) appear further down the tree).

\(S \to SPT | T\)

\(T \to TRF | F\)

\(F \to a | b | c | (S)\)

\(P \to + | -\)

\(R \to * | /\)

\qn{}{If \(L\) is a context-free language, is there always an
unambiguous grammar such that \(L(G) = L\)?
}

No.~This was proven by Rohit Parikh in 1961.

\qn{}{ Can we write a computer program to recognize whether a
grammar is ambiguous?
}

No.~

\qn{}{ Given two CFGs \(G_1\) and \(G_2\), can we determine
whether \(L(G_1) = L(G_2)\). What about determining whether
\(L(G_1) \cap L(G_2) = \emptyset\)?
}

This is still undecidable.

We use parsers to handle CFLs and CFGs.

Formally: Given a CFG \(G = (N, \Sigma, P, S)\) and a terminal string
\(w \in \Sigma^*\), find the derivation, that is, the steps such that
\(S \implies \ldots \implies w\) or prove that \(w \not \in L(G)\)

We can find this with two broad ideas (top-down and bottom-up).

Top down parsing: Start with \(S\) and then try to get to \(w\).

Bottom-up Parsing: Start with \(w\) and work our way backwards to \(S\).

Top-Down Parsing

Start with \(S\) and look for a derivation that gets us closer to \(w\).
Then, repeat with remaining non-terminals until we're done.

The main trick is ``look'' for derivation. Thus, the core problem is to
predict which derivation is right.

We present the \(LL(1)\) algorithm (in practice, real compliers do not
use this).

Here is the pseudocode for a top-down parsing algorithm (before having
the knowledge of a predictor table)

\begin{minted}[frame=lines, linenos, fontsize=\large]
{cpp}
push S
for each 'a' in input do
	while top of stack is A in N do
		pop A
		ask an oracle to tell you which production A -> \gamma to use
		push the symbols in \gamma (rtl)
	end while
	// TOS is terminal
	if TOS is not 'a' then
		Reject
	else 
		pop 'a'
	end if
end for
Accept
\end{minted}

This has some problems.

\begin{enumerate}
\def\labelenumi{\arabic{enumi}.}
\item
  The oracle is not real
\item
  When we reach the end of input, we have no way of realizing we weren't
  done with the stack
\item
  The oracle should be able to tell us that no production matches at all
\end{enumerate}

\section{Lecture 12}\label{lecture-12}

We augment the grammar to begin and end with $\vdash$ and $\dashv$ to solve problem 2.

\(S' \to \mkern9mu \vdash S \dashv\)

\(S \to \mkern9mu LRS | L\)

\(S \to \mkern9mu a | b | c\)

\(R \to \mkern9mu + | - | * | /\)

We treat these symbols as BOF and EOF characters.

Now let's solve problem 1 and 3.

This oracle could try all possible productions (way too expensive).

Solution: Use a single symbol lookahead to determine where to go. (This
is a heuristic, doesn't always work).

We construct a predictor table to tell us where to go: given a
non-terminal on the stack and a next input symbol, what rule should we
use?

Always looking at a terminal input symbol, so figuring out how they
match is non-trivial.

But this is hard. Consider the following

\(0. \mkern9mu S' \to \mkern9mu \vdash S \dashv\)

\(1. \mkern9mu S \to \mkern9mu LRS\)

\(2. \mkern9mu S \to \mkern9mu L\)

\(3. \mkern9mu L \to \mkern9mu a | b | c\)

\(4. \mkern9mu R \to \mkern9mu + | - | * | /\)

How could we decide between 1 and 2 based on the next symbol? They both
start with \(L\), which is the same non-terminal, so either can start
with a, b, or c.~

This is a limitation of the top-down parsing algorithm.

We will look at a simpler grammar.

\(0. \mkern9mu S' \to \mkern9mu \vdash S \dashv\)

\(1. \mkern9mu S \to \mkern9mu LRS\)

\(2. \mkern9mu S \to \mkern9mu L\)

\(3. \mkern9mu L \to \mkern9mu a | b\)

\(4. \mkern9mu R \to \mkern9mu + | -\)

Let's look at \(w = \mkern9mu \vdash a + b + c \dashv\)


\(0. \mkern9mu S' \to \mkern9mu \vdash S \dashv\)

\(1. \mkern9mu S \to \mkern9mu L M\)

\(2. \mkern9mu L \to \mkern9mu I\)

\(3. \mkern9mu M \to \mkern9mu OS\)

\(4. \mkern9mu M \to \mkern9mu \varepsilon\)

\(5. \mkern9mu I \to \mkern9mu a\)

\(6. \mkern9mu I \to \mkern9mu b\)

\(7. \mkern9mu O \to \mkern9mu +\)

\(8. \mkern9mu O \to \mkern9mu -\)


Predictor Table of the simple grammar above.


\begin{table}[h]
    \centering
    \begin{tabular}{|c|c|c|c|c|c|c|} \hline 
         &  $\vdash$&  $\dashv$&  $a$&  $b$&  $+$& $-$\\ \hline 
         $S'$&  0&  &  &  &  & \\ \hline 
         $S$&  &  &  1&  1&  & \\ \hline 
         $L$&  &  &  2&  2&  & \\ \hline 
         $M$&  4&  &  &  &  3& \\ \hline 
         $I$&  &  &  5&  &  & \\ \hline 
 $O$& & & & & 7&8\\ \hline
    \end{tabular}
    \caption{Predictor Table}
    
\end{table}


If in the non-terminal on the left, and encounter the terminal symbol on
the top, pick the option of the number (see grammar above). Pop the
elements that match the input string and repeat until we have finished
(or encounter an error).

\defn{}{A grammar is called \(LL(1)\) if and only if each
cell of the predictor table contains at most one entry.
}

For an \(LL(1)\) grammar, don't need sets in our predict table.

\qn{}{ Why is it called \(LL(1)\)?
}

First \(L\): Scan left to right

Second \(L\): Leftmost derivations

Number of symbol lookahead: \(1\)

Not all grammars work with top-down parsing. Our simple grammar from
above is not \(LL(1)\) since more than one value is in a cell in the
predictor table.

Constructing the Lookahead Table

Our goal is the following function, which is our predictor table.

Predict\((A,a):\) production rule(s) that apply when \(A \in N\) is on
the stack, \(a \in \Sigma\) is the next input character.

To do this, we also introduce the following function, First(\(\beta\))
\(\subseteq \Sigma\): First(\(\beta\)): set of characters that can be
the first symbol of a derivation starting from \(\beta \in V^*\).

\(\beta \implies \ldots \implies ar\). Thus, \(a \in\) First(\(\beta)\)

More formally:

Predict\((A,a) = \{A \to \beta: a \in \text{First}(\beta)\}\)

First(\(\beta) = \{a \in \Sigma: \beta \implies ay, \text{for some } y \in V^*\}\)

Example of First

Recall the grammar from above with 8 steps.

First \((\vdash S \dashv) = \{\vdash\}\)

First \((L M) =\) First(\(L) =\) First(\(I) = \{a, b\}\). So to compute
First(\(LM\)) we first need First(\(I\)).

First (\(I) = \{a,b\}\)

First (\(O S) =\) First (\(O) = \{+, -\}\)

First (\(\varepsilon) = \{\}\)

First (\(a) = \{a\}\)

First (\(b) = \{b\}\)

First(\(+) = \{+\}\)

First (\(-) = \{-\}\)

The problem with
Predict\((A,a) = \{A \to \beta: \beta \implies \ldots \implies ay\), for
some \(\beta, y \in V^*\}\) is that it is possible that
\(A \implies \ldots \implies \varepsilon\). This would mean that the
\(a\) didn't come from \(A\) but rather some symbol after \(A\).

Example \(S' \implies \ldots \implies \vdash abc \dashv\)

\(0. \mkern9mu S' \to \mkern9mu \vdash S \dashv\)

\(1. \mkern9mu S \to AcB\)

\(2.\mkern9mu  A \to ab\)

\(3.\mkern9mu  A \to ff\)

\(4.\mkern9mu  B \to def\)

\(5. \mkern9mu B \to ef\)

\(6. \mkern9mu B \to \varepsilon\)

Notice that
\(S' \implies \vdash S \dashv \implies \vdash AcB \dashv \implies \vdash abcB \dashv \implies \vdash abc \dashv\)

In the top-down parsing algorithm, we reach \(\vdash abcB \dashv\) the
stack is \(\dashv B\) and remaining input is \(\dashv\).

We look at Predict(\(B, \dashv)\) which is empty since
\(\dashv \not \in\) First(\(B)\). Thus we reach an error.

We augment our predict table to include the elements that can follow a
non-terminal symbol, if it can reduce to \(\varepsilon\).

In this case, we need to include that \(\dashv \in\)
Predict(\(B, \dashv)\)

To correct this, we introduce two new functions.

Nullable(\(\beta)\): boolean function; for \(\beta \in V^*\) is true if
and only if \(\beta \implies \ldots \implies \varepsilon\)

Follow(\(A)\): for any \(A \in N\), this is the set of elements of
\(\Sigma\) that can come immediately after \(A\) in a derivation
starting from \(S'\)

\defn{}{ We say that a \(\beta \in V^*\) is nullable if
and only if Nullable(\(\beta)=\) true
}

Example of Follow

\(0. \mkern9mu S' \to \mkern9mu  \vdash S \dashv\)

\(1.\mkern9mu  S \to AcB\)

\(2.\mkern9mu  A \to ab\)

\(3. \mkern9mu A \to ff\)

\(4. \mkern9mu B \to def\)

\(5.\mkern9mu  B \to ef\)

\(6.\mkern9mu  B \to \varepsilon\)

Follow\((S') = \{\}\) (Always)

Follow(\(S) = \{\dashv\}\)

Follow(\(A) = \{c\}\)

Follow(\(B) = \{\dashv\}\)

\qn{}{ What happens with Predict\((A,a)\) if Nullable(\(A)\) =
false?
}

Follow\((A)\) is still some set of terminals but it won't be relevant
since we would need to consider what happens to First\((A)\) first.

Thus, the Follow function only matters if Nullable is true.

This motivates the following correct definition of our predictor table:

\defn{}{
Predict(\(A, a) = \{A \to \beta: a \in \text{First}(\beta)\} \cup \{A \to \beta: \text{Nullable}(\beta) \text{ and } a \in \text{Follow(A)}\}\)
}

This is the full, correct definition. Notice that this still requires
that the table only have one member of the set per entry to be useful as
a deterministic program.

Note that Nullable\((\beta) =\) false whenever \(\beta\) contains a
terminal symbol.

Further, Nullable(\(AB) =\) Nullable(\(A)\) \(\wedge\) Nullable(\(B\))

Thus, it suffices to compute Nullable(\(A\)) for all \(A \in N\).

\section{Lecture 13}\label{lecture-13}

Algorithm of Nullable(\(A\))

\begin{minted}[frame=lines, linenos, fontsize=\large]
{cpp}
Initialize Nullable(A) = false for all A \in N'
repeat
for each production in P do:
	if (P is A -> \varepsilon) or 
        (P is A -> B_1 ... B_k 
        and \wedge_{i=1}^{k} Nullable(B_i) = true) then
	   Nullable(A) = true
	end if
end for
until nothing changes
\end{minted}

Example of Nullable

\(0.\mkern9mu S' \to\mkern9mu \vdash S \dashv\)

\(1. \mkern9muS \to c\)

\(2. \mkern9mu S\to QRS\)

\(3.\mkern9mu Q \to R\)

\(4.\mkern9mu Q \to d\)

\(5.\mkern9mu R \to \varepsilon\)

\(6. \mkern9muR \to b\)

Nullability Table


\begin{table}[h]
    \centering
    \begin{tabular}{|c|c|c|c|c|} \hline 
        Iter & 0 & 1 & 2 & 3\\ \hline 
         $S'$& F & F & F & F\\ \hline 
         $S$& F &  F& F & F\\ \hline 
         $Q$& F & F & T &T \\ \hline 
         $R$& F & T &T  & T\\ \hline
    \end{tabular}
    \caption{Nullability Table}
    
\end{table}


Thus, Nullable(\(S')\) = Nullable(\(S) = F\) and Nullable(\(Q) =\)
Nullable(\(R) = T\)

Notes about First

Main idea: Keep processing \(B_1, B_2, \ldots, B_k\) from a production
rule until you encounter a terminal or a symbol that is not nullable.
Then go to the next rule. Repeat until no changes are made during the
processing.

Remember, \(\varepsilon\), isn't a real symbol, and can't be in a First
set.

For First, we will ignore trivial productions of the form
\(A \to \varepsilon\) based on the above observation.

Further, First(\(S') = \{\vdash\}\) always.

We first compute First(\(A\)) for all \(A \in N\) and then we compute
First(\(\beta)\) for all relevant \(\beta \in V^*\)

Computing First
\begin{minted}[frame=lines, linenos, fontsize=\large]
{cpp}
Initialize First(A) = {} for all A \in N'
repeat
for each rule A -> B_1B_2 ... B_k in P do
	for i in {1,..., k} do
		if B_i in T' then
			First(A) = First(A) \cup {B_i}; break
		else
			First(A) = First(A) \cup First(B_i)
			if Nullable(B_i) == False then break
		end if
	end for
end for
until nothing changes
\end{minted}

Example of First

\(0.\mkern9mu S' \to \vdash S \dashv\)

\(1.\mkern9mu S \to c\)

\(2.\mkern9mu S\to QRS\)

\(3.\mkern9mu Q \to R\)

\(4. \mkern9muQ \to d\)

\(5. \mkern9muR \to \varepsilon\)

\(6. \mkern9muR \to b\)

First Table


\begin{table}[h]
    \centering
    \begin{tabular}{|c|c|c|c|c|} \hline 
        $Iter$ & 0 & 1 & 2 & 3\\ \hline 
        $S'$ & $\{\}$ & $\{\vdash\}$ & $\{\vdash\}$ & $\{\vdash\}$\\ \hline 
        $S$ & $\{\}$ & $\{c\}$ & $\{b,c,d\}$ &$\{b,c,d\}$ \\ \hline 
        $Q$ & $\{\}$ & $\{d\}$ &$ \{b,d\}$ & $\{b,d\}$\\ \hline 
        $R$ & $\{\}$ & $\{b\}$ & $\{b\}$ & $\{b\}$\\ \hline
    \end{tabular}
    \caption{First Table}
    
\end{table}

Recall, Nullable(\(S')\) = Nullable(\(S) = F\) and Nullable(\(Q) =\)
Nullable(\(R) = T\)

Thus, First(\(S') = \{\vdash\}\), First(\(S) = \{b,c,d\}\),
First(\(Q) = \{b,d\}\), First(\(R) = \{b\}\)

Computing First (2)
\begin{minted}[frame=lines, linenos, fontsize=\large]
{cpp}
result = {}
for i \in {1,...,n} do
	if B_i \in T' then
		result = result \cup {B_i}; break
	else 
		result = result \cup First(B_i)
		if Nullable(B_i) == False then break
	end if
end for
\end{minted}

Computing Follow
\begin{minted}[frame=lines, linenos, fontsize=\large]
{cpp}
Initialize Follow(A) = {} for all A \in N
repeat
    for each production A -> B_1 B_2 ... B_k in P' do
        for i \in {i, ..., k} do
            if B_i \in N then
                Follow(B_i) = Follow(B_i) \cup First(B_{i+1} ... B_k)
                if \wedge_{m=i+1}^{k} Nullable(B_m) == True or i == k then
                    Follow(B_i) = Follow(B_i) \cup Follow(a)
		end if
	   end if
	end for
    end for
until nothing changes
\end{minted}


Example of Follow

\(0. \mkern9muS' \to\mkern9mu \vdash S \dashv\)

\(1. \mkern9muS \to c\)

\(2.\mkern9mu S\to QRS\)

\(3. \mkern9muQ \to R\)

\(4. \mkern9mu Q \to d\)

\(5.\mkern9mu R \to \varepsilon\)

\(6. \mkern9muR \to b\)

Follow Table

\begin{table}[h]
    \centering
    \begin{tabular}{|c|c|c|c|} \hline 
        $Iter$ & 0 & 1 & 2\\ \hline 
        $S$ & $\{\}$ & $\{\vdash\}$ & $\{\vdash\}$\\ \hline 
        $Q$ & $\{\}$ & $\{b,c,d\}$ &$ \{b,c,d\}$ \\ \hline 
        $R$ & $\{\}$ & $\{b,c,d\}$ & $\{b,c,d\}$\\ \hline
    \end{tabular}
    \caption{Follow Table}
    
\end{table}


\section{Lecture 14}\label{lecture-14}

Cheat sheet

Nullable: 

\begin{itemize}
    \item \(A \to \varepsilon\) implies that Nullable(\(A) =\) true. Further Nullable(\(\varepsilon) = true\)
    \item If \(A \to B_1 \ldots B_n\) and each of Nullable(\(B_i) = true\) then Nullable(\(A) = true\)
\end{itemize}


First: 
\begin{itemize}
    \item \(A \to \alpha \alpha\) then \(a \in\) First(\(A\))
    \item \(A \to B_1 \ldots B_n\) then First(\(A\)) \(=\) First(\(A\)) \(\cup\)
First(\(B_i\)) for each \(i \in \{1, \ldots, n\}\) until
Nullable(\(B_i\)) is false
\end{itemize}

\begin{itemize}
    \item \(A \to \alpha B \beta\) then Follow(\(B\)) = First(\(\beta\))
    \item \(A \to B \beta\) and Nullable(\(\beta\)) \(= true\), then
Follow(\(B\)) = Follow(\(B\)) \(\cup\) Follow(\(A\))
\end{itemize}

Primary Issue

With this grammar:

\(S \to S + T\) \(S \to T\) \(T \to T * F\)

\(T \to F\)

\(F \to a | b | c | (S)\)

The primary issue is that left recursion is at odds with \(LL(1)\). In
fact, left recursive grammars are always not \(LL(1)\). Examine the
derivations for \(a\) and \(a + b\).

\(S \implies S + T \implies T + T \implies F + T \implies a + T \implies a + b S \implies T \implies F \implies a\)

Notice that they have the same first character but required different
starting rules from \(S\). That is \(\{1,2\} \subseteq\)
Predict(\(S, a)\). Our first step is to at least make this right
recursive.

To make a left recursive grammar right recursive; say

\(A \to A \alpha | \beta\)

where \(\beta\) does not begin with the non-terminal \(A\), we remove
this rule from our grammar and replace it with:

\(A \to \beta A'\)

\(A' \to \alpha A' | \varepsilon\)

The above solves our issue

\(S \to T Z'\)

\(Z' \to +TZ' | \varepsilon\)

\(T \to F T'\)

\(T' \to * FT' | \varepsilon\)

\(F \to a | b | c | (S)\)

we get a right-recursive grammar. This is \(LL(1)\).

However: recall that we didn't want these grammars, because they were
right associative. This is an issue we need to resolve with new
techniques.

Not all right right recursive grammars are \(LL(1)\). Consider

\(S \to T + S\)

\(S \to T\)

\(T \to F*T\)

\(T \to F\)

\(F \to a | b | c | (S)\)

we get a right-recursive grammar, but not \(LL(1)\)

\(S \implies T + S \implies F + S \implies a + S \implies a + T \implies a + b S \implies T \implies F \implies a\)

Again, we have \(\{1,2\} \subseteq\) Predict(\(S,a)\). There is still
hope, we can apply a process known as factoring.

Left factoring

Idea: If \(A \to \alpha \beta_1 | \ldots \alpha \beta_n | y\) where
\(\alpha \ne \varepsilon\) and \(y\) is representative of other
productions that do not begin with \(\alpha\), then we can change this
to the following equivalent grammar by left factoring:

\(A \to \alpha B | y\)

\(B \to \beta_1 | \ldots | \beta_n\)

Recursive-Descent Parsing

Fixing the parse trees from right-recursive and left-factored grammars
is the \#1 thing that recursive-descent ad hoc solutions fix

The actual sequence of steps is \(LL(1)\), but then they generate a
different parse tree by changing it on the fly.

Bottom-up Parsing

Recall: Determining the \(\alpha_i\) in
\(S \implies \alpha_1 \implies \ldots \implies w\)

Idea: Instead of going from \(S\) to \(w\), let's try to go from \(w\)
to \(S\). Overall idea: look for the RHS of a production, replace it
with the LHS. When you don't have enough for a RHS, read more input.
Keep grouping until you reach the start state.

Our stack this time will store the \(\alpha_i\) in reverse order
(Contrast to top-down which stores the \(\alpha_i\) in order)

Our invariant here will be Stack \(+\) Unread Input \(= \alpha_i\)
(Contrast to top-down where invariant was consumed input \(+\) reversed
Stack contents \(= \alpha_i)\)

\begin{minted}[frame=lines, linenos, fontsize=\large]
{cpp}
for each symbol $a$ in the input from left to right do
	// ask an oracle whether to shift, reduce, or reject,
	// and with which production to reduce (if we reduce)
	while the oracle tells us to reduce with some B -> \gamma do
		stack.pop symbols in \gamma
		stack.push $B$
	end while
	if the oracle told us to reject then
		reject
	end if
	stack.push a
end for
accept
\end{minted}


\ex{

Recall our grammar:

\(S' \to \mkern9mu \vdash S \dashv \quad (0)\)

\(S \to AcB  \quad (1)\)

\(A \to ab  \quad (2)\)

\(A \to ff  \quad (3)\)

\(B \to def  \quad (4)\)

\(B \to ef  \quad (5)\)}{}

We wish to process \(w = \mkern9mu \vdash a b c d e f \dashv\)


\begin{table}[H]
    \centering
    \begin{tabular}{|c|c|c|c|} \hline 
         Stack&  Read&  Processing& Action\\ \hline 
         &  $\varepsilon$&  $\vdash a b c d e f \dashv$& Shift $\vdash$\\ \hline 
         $\vdash$&  $\vdash$&  $a b c d e f \dashv$& Shift $a$\\ \hline 
         $\vdash a$&  $\vdash a$&  $b c d e f \dashv$& Shift $b$\\ \hline 
         $\vdash ab$&  $\vdash ab$&  $c d e f \dashv$& Reduce ($2$); pop $b$, $a$, push $A$\\ \hline 
         $\vdash A$&  $\vdash a b$&  $cdef \dashv$& Shift $c$\\ \hline 
         $\vdash Ac$&  $\vdash a b c$&  $def \dashv$& Shift $d$\\ \hline 
         $\vdash Acd$&  $\vdash a b c d$&  $ef \dashv$& Shift $e$\\ \hline 
         $\vdash Acde$&  $\vdash a b c d e$&  $f \dashv$& Shift $f$\\ \hline 
         $\vdash Acdef$&  $\vdash a b c d e f$&  $\dashv$& Reduce ($4$); pop $f, d, e$ push $B$\\ \hline 
         $\vdash AcB$&  $\vdash a b c de f$&  $\dashv$& Reduce ($1$); pop $B, c, A$ push $S$\\ \hline 
         $\vdash S$&  $\vdash a b c d e f$&  $\dashv$& Shift $\dashv$\\ \hline 
         $\vdash S \dashv$&  $\vdash a b c d e f \dashv$&  $\varepsilon$& Reduce $(0)$; pop $\dashv S, \vdash$ push $S'$\\ \hline 
         $S'$&  $\vdash abcdef \dashv$&  $\varepsilon$& Accept\\ \hline
    \end{tabular}
    
\end{table}





\thm{}{For any grammar \(G\), the set of viable prefixes
(stack configurations), namely \(\{\alpha a: \alpha \in V^*\) is a stack
\(a \in \Sigma\) is the next character \(\exists x \in \Sigma^*\) such
that \(S \implies \ldots \implies \alpha a x\}\) is a regular language,
and the NFA accepting it corresponds to items of \(G\). Converting this
NFA to a DFA gives a machine with states that are set of valid items for
a viable prefix.
}

We will show how to use this theorem to create a \(LR(0)\), \(SLR(1)\),
and \(LR(1)\) automata to help us accept the words generated by a
grammar.

Consider the following context-free grammar:

\(S' \to \mkern9mu \vdash S \dashv \quad (0)\)

\(S \to S + T  \quad (1)\)

\(S. \to T  \quad (2)\)

\(T. \to d  \quad (3)\)

\defn{}{ An item is a production with a dot \(\cdot\)
somewhere on the right hand side of a rule
}

Items indicate a partially completed rule. We will begin in a state
labelled by the rule \(S' \to \cdot \vdash S \dashv\)

That dot is called the bookmark

\(LR(0)\) Construction

From a state, for each rule in the state, move the dot forward by one
character. The transition function is given by the symbol you jumped
over.

For example, with \(S' \to \cdot \vdash S \dashv\), we move the
\(\cdot\) over \(\vdash\). Thus, the transition function will consume
the symbol \(\vdash\).

The state we end up in will contain the item
\(S' \to\mkern9mu \vdash \cdot S \dashv\)

In the new state, if the set of items we have \(\cdot A\) for some
non-terminal \(A\), we then add all rules with \(A\) in the left-hand
side of a production with a dot preceding the right-hand side.

In this case, this state will include the rules \(S \to \cdot S + T\)
and \(S \to \cdot T\).

Notice now we also have \(\cdot T\) and so we also need to include the
rules where \(T\) is the left-hand side, adding the rule
\(T \to \cdot d\).

If we find ourselves at a familiar state, reuse it instead of remaking
it.

We continue with these steps until there are no bookmarks left to move.
Then we have the final DFA.

We skipped the \(\varepsilon-\)NFA step by putting all these items in
the same rule. You may see versions of this algorithm that involve
building an \(\varepsilon-\)NFA and then converting, but the result will
be the same.

Back to the example

\(S' \to \mkern9mu \vdash S \dashv  \quad (0)\)

\(S \to S + T  \quad (1)\)

\(S. \to T  \quad (2)\)

\(T. \to d  \quad (3)\)

This automaton is our oracle. Run the stack through the automaton, and:
\begin{itemize}
    \item If you end up in a state with the bookmark at the right-hand side of an item, perform that reduction 
    \item If you end up in a state with the
bookmark elsewhere, shift 
    \item Else (error state), reject
\end{itemize}


\section{Lecture 15}\label{lecture-15}

\(w = \vdash d + d + d \dashv\)


\begin{table}[H]
    \centering
    \begin{tabular}{|c|c|c|c|} \hline 
         Stack&  Read&  Processing& Action\\ \hline 
         &  $\varepsilon$&  $\vdash d + d  + d \dashv$& Shift state 0\\ \hline 
         $\vdash$&  $\vdash$&  $d + d + d \dashv$& State 1 Shift\\ \hline 
         $\vdash d$&  $\vdash d$&  $+ d + d \dashv$& State 5 reduce(3), pop d, push T\\ \hline 
         $\vdash T$&  $\vdash d$&  $+ d + d \dashv$& State 4 reduce(2), pop T, push S\\ \hline 
         $\vdash S$&  $\vdash d$&  $+ d + d \dashv$& State 2, shift\\ \hline 
         $\vdash S \dashv$&  $\vdash d +$&  $d + d \dashv$& State 6, shift\\ \hline 
         $\vdash S + d$&  $\vdash d + d$&  $d \dashv$& State 5, reduce (3), pop d, push T\\ \hline 
         $\vdash S + T$&  $\vdash d + d$&  $+ d \dashv$& State 7, reduce(1) pop S and T, push S\\ \hline 
         $\vdash S$&  &  & \\\hline
    \end{tabular}
    
\end{table}

The stack is a stack, so the bottom of the stack (beginning of our
input) doesn't usually change. We're rerunning the whole DFA even when
the prefix of our stack is the same. Because of this, our algorithm is
\(O(n^2)\).

Remember how we moved through the DFA in a state stack, and push and pop
to the state stack at the same time as the symbol stack. That way, we
don't repeat getting to a state with a prefix that hasn't changed.

This brings us to \(O(n)\).

Stack column above becomes Symbol Stack

\(LR(0)\)


\begin{minted}[frame=lines, linenos, fontsize=\large]
{cpp}
stateStack.push q_0
for each symbol a in EOF x BOF from left to right do
	while Reduce[stateStack.top] is some production B -> \gamma do
		symStack.pop symbols in \gamma
		stateStack.pop |\gamma| states
		symStack.push B
		stateStack.push \delta[stateStack.top, B]
	end while
	symStack.push a
	reject if \delta[stateStack.top, a] is undefined
	stateStack.push \delta[stateStack.top, a]
end for
accept
\end{minted}



\begin{table}[H]
    \centering
    \begin{tabular}{|l||c|c|c|c|} \hline 
          State Stack&Symbol Stack&  Read&  Processing& Action\\ \hline 
          $q_0$&&  $\varepsilon$&  $\vdash d + d  + d \dashv$& Shift state 0\\ \hline 
          $q_0, q_1$&$\vdash$&  $\vdash$&  $d + d + d \dashv$& State 1 Shift\\ \hline 
          $q_0, q_1, q_5$&$\vdash d$&  $\vdash d$&  $+ d + d \dashv$& State 5 reduce(3), pop d, push T\\ \hline 
          $q_0, q_1, q_4$&$\vdash T$&  $\vdash d$&  $+ d + d \dashv$& State 4 reduce(2), pop T, push S\\ \hline 
          &$\vdash S$&  $\vdash d$&  $+ d + d \dashv$& State 2, shift\\ \hline 
          &$\vdash S \dashv$&  $\vdash d +$&  $d + d \dashv$& State 6, shift\\ \hline 
          &$\vdash S + d$&  $\vdash d + d$&  $d \dashv$& State 5, reduce (3), pop d, push T\\ \hline 
          &$\vdash S + T$&  $\vdash d + d$&  $+ d \dashv$& State 7, reduce(1) pop S and T, push S\\ \hline 
          &$\vdash S$&  $\vdash a b c d e f$&  & \\\hline
    \end{tabular}
    
\end{table}

Possible Issues

Issue one (Shift-Reduce): What if a state has two items of the form: 
\begin{itemize}
    \item \(A \to \alpha \cdot a \beta\)
    \item \(B \to \gamma \cdot\)
\end{itemize}


\qn{}{Should we shift or reduce?
}

Note, having two items that shift, e.g.:
\begin{itemize}
    \item \(A \to \alpha \cdot a \beta\)
    \item \(B \to \gamma \cdot b \delta\)
\end{itemize}

is not an issue.

\defn{}{ A grammar is \(LR(0)\) if and only if after
creating the automaton, no state has a shift-reduce or reduce-reduce
conflict.
}

Recall that \(LL(1)\) grammars were at odds with recursive languages.

\qn{}{ Are \(LR(0)\) grammars in conflict with a type of
recursive language?
}

Not usually. Bottom-up parsing can support left and right recursive
grammars. However not all grammars are \(LR(0)\) grammars.

Consider the grammar

\(S' \to\mkern9mu \vdash S \dashv\)

\(S \to T + S\)

\(S. \to T\)

\(T. \to d\)

New \(LR(0)\) Automaton

Conflict

State 4 has a shift-reduce conflict.

Suppose the input began with \(\vdash d\). This gives a stack of
\(\vdash d\) and then we reduce in state 5, so our stack changes to
\(\vdash T\) and we move to state 4 via state 1.

Should we reduce \(S \to T?\) It depends.

We add a lookahead to the automation to fix the conflict. For every
\(A \to \alpha^*\), attack Follow(\(A\)). Recall:

\(S' \to \mkern9mu\vdash S \dashv\)

\(S \to T + S\)

\(S. \to T\)

\(T. \to d\)

Note that Follow(\(S\))\(= \{\dashv\}\) and Follow(\(T\))
\(= \{+, \dashv\}\). So state 4 becomes

\(S \to T \cdot + S\) and \(S \to T \cdot :\{\dashv\}\)

Apply \(S \to T \cdot +S\) if the next token is \(+\), and apply
\(S \to T \cdot \{\dashv\}\) if the next token is \(\dashv\).

With lookahead from Follow sets on reduce states, we call these parses
\(SLR(1)\) parsers.

\(SLR(1)\) is not a simplified version of \(LR(1)\), it's just
different.

Building the Parse Tree

With top-down parsing, when we pop \(S\) from the stack and push
\(B, y\) and \(A: S\) is a node, make the new symbols the children.

With bottom-up parsing, when you reduce \(A \to ab\) (from a stack with
\(a\) and \(b\)). You then keep these two old symbols as children of the
new node \(A\).

A Last Parser Problem

Most famous problem in parsing: the dangling else.

\texttt{if(a)\ if\ (b)\ S1:\ else\ S2}

\begin{minted}[frame=lines, linenos, fontsize=\large]
{cpp}
if (a) {
    if (b) {
        S1;	
    }
} else {
    S2;
}
\end{minted}

\begin{minted}[frame=lines, linenos, fontsize=\large]
{cpp}
if(a) {
    if (b) {
        S1;	
    } else {
        S2;
    }
}
\end{minted}

\section{Lecture 16}\label{lecture-16}

Where are we now?

We have finished syntactic analysis, are now on to type checking
(semantic analysis).

Semantics: Does what is written make sense?

Not everything can be enforced by a CFG. Examples
\begin{itemize}
    \item Type checking
    \item Declaration before use
    \item Scoping
    \item Well-typed expressions
\end{itemize}

To solve these, we can move to context-sensitive languages.

As it turns out, CSL's aren't a very useful formalism.

We already needed to give up many CFGs to make a parser handle CFLs;
with CSLs, it would be even worse.

As such, we treat context-sensitive analysis as analysis (looking over
the parse tree generated by CFL parsing) instead of parsing (making its
own data structure).
\begin{itemize}
    \item pre-order tree traversal
    \item post-order tree
traversal
    \item hybrid of the two
\end{itemize}

We will traverse our parse tree to do our analysis.

We still need to check for:
\begin{itemize}
    \item Variables declared more than once
    \item Variables used but not declared
    \item Type errors
    \item Scoping as it applies to
the above
\end{itemize}

Declaration errors

\qn{}{How do we determine multiple/missing declaration errors?
}

We've done this before. We construct a symbol table.
\begin{itemize}
    \item Traverse the
parse tree for any rules of the form dcl \(\to\) type ID.
    \item Add the ID to the symbol table
    \item If the name is already in the table, give an
error.
\end{itemize}
 

\begin{minted}[frame=lines, linenos, fontsize=\large]
{cpp}
int foo(int a) {
	int x = 0;
	return x + a;
}

int wain(int x, int y) {
	return foo(y) + x;
}
\end{minted}

Checking

To verify that variables have been declared. Check for rules of the form
factor \(\to\) ID, and lvalue \(\to\) ID. If ID is not in the symbol
table, produce an error. The previous two passes can be merged.

def: dcl \(\to\) type ID (make sure not already in symbol table, and add
to symbol table)

Use of variables:

factor \(\to\) ID (check to see if it is in symbol table (return error
if not))

lvalue \(\to\) ID (check to see if it is in symbol table (return error
if not))

\qn{}{ With labels in MIPS in the assembler, we needed two
passes. Why do we only need one in the compiler?
}

We need to declare variables before using them. This is not true for
labels.

Note that in the symbol table, we should also keep track of the type of
variables. Why is this important? Just by looking at the bits, we cannot
figure out what it represents. Types allow us to interpret the contents.

Good systems prevent us from interpreting bits as something we should
not.

For example

\begin{minted}[frame=lines, linenos, fontsize=\large]
{cpp}
int *a = NULL;
a = 7;
\end{minted}

should be a type mismatch.

This is just a matter of interpretation.

In WLP4, there are two types: \texttt{int} and \texttt{int*} for
integers and pointers to integers.

For type checking, we need to evaluate the types of expressions and then
ensure that the operations we use between types corresponds correctly.

\qn{}{ If given a variable (in the wild), how do we determine
it's type?
}

Use its declaration. Need to add this to the symbol table.

We can use a global variable to keep track of the symbol table:

\begin{minted}[frame=lines, linenos, fontsize=\large]
{cpp}
enum class Type {INT, POINTER};
unordered_map<string, Type> symbolTabel; // name->type
\end{minted}


Some things can go wrong. This doesn't take scoping into account. Also
need something for functions/declarations.

Consider the following code. Is there an error?

\begin{minted}[frame=lines, linenos, fontsize=\large]
{cpp}
int foo(int a) {
	int x = 0;
	return x + a;
}
int wain(int x, int y) {
	return foo(y) + x;
}
\end{minted}

No.~Duplicate variables in different procedures are okay.

Is the following an error?

\begin{minted}[frame=lines, linenos, fontsize=\large]
{cpp}
int foo(int a) {
	int x = 0;
	return x + a;
}
int wain(int a, int b) {
	return foo(b) + x;
}

\end{minted}

Yes. The variable \texttt{x} is not in scope in \texttt{wain}.

Is the following an error?

\begin{minted}[frame=lines, linenos, fontsize=\large]
{cpp}
int foo(int a) {
	int x = 0;
	return x + a;
}
int foo(int b) {return b;}
int wain (int a, int b) {
	return foo(b) + a;
}
\end{minted}

Yes. We have multiple declarations of \texttt{foo}.

We resolve this with a separate symbol table per procedure. We also need
a global symbol table.

\qn{}{ A symbol table is a map, what are we mapping each symbol
to?
}

For type checking, we need to know the type of each symbol. In WLP4, a
string could be sufficient, but you should use a data structure so you
can add more later.

\qn{}{What about procedures?
}

Procedures don't have types, they have signatures.

In WLP4, all procedures return \texttt{int}, so we really just need the
argument types: an array of types.

\begin{minted}[frame=lines, linenos, fontsize=\large]
{cpp}
int foo(int x, int y)
// (int, int) -> int
int bar(int *x, int y, int c)
// (int*, int, int) -> int
\end{minted}

Computing the signature.

Simply need to traverse nodes in the parse tree of these forms.

\begin{itemize}
\item
  params \(\to\)
\item
  params \(\to\) paramlist
\item
  paramlist \(\to\) dcl
\item
  paramlist \(\to\) dcl COMMA paramlist
\end{itemize}

This can be done in a single pass.

Consider

\begin{minted}[frame=lines, linenos, fontsize=\large]
{cpp}
int foo(int a) {
	int x = 0;
	return x + a;
}
int wain(int *a, int b) {
	return foo(b) + a;
}
\end{minted}

Global symbol table:
\begin{itemize}
    \item \texttt{foo:\ {[}int{]},\ wain:\ {[}int*,\ int{]}}
\end{itemize}

Local symbol tables: 
\begin{itemize}
    \item \texttt{foo:\ a:\ int,\ x:\ int}
    \item \texttt{wain:\ a:\ int*,\ b:\ int}
\end{itemize}

Type errors

\qn{}{ What are type errors and how to find them?
}

Two separate issues:
\begin{itemize}
    \item What are type errors? (Definition)
    \item  How to find them? (Implementation)
\end{itemize}

Definition of Type (Errors)

Need a set of rules to tell us 

\begin{itemize}
    \item The type of every expression
    \item Whether
an expression makes sense with the types of its subexpressions
    \item Whether
a statement makes sense with the types of its subexpressions
\end{itemize}

Detection of Type (Errors)

There's really only one algorithm with a tree: traverse the tree.
Implement a (mostly) post-order traversal that applies defined rules
based on which expressions it encounters.

Inference rules are Post rules (like in CS245)

If an ID is declared with type \(\tau\) then it has this type:

\begin{align*}
\frac{(\text{id.name}, \tau) \in \text{declarations}}{\text{id.name} : \tau}
\end{align*}

Numbers have type \texttt{int}: \(\overline{\text{NUM: int}}\)

NULL is of type \texttt{int*}: \(\overline{\text{NULL: int*}}\)

Inference rules for types

Inference rules are the true case. If no inference rule matches, that
means the expression or statement doesn't type check: type error.

Look for good, not for bad: errors should always be the ``else'' case.

Parentheses do not change the type \(\frac{E: \tau}{(E) : \tau}\)

The Address of an \texttt{int} is of type \texttt{int*}
\(\frac{E : \text{int}}{\&E: \text{int*}}\)

\begin{minted}[frame=lines, linenos, fontsize=\large]
{cpp}
int x = 0;
int *p = NULL;
p &x; // int*
\end{minted}

Dereferencing \texttt{int*} is of type \texttt{int}
\(\frac{E: \text{int*}}{*E: \text{int}}\)

\begin{minted}[frame=lines, linenos, fontsize=\large]
{cpp}
x = *p // int
\end{minted}

If \(E\) has type \texttt{int} then \texttt{new\ int{[}E{]}} is of type
\texttt{int*} \(\frac{E: int}{\text{new int[E]: int*}}\)

\begin{minted}[frame=lines, linenos, fontsize=\large]
{cpp}
p = new int[*p] // int*
\end{minted}

Arithmetic Operations

Multiplication
\(\frac{E_1: \text{int} \mkern9mu E_2: \text{int}}{E_1 * E_2: \text{int}}\)

Division
\(\frac{E_1: \text{int} \mkern9mu E_2: \text{int}}{E_1 / E_2: \text{int}}\)

Module
\(\frac{E_1: \text{int} \mkern9mu E_2: \text{int}}{E_1 \% E_2: \text{int}}\)

Addition 
\begin{align*}
\frac{E_1: \text{int} \mkern9mu E_2: \text{int}}{E_1 + E_2: \text{int}} \\ \\
\frac{E_1: \text{int*} \mkern9mu E_2: \text{int}}{E_1 + E_2: \text{int*}} \\ \\
\frac{E_1: \text{int} \mkern9mu E_2: \text{int*}}{E_1 + E_2: \text{int*}}
\end{align*}


Subtraction 
\begin{align*}
\frac{E_1: \text{int} \mkern9mu E_2: \text{int}}{E_1 - E_2: \text{int}} \\ \\
\frac{E_1: \text{int*} \mkern9mu E_2: \text{int}}{E_1 - E_2: \text{int*}} \\ \\
\frac{E_1: \text{int*} \mkern9mu E_2: \text{int*}}{E_1 - E_2: \text{int}}
\end{align*}


Procedure Calls:

\begin{align*}
\frac{(f, \tau_1, \ldots, \tau_n) \in \text{declarations} \quad E_1: \tau_1 \mkern9mu E_2: \tau_2 \mkern9mu \ldots \mkern9mu E_n: \tau_n}{f(E_1, \ldots, E_n): \text{int}}
\end{align*}


The basic kind of statement type is an expression statement. An
expression statement is okay as long as the expression has a type. We
will need rules for all the other statements too. Statement's don't have
a type, but can be ``well typed''.

\section{Lecture 17}\label{lecture-17}

Control statements:

\begin{minted}[frame=lines, linenos, fontsize=\large]
{cpp}
while (T) {S}
if (T) {S_1} else {S_2}
\end{minted}

The value of \texttt{T} above should be a boolean. But WLP4 doesn't have
booleans.

Our grammar forces it to be a boolean expression, so we don't need to
check that.

But, we still need to check its subexpressions.

Inference rules for well-typed

\(\frac{E_1 : \tau \quad E_2 : \tau}{\text{well-typed}(E_1 < E_2)}\)

\(\frac{E_1 : \tau \quad E_2 : \tau}{\text{well-typed}(E_1 > E_2)}\)

\(\frac{E_1 : \tau \quad E_2 : \tau}{\text{well-typed}(E_1 == E_2)}\)

\(\frac{E_1 : \tau \quad E_2 : \tau}{\text{well-typed}(E_1 <= E_2)}\)

\(\frac{E_1 : \tau \quad E_2 : \tau}{\text{well-typed}(E_1 >= E_2)}\)

\(\frac{E_1 : \tau \quad E_2 : \tau}{\text{well-typed}(E_1 = E_2)}\)

An if statement is well-typed if and only if all of its components are
well typed

\(\frac{\text{well-typed}(T) \quad \text{well-typed}{(S_1) \quad \text{well-typed}(S_2)}}{\text{well-typed}(if (T) else \{S_2\})}\)

A while statement is well-typed if and only if all of its components are
well-typed

\(\frac{\text{well-typed}(T) \quad \text{well-typed}(S)}{\text{well-typed}(while (T) \{S\})}\)

There is a final sanity check with the left- and right-hand sides of an
assignment statement.

Given an expression, say \(x = y\), notice the left-hand side and the
right-hand side represent different things.

The left-hand side represents a place to store data; it must be a
location of memory. The right-hand side must be a value; that is, any
well-typed expression.

Anything that denotes a storage location is an lvalue.

Consider the following two snippets of code

\begin{minted}[frame=lines, linenos, fontsize=\large]
{cpp}
int x = 0;
x = 5;
\end{minted}

This is okay; the lvalue \texttt{x} is a storage location.

\begin{minted}[frame=lines, linenos, fontsize=\large]
{cpp}
int x = 0;
5 = x;
\end{minted}

This is not okay; the lvalue 5 is an integer and not a storage location.

For us, lvalues are any of variable names, dereferenced pointers and any
parenthetical combinations of these. These are all forced on us by the
WLP4 grammar so the checking is done for you.

The empty sequence is well-typed \(\overline{\text{well-typed}()}\)

Consecutive statements are well-typed if and only if each statement is
well-typed

\(\frac{\text{well-typed}(S_1) \quad \text{well-typed}(S_2)}{\text{well-typed}(S_1 ; S_2)}\)

Procedures are well-typed if and only if the body is well-typed and the
procedure returns an \texttt{int}.

\(\frac{\text{well-typed}(S) \quad E: int}{\text{well-typed}(int f(dcl_1, \ldots, dcl_n) \{ dcls S return E;\})}\)

Type-checking recommendations
\begin{itemize}
    \item Brush up on recursion. Everything from
this point on is traversing a tree.
\end{itemize}

\ex{Type-check a tree. We will use the code from last time.}{}

\begin{minted}[frame=lines, linenos, fontsize=\large]
{cpp}
int foo(int a) {
	int x = 0;
	return x + a;
}
int wain(int *a, int b) {
	return foo(b) + a;
}
\end{minted}

\begin{minted}[frame=lines, linenos, fontsize=\large]
{cpp}
void check(ParseTreeNode *)
post-order
\end{minted}


Infer types for sub expressions using the rules.

Find three distinct one-character changes that make this code fail to
type-check (while still passing parsing):

\begin{minted}[frame=lines, linenos, fontsize=\large]
{cpp}
int wain(int a, int b) {
	return a + b;
}
\end{minted}

Change \texttt{int\ a} to \texttt{int*\ a}.

Change \texttt{return\ a\ +\ b;} to \texttt{return\ *a\ +\ b}.

Change \texttt{return\ a\ +\ b} to \texttt{return\ \&a\ +\ b}.

There are infinitely many equivalent MIPS programs for a single WLP4
program.

\qn{}{ Which should we output?
}

Correctness is most important. We seek a simplistic solution. Efficiency
to compile and efficiency of the code itself.

Real compilers have an intermediate representation (IR) that's close to
assembly code but (at least) has infinite registers. This IR is good for
optimization. We don't do optimization, we we won't use IR.

Our step of generating assembly will be (more-or-less) the ``generate
IR'' step of a larger compiler. MIPS is our IR.

\begin{minted}[frame=lines, linenos, fontsize=\large]
{cpp}
int wain(int a, int b) {
	return a;
}
\end{minted}

Recall our \texttt{mips.twoints} convention that registers \$1 and \$2
store the input values, and our usual convention to return in register
\$3.

\begin{tcolorbox}
\begin{verbatim}
add $3, $1, $0
jr $31
\end{verbatim}
\end{tcolorbox}

\qn{}{How did we know where \texttt{a} was stored? What if we
had \(>2\) arguments? What if we had done something complex with
intermediary results?}

The parse tree will be the same if we'd done \texttt{return\ b} instead
of \texttt{return\ a}.

The parse tree isn't going to be enough to determine the difference
between the two pieces of code.

\qn{}{ How can we resolve this? How can we distinguish between
these two codes?
}

We use the symbol table.


\begin{table}[h]
    \centering
    \begin{tabular}{|c|c|c|}\hline
         Symbol&  Type& Location\\\hline
         $a$&  int& $\$1$\\\hline
         $b$&  int& $\$2$\\\hline
    \end{tabular}
    
\end{table}


\begin{minted}[frame=lines, linenos, fontsize=\large]
{cpp}
int wain(int a, int b) {return a;}
\end{minted}

\begin{tcolorbox}
\begin{verbatim}
lis $4
.word 4
sw $1, -4($30)
sub $30, $30, $4
sw $2, -4($30)
sub $30, $30, $4
lw $3, 4($30)
add $30, $30, $4
add $30, $30, $4
jr $31
\end{verbatim}
\end{tcolorbox}

We make the convention that \texttt{\$4} always contains the number 4.

Instead of the symbol table storing registers, it should store the
offset from the stack pointer.

\begin{table}[h]
    \centering
    \begin{tabular}{|c|c|c|} \hline 
         Symbol&  Type& Location\\ \hline 
         $a$&  int& $4$\\ \hline 
         $b$&  int& $0$\\ \hline
    \end{tabular}
    
\end{table}

Offset from stack pointer will cause problems.

Variables also have to go on the stack but we don't know what the
offsets should be until we process all of the variables and parameters.

For example


\begin{minted}[frame=lines, linenos, fontsize=\large]
{cpp}
int wain(int a, int b)
{ int c = 0; return a; }
\end{minted}

Code generated

\begin{tcolorbox}
\begin{verbatim}
lis $4
.word 4
sw $1, -4($30)
sub $30, $30, $4
sw $2, -4($30)
sub $30, $30, $4
sw $0, -4($30) ; For int c = 0
sub $30, $30, $4
lw $3, 8($30) ; Offset changed due to presence of c!
...
jr $31
\end{verbatim}
\end{tcolorbox}

As we process the code, we need to be able to compute the offset as we
see the values. Also, we need to handle intermediate values of
complicated arithmetic expressions by storing on the stack.

\begin{table}[h]
    \centering
    \begin{tabular}{|c|c|c|} \hline 
         Symbol&  Type& Location\\ \hline 
         $a$&  int& 8\\ \hline 
         $b$&  int& 4\\ \hline 
 $c$& int&0\\ \hline
    \end{tabular}
    
\end{table}

How then do we arrange it so that when we see the variable, we know what the offset is? Remember that the key issue here is that \$30 (the top of the stack) changes.

Reference the offset from the bottom of the stack frame. We will store
this value in \$29. This is called the ``frame pointer''.

If we calculate offsets from \$29, then no matter how far we move the
top of the stack, the offsets from \$29 will be unchanged.

\begin{tcolorbox}
\begin{verbatim}
lis $4
.word 4
sub $29, $30, $4
sw $1, -4($30)
sub $30, $30, $4
sw $2, -4($30)
sub $30, $30, $4
sw $0, -4($30)
sub $30, $30, $4
lw $3, 0($29) ; Offset always 0 from $29
...
jr $31
\end{verbatim}
\end{tcolorbox}

\begin{table}[h]
    \centering
    \begin{tabular}{|c|c|c|} \hline 
         Symbol&  Type& Location\\ \hline 
         $a$&  int& 0\\ \hline 
         $b$&  int& -4\\ \hline 
        $c$& int&-8\\ \hline
    \end{tabular}
    
\end{table}

What about a more complicated program?


\begin{minted}[frame=lines, linenos, fontsize=\large]
{cpp}
int wain(int a, int b) {
	return a - b;
}
\end{minted}

\qn{}{ How do we handle this?
}

When \texttt{a} and \texttt{b} were in registers, we could just subtract
them. Now we need to load them, then subtract them.

Load them into registers? We'll run out of registers again with more
complicated behaviour.

We'll continue to use \$3 for the result of any expression. Also use \$5
for intermediate scratch values.

\begin{tcolorbox}  

\begin{verbatim}
lis $4
.word 4
sub $29, $30, $4
sw $1, -4($30)
sub $30, $30, $4
sw $2, -4($30)
sub $30, $30, $4
lw $3, 0($29) ; a
add $5, $3, $0 ; Move a to $5
lw $3, -4($29) ; b
sub $3, $5, $3 ; a -b
... ; restore stack
jr $31
\end{verbatim}
\end{tcolorbox}

\qn{}{Where does this approach break down?
}

Consider something like \(a + (b - c)\). Would need to load \(a\), load
\(b\), load \(c\), compute \(b-c\), then compute the answer. This would
require a third register.

\qn{}{Where should we store these values instead?
}

On the stack again.

Abstraction: We'll use some shorthand for our code

code(\(x\)) represents the generated code for \(x\).

code(\(a\)): (where \(a\) is a variable) \texttt{lw\ \$3,\ N(\$29)}

push(\$x):
\begin{tcolorbox}  

\begin{verbatim}
sw $x, -4($30) 
sub $30, $30, $4
\end{verbatim}
\end{tcolorbox}

pop(\$x):
\begin{tcolorbox}  

\begin{verbatim}
add $30, $30, $4
lw $x, -4($30)
\end{verbatim}
\end{tcolorbox}

\begin{tcolorbox}  

\begin{verbatim}
code (a-b):
    code(a) + 
    push($3) +
    code(b) + 
    pop($5) + 
    sub $3, $5, $3
\end{verbatim}
\end{tcolorbox}

\section{Lecture 18}\label{lecture-18}

Let's compute the MIPS code for

\begin{minted}[frame=lines, linenos, fontsize=\large]
{cpp}
int wain(int a, int b) {
	int c = 3;
	return a + (b - c);
}
\end{minted}

Solution

\begin{tcolorbox}
\begin{verbatim}
lw $3, 0($29) ; a 
sw $3, -4($30) ; push($3) 
sub $30, $30, $4 
lw $3, -4($29) ; b 
sw $3, -4($30) ; push($3) 
sub $30, $30, $4 
lw $3, -8($29) ; c 
add $30, $30, $4 ; pop($5) 
lw $5, -4($30) 
sub $3, $5, $3 ; b – c 
add $30, $30, $4 ; pop($5) l
w $5, -4($30) 
add $3, $5, $3
\end{verbatim}
\end{tcolorbox}

We can generalize this technique so we only need two registers for any
computation. (Divide and conquer).

Singleton grammar productions are relatively straightforward to
translate:

\texttt{code(S -\textgreater{} BOF procedures EOF): code(procedures)}

\texttt{code(expr\ -\textgreater{}\ term):\ code(term)}

lvalues are odd. Recursion might do the wrong thing.

The basic idea of our code function is that it produces the code to put
the value of the expression in \texttt{\$3}.

lvalues shouldn't actually generate values.

If we have \texttt{\{\ int\ x\ =\ 0;\ x\ =\ 3;\ \}}, having the code for
\texttt{x} in \texttt{x=3} generate 0 would be useless.

We could have the code function do something different for lvalues

\texttt{code(lvalue\ -\textgreater{}\ anything)} produces an address in
\texttt{\$3} instead of its value.

Or, we could have the code function for expressions that include lvalues
do something different based on the kind of lvalue.

\texttt{code(stmt\ -\textgreater{}\ (lvalue\ -\textgreater{}\ ID)\ BECOMES\ expr\ SEMI)}
has to be distinct from
\texttt{code(stmt-\textgreater{}(lvalue\ -\textgreater{}\ STAR\ expr)\ BECOMES\ expr\ SEMI)}.

This code is less modular and less maintainable (not recommended).

We have two ways of outputting and one way of inputting:
\texttt{println}, \texttt{putchar}, \texttt{getchar}.

Character I/O corresponds directly to memory-mapped I/O addresses:

\begin{tcolorbox}
\begin{verbatim}
code(putchar(expr);) = 
    code(expr) + 
    lis $5 + 
    .word 0xffff000c +
    sw $3, 0($5)    
\end{verbatim}
\end{tcolorbox}

\begin{tcolorbox}
\begin{verbatim}
code(getchar()) = 
    lis $5 + 
    .word 0xffff0004 +
    lw $3, 0($5)
\end{verbatim}
\end{tcolorbox}

\texttt{println} is more complex than \texttt{getchar} or
\texttt{putchar}.

What do we generate for
\texttt{stmt\ -\textgreater{}\ PRINTLN\ LPAREN\ expr\ RPAREN\ SEMI}.

We had to write MIPS code to do this, but that would be a lot of code to
generate each time.

A compiler mixes the code it outputs with code from a runtime
environment

\defn{}{A runtime environment is the execution
environment provided to an application or software by the operating
system to assist programs in their execution. Such an environment may
include things such as procedures, libraries, environment variables etc.
}

MERL files. MIPS Executable Relocatable Linkable. Format for object
files. MERL helps us store additional information needed by the loader
and linker from output.

We will now need to do

\begin{minted}[frame=lines, linenos, fontsize=\large]
{bash}
./wlpgen < source.wlp4ti > source.asm
cs241.linkasm < source.asm > source.merl
cs241.merl source.merl print.merl > exec.mips
\end{minted}


gcc and clang are not compilers, they are compiler drivers. They call
other programs to compile, assemble, and link code.

To use print, we need to add \texttt{.import\ print} to the BOF.

After this we can use \texttt{print} in our MIPS code. It will print the
contents of \texttt{\$1}.

\begin{tcolorbox}
\begin{verbatim}
code(println(expr);) = push($1)
                    + code(expr)
                    + add $1, $3, $0
                    + push ($31)
                    + lis $5 + .word print
                    + jalr $5 + pop($31)
                    + pop($1)
\end{verbatim}
\end{tcolorbox}

We will write Baby's First Operating System

\begin{tcolorbox}
\begin{verbatim}
repeat:
    p <- next program to run
    read the program into memory at address 0x0
    jalr $0
    beq $0, $0, repeat
\end{verbatim}
\end{tcolorbox}

Where should this be stored? Could choose different addresses at
assembly time, but how do we make sure they don't conflict.

A more flexible option is to make sure that code can be loaded anywhere.

Loader's job:
\begin{itemize}
    \item Take a program \(P\) as input 
    \item Find a location \(\alpha\) in memory for \(P\)
    \item Copy \(P\) to memory, starting at \(\alpha\)
    \item Return \(\alpha\) to OS
\end{itemize}

Baby's First OS v2

\begin{tcolorbox}
\begin{verbatim}
repeat:
    p <-- choose program to run
    $3 <--- loader(p)
    jalr $3
    beq $0, $0, repeat

loader
a <-- findFreeRAM(N)
for (i = 0; i < codeLength; ++i) {
    mem[a+4i] = file[i];
}
$30 <-- a + N
return a to OS
\end{verbatim}
\end{tcolorbox}

\qn{}{ How did we assemble \texttt{.word\ label}?
}

It compiles to an address; the address of that label assuming that the
program was loaded at 0. Loader needs to fix this.

More problems

\texttt{.word\ id}: need to add alpha to id

\texttt{.word\ constant}: do not relocate

\texttt{beq} \texttt{bne} (whether they use labels or not): do not
relocate

We translate assembly code into machine code (bits). Given:
\texttt{0x00000018}, is this \texttt{.word\ 0x18} or \texttt{.word\ id}?
We can't know.

Thus, we need a way for our loader to know what does and what doesn't
need to be relocated. We need to remember which .words were labels.

In our MERL files we need the code, but also the location of any
\texttt{.word} ids.

\begin{tcolorbox}
\begin{verbatim}
lis $3
.word 0xabc
lis $1
.word A
jr $1
B: jr $31
A: 
beq $0, $0, B
.word B
\end{verbatim}
\end{tcolorbox}

MERL would be

\begin{tcolorbox}
\begin{verbatim}
beq $0, $0, 2
.word end Module
.word endCode
---
lis $3
.word 0xabc
lis $1
reloc1: .word A ; different
jr $1
B: 
jr $31
A: 
beq $0, $0, B
reloc2: .word B ; different
---
endCode: 
.word 1
.word reloc1
.word 1
.word reloc2
endModule:
\end{verbatim}
\end{tcolorbox}
Loading requires two passes:

Pass 1: Load the code from the file into the selected location in
memory.

Pass 2: Using the relocation table, update any memory addresses that
have relocation entries.

Even with this, it is possible to write code that only works at address
0:

\begin{tcolorbox}
\begin{verbatim}
lis $2
.word 12
jr $2
jr $31
\end{verbatim}
\end{tcolorbox}

We should never encode address as anything other than labels, so that
your loader can update the references.

\begin{minted}[frame=lines, linenos, fontsize=\large]
{cpp}
read_word // skip first word in MERL file
endMod ← read_word() // second word is address of end of MERL
codeSize ← read_word() - 12 // compute size of code
a ← findFreeRam(codeSize)

for (int i = 0; i < codeSize; i+=4) { // load program
	MEM[a + i] ← read_word()
}
i ← codeSize + 12 // start relocation of table
while (i < endMod) {
	format ← read_word()
	if (format == 1) {
		rel ← read_word() //relocate address
		MEM[a + rel - 12] += a - 12 // go forward by a but back by 12
	} else {
		ERROR // unknown format type
	}
	i += 8 // update to next entry
}
\end{minted}

\section{Lecture 19}\label{lecture-19}

Most of our statements have been completed (except for if and while).

We need to handle boolean tests for conditionals. Convention is to store
1 inside of \texttt{\$11}.

Code structure

\begin{tcolorbox}
\begin{verbatim}
; Prologue
lis $4
.word 4
lis $10
.word print
lis $11
.word 1
sub $29, $30, $4
; end Prologue

add $30, $29, $4
jr $31
\end{verbatim}
\end{tcolorbox}

\qn{}{What is the code for the rule test \(\to\) expr\(_A <\)
expr\(_B\)?
}

\begin{tcolorbox}
\begin{verbatim}
code(expr_A)
+ push($3)
+ code(expr_B)
+ pop($5)
+ slt $3, $5, $3
\end{verbatim}
\end{tcolorbox}

For test \(\to\) expr\(_{A} >\) expr\(_B\) we change
\texttt{slt\ \$3,\ \$5,\ \$3} to \texttt{slt\ \$3,\ \$3,\ \$5}.

Translate test \(\to\) expr\(_{A}\) != expr\(_{B}\).

\begin{tcolorbox}
\begin{verbatim}
code(expr_A)
+ push($3)
+ code(expr_B)
+ pop($5)
; maybe store $6 and $7 if used
+ slt $6, $3, $5
+ slt $7, $5, $3
; Note 0 <= $6 + $7 <= 1
+ add $3, $6, $7
\end{verbatim}
\end{tcolorbox}

Now for test \(\to\) expr\(_A ==\) expr\(_{B}\)?

The key idea is \(a == b\) is the same as \(!( a\) != \(b)\).

We have \(!\) by adding the line \texttt{sub\ \$3,\ \$11,\ \$3} to flip
\(0\) to \(1\) and vice versa.

For test \(\to\) expr\(_{A} \le\) expr\(_{B}\) by using the fact that
\(a \le b\) is the same as \(!(a > b)\)

Rule:
\texttt{statement\ →\ IF\ (test)\ \{stmts\ 1\}\ ELSE\ \{stmts\ 2\}}

\begin{tcolorbox}
\begin{verbatim}
code (statement) = code(test)
                + beq $3, $0, else
                + code(stmts 1)
                + beq $0, $0, endif
                + else: code(stmts 2)
                + end if:
\end{verbatim}
\end{tcolorbox}

If we have multiple if statements, the label names will conflict.

We need a way of inventing totally unique label names.

We can keep track of how many if statements we have. Keep a counter
\texttt{ifcounter}.

Each time we have an if statement, increment this counter.

Use label names like \texttt{else\#} and \texttt{endif\#} where
\texttt{\#} corresponds to the \texttt{ifcounter}.

Rule: \texttt{statement\ →\ WHILE\ (test)\ \{statements\}}

\begin{tcolorbox}
\begin{verbatim}
code(statement) = loop: code(test)
                + beq $3, $0, endWhile
                + code(stmts)
                + beq $0, $0, loop
                + endWhile:
\end{verbatim}
\end{tcolorbox}

Since we are generating MIPS code; we can also generate comments with
MIPS code. Debugging code generators is hard.

Recap

\begin{itemize}
    \item \texttt{\$0} is always \(0\)
    \item \texttt{\$1} and \texttt{\$2} are for arguments \(1\) and \(2\) in \texttt{wain}
    \item \texttt{\$3} is always
for output 
    \item \texttt{\$4} is always \(4\)
    \item  \texttt{\$5} is always for
intermediate computations
    \item \texttt{\$6} and \texttt{\$7} may be for
intermediate computations 
    \item \texttt{\$10} will store \texttt{print}
    \item \texttt{\$11} is reserved for \(1\)
    \item \texttt{\$29} is our frame pointer
(fp)
    \item \texttt{\$30} is our stack pointer (sp)
    \item \texttt{\$31} is our
return address (ra).
\end{itemize}

Prologue

At the beginning of the the code, we 

\begin{itemize}
    \item Load \(4\) into \texttt{\$4} and
\(1\) into \texttt{\$11}
    \item Import print. Store in \texttt{\$10}
    \item Store
the return address on the stack 
    \item Initialize the frame pointer hence
creating a stack frame 
    \item  Store registers \(1\) and \(2\)
\end{itemize}


Body

Need to store local variables in the stack frame. Contain MIPS code
corresponding to the WLP4 program.

Epilogue

Need to pop the stack frame. Also need to restore the previous
variables.

We have reached pointers. We need to support all the following

\begin{itemize}
    \item NULL
    \item Allocating and deallocating heap memory
    \item Dereferencing
    \item Address-of
    \item Pointer arithmetic
    \item Pointer comparisons
    \item Pointer assignments and pointer access
\end{itemize}


NULL cannot be the value \texttt{0x0} it is a valid memory address. We
want our NULL to crash in attempt to dereference. We pick a NULL that is
not word-aligned (not a multiple of 4). So we can pick \texttt{0x1}.

\begin{tcolorbox}
\begin{verbatim}
code(factor → NULL) =
add $3, $0, $11
\end{verbatim}
\end{tcolorbox}

\qn{}{ What about dereferencing?
}

factor\(_1 \to\) STAR factor\(_2\)

The value in factor\(_2\) is a pointer (otherwise a type error). We want
to access the value at factor\(_2\) and load it somewhere.

We load into \texttt{\$3}. Since factor\(_2\) is a memory address, we
want to load the value in the memory address at \texttt{\$3} and store
in \texttt{\$3}.

\begin{tcolorbox}
\begin{verbatim}
code(factor_1 → STAR factor_2) =
code(factor_2) + lw $3, $0($3)
\end{verbatim}
\end{tcolorbox}

Need to be careful of the difference between lvalues and pointers.

Recall that an lvalue is something that can appear as the LHS of an
assignment rule. We can have a rule \texttt{factor\ →\ AMP\ lvalue}.
Suppose we have an ID value \texttt{a}. How do we find out where
\texttt{a} is in memory?

We use our symbol table. We can load the offset first and then use this
to find out the location.

Comparisons of pointers work the same as with integers with one
exception. Pointers cannot be negative, so \texttt{slt} is not what we
want to use. We should use \texttt{sltu} instead.

Given \texttt{test\ →\ expr\ COMP\ expr} how can we tell which of
\texttt{slt} or \texttt{sltu} to use? We check the type of exprs.

Recall for addition and subtraction we have several contracts. The code
for addition will vary based on the type of its subexpressions. For
\texttt{int\ +\ int} or \texttt{int\ -\ int}, we proceed as before. This
leaves \(4\) contracts we need to consider that use pointers.

Addition \texttt{expr\_1\ →\ expr\_2\ +\ term} where
\texttt{type(expr\_2)\ ==\ int*} and \texttt{type(term)\ ==\ int}

\begin{tcolorbox}
\begin{verbatim}
code(expr_1) = code(expr_2)
            + push($3)
            + code(term)
            + mult $3, $4
            + mflo $3
            + pop($5)
            + add $3, $5, $3
\end{verbatim}
\end{tcolorbox}

\texttt{expr\_1\ →\ expr\_2\ +\ term} where
\texttt{type(expr\_2)\ ==\ int} and \texttt{type(term)\ ==\ int*}

\begin{tcolorbox}
\begin{verbatim}
code(expr_1) = code(expr_2)
            + mult $3, $4
            + mflo $3
            + push($3)
            + code(term)
            + pop($5)
            + add $3, $5, $3
\end{verbatim}
\end{tcolorbox}

Subtraction

\texttt{expr\_1\ →\ expr\_2\ -\ term} where
\texttt{type(expr\_2)\ ==\ int*} and \texttt{type(term)\ ==\ int}

\begin{tcolorbox}
\begin{verbatim}
code(expr_1) = code(expr_2)
            + push($3)
            + code(term)
            + mult $3, $4
            + mflo $3
            + pop($5)
            + sub $3, $5, $3
\end{verbatim}
\end{tcolorbox}

\texttt{expr\_1\ →\ expr\_2\ -\ term} where
\texttt{type(expr\_2)\ ==\ int*} and \texttt{type(term)\ ==\ int*}

\begin{tcolorbox}
\begin{verbatim}
code(expr_1) = code(expr_2)
            + push($3)
            + code(term)
            + pop($5)
            + sub $3, $5, $3
            + div $3, $4
            + mflo $3
\end{verbatim}
\end{tcolorbox}

\section{Lecture 20}\label{lecture-20}

We need to handle calls such as \texttt{new} and \texttt{delete}

We can outsource this work to the runtime.

Prologue Additions

\begin{tcolorbox}
\begin{verbatim}
.import init
.import new
.import delete
\end{verbatim}
\end{tcolorbox}

The command \texttt{init} initializes the heap. Must be called at the
beginning. Takes a parameter in \texttt{\$2} and initializes data
structure.

New
\begin{itemize}
    \item Finds the number of new words needed as specified in \texttt{\$1}
    \item  Returns a pointer to memory of beginning of this many words in\texttt{\$3} if successful (otherwise places \(0\) in \texttt{\$3})
\end{itemize}

\begin{tcolorbox}
\begin{verbatim}
code(new int[expr]) = code(expr)
                    + add $1, $3, $0
                    + call(new)
                    + bne $3, $0, 1
                    + add $3, $11, $0
\end{verbatim}
\end{tcolorbox}

Delete
\begin{itemize}
\item Requires that \texttt{\$1} is a memory address to be
deallocated    
\end{itemize}

\begin{tcolorbox}
\begin{verbatim}
code(delete [] expr) = code(expr)
                    + beq $3, $11, skipDelete
                    + add $1, $3, $0
                    + call(delete)
                    + skipDelete:
\end{verbatim}
\end{tcolorbox}

We need to now deal with multiple function calls.

\qn{}{ Who should save which registers? The caller? The callee?
What do functions need to update/initialize? How do we update our symbol
table?
}

\qn{}{ What do we need to do for wain?
}

Import \texttt{print}, \texttt{init}, \texttt{new}, \texttt{delete}

Initialize \texttt{\$4}, \texttt{\$10}, \texttt{\$11}

Call \texttt{init}

Save \texttt{\$1}, \texttt{\$2}

Reset stack

Call \texttt{jr\ \$31}

For general procedures we don't need any imports. But we need to update
\texttt{\$29}, save registers, restore registers and stack and
\texttt{jr\ \$31}

\qn{}{ Who is responsible for saving and restoring registers?
}

\defn{}{ The caller is a function \(f\) that calls another
function \(g\)
}

\defn{}{ The callee is a function \(g\) that is being
called by another function \(f\)
}

Our current convention:

\begin{itemize}
    \item Caller needs to save \texttt{\$31}. Otherwise
we lose the return address (to, e.g., the loader) once we compute our
call to \texttt{jalr} 
    \item Callee has been saving registers it will modify
and restore at the end. The function \(f\) shouldn't be worried about
which registers \(g\) might be using. This makes sense as well from a
programming point of view.
    \item Note that we have only used registers from
\texttt{\$1} to \texttt{\$7} (and registers \texttt{\$4}, \texttt{\$10},
\texttt{\$11} are constant) as well as registers \texttt{\$29},
\texttt{\$30}, and \texttt{\$31}.
    \item \texttt{\$30} is preserved through symmetry
\end{itemize}

\qn{}{Who should save \texttt{\$29}?
}

Assume that we will require that the callee will save \texttt{\$29}.
Thus they will initialize \texttt{\$29} first:

\texttt{g:\ sub\ \$29,\ \$30,\ \$4}

and then \(g\) saves registers. Is this the right order to do things in?

If we save registers first, \texttt{\$29} is supposed to point to the
beginning of the stack frame, but \texttt{\$30} has already changed to
store all registers. This is fine for now, but \texttt{\$29} might be
pointing to somewhere in the middle of the stack frame.

Having \texttt{\$29} in the middle is annoying since we will need it
later. We choose to make \texttt{\$29} the bottom.

Therefore, we do:

\begin{tcolorbox}
\begin{verbatim}
push($29)
add $29, $30, $0
;push other registers
\end{verbatim}
\end{tcolorbox}

This callee-save approach with \texttt{\$29} will work.

Caller-save approach: We could have the caller save \texttt{\$29}

\begin{tcolorbox}
\begin{verbatim}
push($29)
push($31)
lis $5
.word g
jalr $5
pop($31)
pop($29)
\end{verbatim}
\end{tcolorbox}

This is much easier (we are going to do this).

Must be careful where everything is relative to \texttt{\$29}.

Procedures:

We need to store the arguments to pass to a function.

For \texttt{factor\ →\ ID(expr1,\ ...,\ exprn)}, we have

\begin{tcolorbox}
\begin{verbatim}
code(factor) = push($29) + push($31)
            + code(expr1) + push($3)
            + code(expr2) + push($3)
            + ...
            + code(exprn) + push($3)
            + lis $5
            + .word ID
            + jalr $5
            + pop n times (pop all regs)
            + pop($31) + pop($29)
\end{verbatim}
\end{tcolorbox}

For
\texttt{procedure\ →\ int\ ID(params)\ \{dcls\ stmts\ RETURN\ expr;\}}
we have

\begin{tcolorbox}
\begin{verbatim}
code(procedure) = ID: sub $29, $30, $4
                + ; Save regs here?
                + code(dcls); local vars
                + ; OR save regs here?
                + code(stmts)
                + code(expr)
                + pop regs ; restore saved
                + add $30, $29, $4
                + jr $31
\end{verbatim}
\end{tcolorbox}

\qn{}{ When do we save registers? Before \texttt{code(dcls)} or
after?
}

Saving them before is strange.

Stack


\begin{table}[h]
    \centering
    \begin{tabular}{|c|c|c|} \hline 
         $\$30$&  & \\ \hline 
         &  local vars of $g$& frame of $g$\\ \hline 
         &  saved regs of $g$& frame of $g$\\ \hline 
         $\$29$&  args of $g$& frame of $f$\\ \hline 
         &  $\$31$& frame of $f$\\ \hline 
         &  $\$29$& frame of $f$\\ \hline
    \end{tabular}
    
\end{table}

Symbol Table Revisited


\texttt{int g(int a, int b) \{ int c = 0; int d;\}}

Symbol table for \(g\) looks like

\begin{table}[h]
    \centering
    \begin{tabular}{|c|c|c|} \hline 
         Symbol&Table  & Offset (from $\$29$) \\ \hline 
         a&int&8\\ \hline 
         b&  int& 4\\ \hline 
         c&  int& ???\\ \hline 
         d&  int& ???\\ \hline 
    \end{tabular}
    
\end{table}

Let's try pushing the registers after pushing the declarations.

For
\texttt{procedure\ -\textgreater{}\ int\ ID(params)\ \{dcls\ stmts\ RETURN\ expr;\}}

\begin{tcolorbox}
\begin{verbatim}
code(procedure) = ID: sub $29, $30, $4
                + code(dcls)
                + push regs 
                + code(stmts)
                + code(expr)
                + pop regs
                + add $30, $29, $4
                + jr $31
\end{verbatim}
\end{tcolorbox}

New stack

\begin{table}[h]
    \centering
    \begin{tabular}{|c|c|c|} \hline 
         $\$30$&  & \\ \hline 
         &  saved regs of $g$& frame of $g$\\ \hline 
         &  local vars of $g$& frame of $g$\\ \hline 
         $\$29$&  args of $g$& frame of $f$\\ \hline 
         &  $\$31$& frame of $f$\\ \hline 
         &  $\$29$& frame of $f$\\ \hline
    \end{tabular}
    
\end{table}

Symbol Table Revisited Revisited

\texttt{int\ g(int\ a,\ int\ b)\ \{int\ c\ =\ 0;\ ind\ d;\}}

\begin{table}[h]
    \centering
    \begin{tabular}{|c|c|c|} \hline 
         Symbol&Table  & Offset (from $\$29$) \\ \hline 
         a&int&8\\ \hline 
         b&  int& 4\\ \hline 
         c&  int& 0\\ \hline 
         d&  int& -4\\ \hline 
    \end{tabular}
    
\end{table}

Parameters should have positive offsets. Local variables should have
non-positive offsets.

Symbol table should have added 4 \(\cdot\) \(num\)\((\)params\()\) to
each entry in the table.

This complicates pushing registers, because we're now generating some
code before we preserve register values.

Does this matter for us? No, because our declarations are forced to be
simple. If the language allowed complex expressions then it would
matter.

Labels can introduce another annoying problem

Consider the code

\begin{minted}[frame=lines, linenos, fontsize=\large]
{cpp}
int print(int a ) {
    return a;
}
\end{minted}

\qn{}{ What is the problem here?
}

We already have a label called \texttt{print}. Since it is not a WLP4
procedure, it shouldn't interfere with WLP4.

We can ban WLP4 code that uses function names that match some of our
reserved labels like \texttt{new}, \texttt{init}, etc. This is not very
future proof.

Since MIPS labels don't have to be identical to WLP4 procedure names, we
can change them.

We will prepend an `F' to the front of labels corresponding to
procedures. Then, so long as we don't create any labels with a `F' at
the beginning for any other purpose it should be okay.

The print function above would correspond to a label \texttt{Fprint}.

Revisiting Translation

\texttt{factor\ →\ ID(expr\ 1,\ ...,\ exprn)} we have,

\begin{tcolorbox}
\begin{verbatim}
code(factor) = push($29) + push($31)
            + code(expr1) + push($3)
            + code(expr2) + push($3)
            + ...
            + code(exprn) + push($3)
            + lis $5
            + .word FID
            + jalr $5
            + pop n times (pop all regs)
            + pop($31) + pop($29)
\end{verbatim}
\end{tcolorbox}

Compete example, with procedures:

\begin{minted}[frame=lines, linenos, fontsize=\large]
{cpp}
int add(int a, int b) {
	int c = 0;
	c = a + b;
	return c;
}
int wain(int a, int b) {
	return add(a, b);
}
\end{minted}

\begin{tcolorbox}
\begin{verbatim}
.import print/new, delete, init

lis $4
.word 4
lis $10
.word print
lis $11
.word 1
---
beq $0, $0, Fwain
Fadd: sub $29, $30, $4
    code(int c = 0);
        push($0) ; above translated
    ; save regs
    push($5)
    push($6)
    push($7)
    push($31)
    ; done with register saving (now stmts)
    code(stmts)
        code(c = a + b)
            code(lvalue/c)
            push($3)
            code(expr/a+b)
            pop($5)
            sw $3, 0($5)
    ; final expr
    code(c)
    ; restore regs
    ; pop/locals 
    jr $31 ; need to restore the stack
Fwain:
    push($1) ; first param
    push($2) ; second param
    sub $29, $30, $4
    push($31)
    ; save register $2
    $2 = 0
    call init ($2)
    ; no stmts
    ; return expr
    code(add(a, b))
    pop($0) ; pop a
    pop($0) ; pop b
    jr $31
\end{verbatim}
\end{tcolorbox}

We have finished code generation.

\emph{Compiler Optimizations}

The goal of optimizations is generally to make code run faster

We want to reduce code size and make code run faster. (For bonus on A8,
we are only concerned with reducing code size).

Implementing these optimizations is not easy.

Constant Folding

If we want to generate the code for \(1+2\):

\begin{tcolorbox}
\begin{verbatim}
lis $3 ; $3 = 1
.word 1
sw $3, -4($30)
sub $30, $30, $4
lis $2 ; $3 = 2
.word 2
add $30, $30, $4 ; pop($5)
lw $5, -4($30)
add $3, $5, $3 ; $3 = 1 + 2
\end{verbatim}
\end{tcolorbox}

Our code generator could produce the following code for \(1 + 2\)

\begin{tcolorbox}
\begin{verbatim}
lis $3
.word 3
\end{verbatim}
\end{tcolorbox}

If we notice that each element of the expression is a constant, we can
add the constants at compile time and output the code for the final
value (instead of doing it at runtime).

If the expression was \(1 + x\), we would need to know the value of
variable \(x\). We cannot determine this at compile time.

Constant Propogation

Sometimes the value of a variable is known at compile time:

\begin{minted}[frame=lines, linenos, fontsize=\large]
{cpp}
int x = 1;
return x + x;
\end{minted}

We can replace \(x\) with its known value, so this is equivalent to:

\begin{minted}[frame=lines, linenos, fontsize=\large]
{cpp}
int x = 1;
return 1 + 1;
\end{minted}

We can then apply Constant Folding

\begin{minted}[frame=lines, linenos, fontsize=\large]
{cpp}
int x = 1;
return 2;
\end{minted}

Since \texttt{x} is not used anywhere, we can eliminate the variable
declaration entirely:

\begin{minted}[frame=lines, linenos, fontsize=\large]
{cpp}
return 2;
\end{minted}

\section{Lecture 21}\label{lecture-21}

Constant propagation is more difficult than constant folding.

\begin{minted}[frame=lines, linenos, fontsize=\large]
{cpp}
int wain(int x, int y) {
	println(x + x); // Constant propagation can't be applied
	x = 1;
	println(x + x); // Can be applied
	x = y;
	return x + x; // Can't be applied
}
\end{minted}

We can only apply it if we know the variable's value does not depend on
the input during the part of the program we're processing.

Common Subexpression Elimination

Even if the value of \(x\) is not known, there is a simplification we
can make when generating code for \(x + x\).

Here is the naïve code (assuming \(x\) is at offset \(0\) from
\texttt{\$29})

\begin{tcolorbox}
\begin{verbatim}
lw $3, 0($29) ; $3 = x
sw $3, -4($30) ; push($3)
sub $30, $30, $4
lw $3, 0($29) ; $3 = x
add $30, $30, $4 ; pop($5)
lw $5, -4($30)
add $3, $5, $5 ; $3 = x + x
\end{verbatim}
\end{tcolorbox}

Even if the value of \(x\) is not known, there is a simplification we
can make when generating code for \(x+x\).

Since we're adding the same variable twice, we can just do this.

\begin{tcolorbox}
\begin{verbatim}
lw $3, 0($29) ; $3 = x
add $3, $3, $3 ; $3 = x + x
\end{verbatim}
\end{tcolorbox}

We can do the same trick with larger expressions, e.g., if we have
\((a * b - c) + (a * b - c)\):

\begin{tcolorbox}
\begin{verbatim}
; block of code that computes a * b - c
add $3, $3, $3
\end{verbatim}
\end{tcolorbox}

\qn{}{Can we apply common subexpression elimination to this
code?
}

\begin{minted}[frame=lines, linenos, fontsize=\large]
{cpp}
int f(int x) {
	println(x);
	return 2*x;
}
int wain(int a, int b) {
	return f(a) + f(a);
}
\end{minted}

No, CSE must not eliminate side effects.

Dead Code Elimination

Sometimes the compiler can determine that certain code will never
execute, and can eliminate this code
\begin{minted}[frame=lines, linenos, fontsize=\large]
{cpp}
int wain(int a, int b) {
	if (a < b) {
		if (b < a) {
			b = 0;	
		} else { }
	} else { b = 0; }
	return a + b;
}
\end{minted}

The code inside the innermost if can be ignored.

\begin{minted}[frame=lines, linenos, fontsize=\large]
{cpp}
int wain(int a, int b) {
	if (a < b) {
		if (b < a) {
			// dead code	
		} else { }
	} else { b = 0; }
	return a + b;
}
\end{minted}

Deleting this code has a size benefit, but no real performance benefit.
\begin{minted}[frame=lines, linenos, fontsize=\large]
{cpp}
int wain(int a, int b) {
	if (a < b) {
	// if condition eliminated
	} else { b = 0; }
	return a + b;
}
\end{minted}

Dead code elimination interacts with other optimizations.

\begin{minted}[frame=lines, linenos, fontsize=\large]
{cpp}
int wain(int x, int y) {
	int releaseVersion = 0;
	if (releaseVersion == 1) {
		x = 1;
	} else {
		x = 0;	
	}
	return x * y;
}
\end{minted}

Normally we can't apply constant propagation to \(x\) in the return.

\begin{minted}[frame=lines, linenos, fontsize=\large]
{cpp}
int wain(int x, int y) {
	int releaseVersion = 0;
	x = 0;
	return x * y;
}
\end{minted}

Now constant propagation can be used on \(x\) as well.

DCE can allow constant propagation to occur. Conversely, constant
propagation can allow the compiler to prove code is dead.

Register Allocation

We ran into the issue that for sufficiently complicated code, it is not
possible to store all values in registers.

Our solution was to put everything on the stack because this makes
generating code simpler and more consistent.

But using registers for storage is much faster than using RAM.

Real-world compilers try to use registers as much as possible.

A variable is live if the current value of the variable will be used at
a later point in the program.

A variable should be in a register if and only if it is live.

If too many variables or values are live at the same time, we have to
choose which ones to put in RAM vs registers.

\begin{tcolorbox}
\begin{verbatim}
x = 3;
y=10;
println(x);
z = 7;
y = y-x;
y = y-z;
println(z);
return z;
\end{verbatim}
\end{tcolorbox}

\begin{itemize}
\item
  \(x\) becomes live on line 1, and is last used on line 5
\item
  \(y\) becomes live on line 2, and is last used on line 6
\item
  \(z\) becomes live on line 4, and is last used on line 8
\end{itemize}

On lines 4 to 5, all three variables are live. If we only had two
registers available, we would need to put one variable in RAM.

We can use live ranges to construct a graph indicating which ranges
overlap, and use graph colouring algorithms to allocate registers (see
\href{https://www.jaidenratti.com/MATH239.pdf}{MATH239})

If the live range graph can be \(k-\)coloured, where \(k\) is the number
of available registers, we can allocate all variables to registers.

Graph colouring can be slow (NP-complete problem), so it is usually
approximated.

If the address-of operator is used on a variable then this variable must
go in RAM.

Significant gains are possible just by implementing a basic register
allocator. Optimizations to eliminate pushes/pops or decrease the number
of instructions for a push/pop are effective on A8.

A heuristic can be allocating variables and temporaries to registers on
a ``first-come, first-served'' basis.

In this case we'd need to modify the offset table so that there are two
kinds of variable locations: offsets from the frame pointer, or
registers.

Allocate non-parameter local variables in registers whenever possible.

Strength Reduction

This optimization involves replacing costly operations with equivalent
faster operations. For example, multiplication is slower than addition.

\begin{itemize}
    \item \((x + y) \times 2\) can be replaced with \((x + y) + (x + y)\), which
can then be optimized further using common subexpression elimination.
\end{itemize}

A more complex version involves optimizing loops which perform expensive
operations involving the loop counter.

Peephole Optimization

This optimization happens after code generation is finished. This is
used in LLVM.

Instead of directly outputting the generated code, the code is placed in
a data structure and subject to further analysis. The analysis tries to
find sequences of instructions that can be replaced with simpler
sequences.

For example

\begin{tcolorbox}
\begin{verbatim}
add $3, $1, $0 ; $3 = a
add $7, $3, $0 ; copy $3 to temporary register
\end{verbatim}
\end{tcolorbox}

A peephole optimization could change this to
\texttt{add\ \$7,\ \$1,\ \$0}. This might be easier than making the code
generation itself smarter.

We use a sliding window.

Inlining Functions

We replace a function call with the body of the function itself

\begin{minted}[frame=lines, linenos, fontsize=\large]
{cpp}
int foo(int x) {
	return x + x;
}
int wain(int a, int b) {
	return foo(a);
}
\end{minted}
This is equivalent to:

\begin{minted}[frame=lines, linenos, fontsize=\large]
{cpp}
int wain(int a, int b) {
	return a + a;
}
\end{minted}

This removes the overhead of a function call.

Tail Recursion

A recursive function call is in tail position if it is the last thing
the function executes before returning.

In this case, what happens normally is:
\begin{itemize}
    \item Recursive call happens, pushes
local variables to the stack
    \item Recursive call finishes, pops from the
stack, returns
    \item Original call finishes, pops from the stack, returns
\end{itemize}

Tail call optimization is based on the observation that in this
situation, the recursive call can reuse the stack frame of the original
call instead of pushing its own stack frame. This saves a lot of space.

Original call pops the reused stack frame.

\begin{minted}[frame=lines, linenos, fontsize=\large]
{cpp}
int fac(int n) {
	return fac_rec(1, n);
}
int fac_rec(int acc, int n) {
	if (n < 2) {
		return acc;
	}
	return fac_rec(n*acc,n-1);
}
\end{minted}

Can be replaced by

\begin{minted}[frame=lines, linenos, fontsize=\large]
{cpp}
int fac_rec(int acc, int n) {
	TOP:
	if (n < 2) return acc;
	acc = n * acc;
	n = n - 1;
	gotoTOP
}
\end{minted}

\section{Lecture 22}\label{lecture-22}

Why do we split programs into multiple files?

Modularity, team development, faster build time


 \qn{}{How do we resolve situations where we have labels in
different files?
}

One option is to cat all such files together, and then compile? 

\begin{itemize}
    \item Duplicate labels defined in different files
    \item Accidental use of labels
which should be private
\end{itemize}

Can we assemble files first and then cat?

Almost. Only one piece of code can be at \texttt{0x0} at a time. These
assembled files need to be MERL files, not just MIPS.

Concatenating two MERL files does not give a valid MERL file.

We still haven't resolve the issue of labels in different files.

When we encounter a \texttt{.word} where the label is not in the file,
we need to use a placeholder (an arbitrary value), and indicate that we
cannot run this program until the value of the label is given.

\begin{tcolorbox}
\begin{verbatim}
a.asm
lis $3
.word label
jr $31
\end{verbatim}
\end{tcolorbox}

\begin{tcolorbox}
\begin{verbatim}
b.asm
label: sw $4, -4($30)
\end{verbatim}
\end{tcolorbox}

You cannot run \texttt{a.asm} without linking with \texttt{b.asm}.

We need to extend our MERL file to notify us when we need to assembly
with multiple files.

But sometimes we make typos. Consider

\begin{tcolorbox}
\begin{verbatim}
lis $3
.word bananana
banana:
\end{verbatim}
\end{tcolorbox}

\qn{}{ Did we make a mistake? Did we mean
\texttt{.word\ banana}, or dod we mean for bananana to be provided by
another MERL file?
}

How do we recognize such errors? Without any other changes, our
assembler will believe that a label banana exists somewhere and would
load this.

\texttt{.import\ id} is the directive that tells the assembler which
symbols to link in. This will not assemble to a word of MIPS. Errors
occur if the label id is not in the current file and there is no
\texttt{.import\ id} in the file.

We need to add entries in the MERL symbol table. Previously we used the
code \texttt{0x1} for relocation entries, but this isn't a relocation
entry.

New format code: \texttt{0x11} for External Symbol Reference (ESR).

\qn{}{ What needs to be in an ESR entry?
}

\begin{enumerate}
\def\labelenumi{\arabic{enumi}.}
\item
  Where the symbol is being used
\item
  The name of said symbol
\end{enumerate}

Format:

\begin{tcolorbox}
\begin{verbatim}
0x11 ; Format code
; location used
; length of name of symbol (n)
; 1st ASCII char of name of symbol
; 2nd ASCII char of name of symbol
; ...
; nth ASCII char of name of symbol
\end{verbatim}
\end{tcolorbox}

\qn{}{ What if labels are duplicated?
}

Suppose we have \texttt{c.asm} along with our two other files that has:

\begin{tcolorbox}
\begin{verbatim}
label: add $1, $0, $0
; more code
beq $1, $0, label
\end{verbatim}
\end{tcolorbox}

We want label to not be exported, it should be self-contained.

\texttt{.export\ label} will make \texttt{label} available for linking
with other files. As with \texttt{.import}, it does not translate to a
word in MIPS. It tells the assembler to make an entry in the MERL symbol
table.

The assembler makes an ESD, or an External Symbol Definition, for these
types of words. It follows this format:

\begin{tcolorbox}
\begin{verbatim}
0x05 ; Format code
; Address the symbol represents
; length of name of symbol (n) 
; 1st ASCII char of name of symbol
; 2nd ASCII char of name of symbol
; ...
; nth ASCII char of name of symbol
\end{verbatim}
\end{tcolorbox}

Our linker now has everything it needs to do its job.

Linking Algorithm

\begin{minted}[frame=lines, linenos, fontsize=\large]
{cpp}
// Check for duplicate export errors
for each ESD in m1.table {
	if there is an ESD with the same name in m2.table {
		ERROR (duplicate exports)
	}
}

// Combine the code segments for the linked file
// The code for m2 must appear after the code for m1

linked_code = concatenate m1.code and m2.code
\end{minted}

\begin{minted}[frame=lines, linenos, fontsize=\large]
{cpp}
// Relocate m2's table entries
reloc_offset = end of m1.code - 12 // length of m1.code
for each entry in m2.table {
	add reloc_offset to the number stored in the entry
}

// Relocate m2.code
// It is essential for this to happen after the last step
for each relocation entry in m2.table {
	index = (address to relocate - 12) / word size
	add relocation offset to linked_code[index]
}
\end{minted}

\begin{minted}[frame=lines, linenos, fontsize=\large]
{cpp}
// Resolve imports for m1
for each ESR in m1.table {
	if there is an ESD in m2.table with a matching name {
		index = (address of ESR - 12) / word size
		overwrite linked_code[index] with the exported label value
		change the ESR to a REL
	}
}

// Resolve imports for m2
// Repeat previous step for imoprts from m2 and exports for m1
\end{minted}

\begin{minted}[frame=lines, linenos, fontsize=\large]
{cpp}
// Combine the tables for the linked file
linked_table = concatenate modified m1.table and modified m2.table

// Compute the header information
endCode = 12 + linked_code size in bytes
endModule = endCode + linked_table size in bytes

// Output the MERL file
output merl cookie
output endModule
output endCode
output linked_code
output linked_table
\end{minted}

Linking Example

\begin{tcolorbox}
\begin{verbatim}
m1.asm

.import b
.export f
f: .word f
    .word b
l: word l
\end{verbatim}
\end{tcolorbox}

\begin{tcolorbox}
\begin{verbatim}
;; HEADER of m1
0x00: beq $0, $0, 2 ; header: beq
0x04: 0x48 ; header: endModule
0x08: 0x18 ; header: endCode
;; CODE
0x0c: 0x0c ; f: .word f
0x10: 0x00 ; .word b ; placeholder
0x14: 0x14 ; l: .word l
;; FOOTER
0x18: 0x01 ; footer: relocation entry
0x1c: 0x0c ; relocation entry at 0x0c for f:
0x20: 0x01 ; footer: relocation entry
0x24: 0x14 ; relocation entry at 0x14 for l:
0x28: 0x11 ; footer: external symbol reference (ESR)
0x2c: 0x10 ; address where ESR is used, i.e., "b"
0x30: 0x01 ; length of the label "b"
0x34: 0x62 ; ASCII for "b"
0x38: 0x05 ; footer: external symbol definition (ESD)
0x3c: 0x0c ; address where ESD is defined, i.e., "f"
0x40: 0x01 ; length of the label "f"
0x44: 0x66 : ASCII for "f"
\end{verbatim}
\end{tcolorbox}

\begin{tcolorbox}
\begin{verbatim}
m2.asm

.import f
.export b
    .word f
b: .word b
l: .word l
\end{verbatim}
\end{tcolorbox}

\begin{tcolorbox}
\begin{verbatim}
;; HEADER of m2
0x00: beq $0, $0, 2 ; header: beq
0x04: 0x48 ; header: endModule
0x08: 0x18 ; header: endCode
;; CODE
0x0c: 0x00 ; .word f ; placeholder
0x10: 0x10 ; b: .word b
0x14: 0x14 ; l: .word l
;; FOOTER
0x18: 0x01 ; footer: relocation entry
0x1c: 0x10 ; relocation entry at 0x0c for b:
0x20: 0x01 ; footer: relocation entry
0x24: 0x14 ; relocation entry at 0x14 for l:
0x28: 0x11 ; footer: external symbol reference (ESR)
0x2c: 0x0c ; address where ESR is used, i.e., "f"
0x30: 0x01 ; length of the label "f"
0x34: 0x66 ; ASCII for "f"
0x38: 0x05 ; footer: external symbol definition (ESD)
0x3c: 0x10 ; address where ESD is defined, i.e., "b"
0x40: 0x01 ; length of the label "b"
0x44: 0x62 : ASCII for "b” 
\end{verbatim}
\end{tcolorbox}

Linked code (after a lot of work)

\begin{tcolorbox}
\begin{verbatim}
;; HEADER
0x00: beq $0, $0, 2 ; header: beq
0x04: 0x74 ; header: endModule
0x08: 0x24 ; header: endCode
;; CODE from m1.
0x0c: 0x0c ; f: .word f
0x10: 0x1c ; .word b
0x14: 0x14 ; l: .word l
;; CODE from m2.
0x18: 0x0c ; .word f ; placeholder
0x1c: 0x1c ; b: .word b
0x20: 0x20 ; l: .word l
\end{verbatim}
\end{tcolorbox}

\begin{tcolorbox}
\begin{verbatim}
;; FOOTER of m1
0x24: 0x01 ; footer: relocation entry
0x28: 0x0c ; relocation entry at 0x0c for f:
0x2c: 0x01 ; footer: relocation entry
0x30: 0x14 ; relocation entry at 0x14 for l:
0x34: 0x01 ; footer: relocation entry
0x38: 0x10 ; relocation entry at 0x10 for b:
0x3c: 0x05 ; footer: external symbol definition (ESD)
0x40: 0x0c ; address where ESD is defined, i.e., "f"
0x44: 0x01 ; length of the label "f"
0x48: 0x66 : ASCII for "f"
;; FOOTER of m2
0x4c: 0x01 ; footer: relocation entry
0x50: 0x1c ; relocation entry at 0x0c for b:
0x54: 0x01 ; footer: relocation entry
0x58: 0x20 ; relocation entry at 0x14 for l:
0x5c: 0x01 ; footer: relocation entry
0x60: 0x18 ; relocation entry at 0x18 for f:
0x64: 0x05 ; footer: external symbol definition (ESD)
0x68: 0x1c ; address where ESD is defined, i.e., "b"
0x6c: 0x01 ; length of the label "b"
0x70: 0x62 : ASCII for "b”
\end{verbatim}
\end{tcolorbox}

\section{Lecture 23}\label{lecture-23}

Heap Management

In this course we have a library to deal with all of the memory
management features, including \texttt{init}, \texttt{new}, and
\texttt{delete}.

This allows for data to exist in memory that is out of scope, that is,
out of the boundaries of your stack frame.

\qn{}{ How do we manage this memory?
}

The memory not on the stack is either in code, or on the heap. The
\texttt{init} procedure initializes a heap for us to use.

This is much more problematic than a stack to take care of. Stacks are
nice and ordered. Calls can be made to heaps using \texttt{delete} or
\texttt{new} in arbitrary orders, so we can't simply push and pop
memory.

In our world:


\begin{table}[h]
    \centering
    \begin{tabular}{|c|}\hline
        Code\\\hline
        Read-only data\\\hline
        Global data\\\hline
        Heap

Stack\\\hline
    \end{tabular}
    
\end{table}

Code is just data. You can store data in your code. In C, it's common to
separate out static, global information from code, but it's all just the
stuff before the dynamic heap.

\qn{}{ How does a heap work?
}

We have a variety of implementations.

Example 1: No reclamation of Memory and Fixed Blocks

After \texttt{init}, we get two pointers, one to the start of memory on
the heap and one at the end.

Initialization is \(O(1)\). Allocation is also \(O(1)\). We never
delete.

Clearly not the best choice. We will run out of memory quickly since we
aren't reusing reclaimed memory.

Example 2: Explicit Reclamation and Linked List of Fixed-Size Blocks

Keep the fixed size idea the same, but keep track of a free list (linked
list of free memory blocks) and we can allocate from this linked list.

Example 3: Variable-Sized Blocks

We once again used a linked list but here our linked list will store a
number of bytes and the next node.

Init: Start with the entire heap being free.

Let's say we want to allocate 50 bytes. What we will do is allocate 54
bytes.

The first 4 bytes are the size of the block (integer), and the rest is
the requested bytes. We need this bookkeeping because delete doesn't
take a size. We return a pointer to the start of the 50 bytes.

Memory:


\begin{table}[h]
    \centering
    \begin{tabular}{|c|c|c|c|c|}\hline
        54 & start 50 $p$ & $\ldots$ & end 50 & \\\hline
    \end{tabular}
    
\end{table}

Free List is one node with \(970\) and \(0x4036\) (since we started with
\(1024\) bytes at address \(0x4000\)).

Next, we allocate 28 bytes. We allocated 32 bytes and return a pointer
to the start of the 28 bytes.

\begin{table}[h]
    \centering
    \begin{tabular}{|c|c|c|c|c|c|c|}\hline
        54 & $p$ & $\ldots$ & 32 & $q$ & $\ldots$ & end 28 \\\hline
    \end{tabular}
    
\end{table}

Free list is one node with \(938\) and \(0x4056\)

Freeing the 50 bytes results in

\begin{table}[h]
    \centering
    \begin{tabular}{|c|c|c|}\hline
        32 & $q$ & $\ldots$ \\\hline
    \end{tabular}
    
\end{table}

Free list is \(54\) \(0x4000\) pointing to \(938\) \(0x4056\).

Freeing the other 32 bytes results in a free list of \(54\) \(0x4000\)
pointing to \(32\) \(0x4036\) pointing to \(938\) \(0x4056\)

We can do some consolidation. Notice that \(54 + 0x4000 = 0x4036\) and
so the first two nodes in our linked list can collapse to a single node.

We can further consolidate since \(86 + 0x4000 = 0x4056\).

The biggest issue with this approach is fragmentation. Suppose we have
48 bytes and we make the following calls:

\begin{itemize}
    \item Allocate \(4 + 8\)
    \item Allocate \(4 + 16\)
    \item Allocate \(4 + 4\)
    \item Free \(20\)
    \item Allocate
\(4 + 4\)
\end{itemize}


We can't allocate \(16\), despite having \(24\) bytes free.

We've been showing the free list as a separate data structure. But we're
the ones allocating the data structures. So how do we allocate space for
the free itself?

The free list goes in the space itself. It is free, so the user doesn't
care what we put there.

\section{Lecture 24}\label{lecture-24}

Dealing with Fragmentation

Heuristics:

\begin{itemize}
    \item First fit: Put memory in the first available spot
    \item Best
fit: Put the block in an exact match (or as close to it so there is less
waste)
    \item Worst fit: Exact match if possible, otherwise put the block in
the largest available space
\end{itemize}


Problem: Best- and worst-fit involve looking over the whole free list
(so they are slow).

Other ideas include \texttt{dmalloc} and the binary buddy system.

Binary Buddy System

Start with \(512\) bytes of heap memory.

Suppose we try to allocate \(19\) bytes. We need an extra one for
bookkeeping (so \(20\)). This fits in a block of size \(2^5 = 32\). We
split memory until we find such a block and reserve the entire block.


\begin{table}[h]
    \centering
    \begin{tabular}{|c|c|}\hline
         256& 256\\\hline
    \end{tabular}
    
\end{table}

\begin{table}[h]
    \centering
    \begin{tabular}{|c|c|c|}\hline
         128& 128 & 256\\\hline
    \end{tabular}
    
\end{table}

\begin{table}[h]
    \centering
    \begin{tabular}{|c|c|c|c|}\hline
         64& 64 & 128&256\\\hline
    \end{tabular}
    
\end{table}

\begin{table}[h]
    \centering
    \begin{tabular}{|c|c|c|c|c|}\hline
         32& 32 & 64&128&256\\\hline
    \end{tabular}
    
\end{table}

Binary buddy still creates fragmentation, just of a different sort. If
most allocations are of nice powers of two, it defragments well.

Both the size and location of any block can be encoded using a list of
``left'' and ``right'' actions.

Binary Buddy Code

Start with a \(1\). The code for the whole of the heap is just \(1\)

For each block division, append a \(0\) if the left block is selected,
or a \(1\) if the right block is selected.

Since the first bit is always \(1\), we can tell the length of the code
by looking for the first \(1\).

Since each bit represents a power of two, very short codes represent a
lot of information.

One word is plenty for a code.

Some languages take care of deallocation.

\begin{minted}[frame=lines, linenos, fontsize=\large]
{cpp}
int f() {
	Myclass ob = new MyClass(); 
} // ob no longer accessible
// garbage collector reclaims
\end{minted}


Second example

\begin{minted}[frame=lines, linenos, fontsize=\large]
{cpp}
int f() {
	MyClass ob2 = null;
	if (x == y) {
		MyClass ob1 = new MyClass();
		ob2 = ob1;
	} // ob1 goes out of scope
	// ob2 still holds the object
} // ob2 no longer accessible
\end{minted}

In order for automatic memory management to happen, the compiler and the
allocator need to coordinate in some way.

At the minimum, the compiler needs to tell the allocator when something
goes out of scope.

Technique 1: Reference Counting

For each heap block, keep track of the number of pointers that point to
it.

Must watch every pointer and update reference counts each time a pointer
is reassigned. The compiler needs to call some procedure provided by the
allocator every time a pointer goes into or out of scope.

If a block's reference count reaches \(0\), reclaim it.

\begin{minted}[frame=lines, linenos, fontsize=\large]
{cpp}
struct List {
	List *next;
	int val;
};
l = new List;
l->next = new List;
// l->next is not a variable in scope
// it's a field of an object, and there is
// a reference to that object in scope
\end{minted}

\qn{}{ What issues are there to this?
}

If a block points to another block and vice versa, then the cluster is
unreachable and should be cleaned.

But both will retain a reference to each other.

\qn{}{If reference counting is so bad, why do people use it?
}

It's very straightforward to implement, so it's an easy way to get some
automatic memory management easily.

But it's expensive and flawed.

\qn{}{ Is it possible to fix this issue with reference
counting?
}

Short answer: No.~Long answer: What if we delayed counting references,
so that we would never get these problematic clusters in the first
place.

The remaining techniques are classed as garbage collection.

Technique 2: Mark and Sweep

Scan the entire stack (and global variables) and search for pointers.
Mark the heap blocks that have pointers to them. If these contain
pointers, continue following.

Then scan the heap, reclaim any blocks that aren't marked. Boils down to
a graph traversal problem.

The compiler needs to tell the allocator
\begin{itemize}
    \item Where the pointers are in the stack
    \item Where the pointers are in the heap
\end{itemize}

This means that the compiler and allocator need to agree on both stack
layout and object layout.

This is a much greater degree of co-design than is needed for reference
counting (so mark and sweep is rarely bolted onto a language post-hoc).

Technique 3: Copying Collector

Split the heap into two halves, say \(H_1\) and \(H_2\).

Allocate memory in \(H_1\). Perform a mark-and-copy; but mark by coping
objects from \(H_1\) into \(H_2\).

After the copy \(H_2\) has all living objects stored contiguously.

Once finished copying, begin allocation to \(H_2\) (reverse the roles).

Three major benefits

\begin{itemize}
    \item Allocation is always extremely fast (filling a heap)
    \item The "sweep" phase is free
    \item We have no fragmentation in memory
\end{itemize}

If most objects aren't long-lived, we don't actually do much.

Since copying an object from one heap to an object is moving objects,
the compiler needs to be prepared for the allocator to modify active
memory.

In practice, most objects are short-lived (good for copy), but those
objects that aren't short lived are nearly immortal.

Most practical garbage collectors are generational collectors: they use
copying from \(H_1\) to \(H_2\), but then \(H_2\) is mark-and-sweep.

With generational, we benefit from the free sweep for young objects, and
the lack of copying for old objects.

\end{document}