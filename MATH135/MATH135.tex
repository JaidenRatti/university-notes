\documentclass{article}
\usepackage[utf8]{inputenc}
\usepackage[T1]{fontenc}
\usepackage[utf8]{inputenc}
\usepackage{lmodern}
\usepackage{xhfill}
\usepackage[a4paper, margin=1in]{geometry}
\usepackage{parskip}
\usepackage{fancyhdr}
\usepackage{color,soul}
\usepackage{amssymb}
\usepackage{graphicx}
\usepackage{comment}
\usepackage{amsmath}
\usepackage{amsthm}
\usepackage{amsmath}
\usepackage{amssymb}
\usepackage{mathtools}
\usepackage{blindtext}
\usepackage{titlesec}
\usepackage{hyperref}

\newcommand\vv{\vec}


\pagestyle{fancy}
\lhead{1229}
\chead{MATH 135: Algebra}
\rhead{Jaiden Ratti}

\usepackage{minted}
\large
\title{MATH 135: Algebra}
\begin{document}
\begin{titlepage}
	\begin{center}
    \line(1,0){300}\\
    [0.65cm]
	\huge{\bfseries Language and Proofs in Algebra}\\
	\line(1,0){300}\\
	\textsc{\Large MATH135}\\
	\textsc{\Large  Jaiden Ratti}\\
        \textsc{\Large Prof. J.P. Pretti}\\
    \textsc{\Large 1229}\\
	[5.5cm]
	\end{center}
\end{titlepage}




\tableofcontents

\pagebreak

\section{Introduction to the Language of Mathematics}

\subsection{Sets}

Sets are not ordered. 

$\{7, \pi\} = \{\pi, 7\}$

Denote element of set by $7 \in \{2,7,3\}$. $\{7\} \notin \{7,3,2\}$, but $\{7\} \in \{\{7\},3,2\}$.

$\{\} = \emptyset$, $\emptyset \ne \{\emptyset\}$

$\emptyset \notin \{7,3\}$, $\emptyset \notin \emptyset$

$\mathbb{Z} \to$ set of integers.\\
$\mathbb{N} \to$ set of natural numbers.\\
$\mathbb{Q} \to$ set of rational numbers.\\
$\mathbb{R} \to$ set of real numbers.

\subsection{Mathematical Statements and Negation}

Statements are true or false. 

$9+6 =15$ is a statement

$x > 2$ is not a statement (Open sentence. If you knew $x$, it would be a statement)

$10 > 7$ is a statement

Open sentence $\ne$ statement. 

\underline{Negation}

$P$ is a statement

Negation of $P (\neg P)$ is true when $P$ is false.

\subsection{Quantifiers and Quantified Statements}

\subsubsection{Universal and Existential Quantifiers}

$x^2 - x \ge 0$ is an open statement.

$\forall x \in \mathbb{N}, x^2 - x \ge 0$. This is "for all natural numbers $x, x^2-x\ge0$" We know this is true.

Changing the domain makes it false.

$\forall x \in \mathbb{R}, x^2-x \ge 0$

When domain is empty $(\forall x \in \emptyset)P(x)$ is always true.

$\forall x \in \emptyset, x^2 - x  \ge 0$ is true. All elephants in the room have 20 legs $\ddot\smile$

Let $x \in \mathbb{R} \leftarrow$ universally quantifying the following statement.

\underline{Existential Quantifier}

$\exists x \in S, P(x)$. This is "there exists a number $x$ in the set $S$ such that $P(x)$ is true." There just has to be one such case.

$\exists m \in \mathbb{Z}, \frac{m-7}{2m+4} = 5, m = -3. \therefore true$.

Once again, domain matters. 

$\exists x \in \emptyset, P(x)$ is always false.

Exercises

\begin{align*}
    64 \text{ is a perfect square} &\iff \exists x \in \mathbb{Z}, x^2 = 64 \\
    y = x^3 - 2x + 1 \text{ has no } x \text{-ints} &\iff \forall x \in \mathbb{R}, x^3-2x+1 \ne 0 \\
    &\iff \neg(\exists x \in \mathbb{R}, x^3-2x+1=0) \\
    2^{2a-4} = 8 \text{ has a rational solution} &\iff \exists a \in \mathbb{Q}, 2a-4=3 \\
    \frac{n^2+n-6}{n+3} \text{ is an integer as long as } n \text{ is an integer} &\iff \forall n \in \mathbb{Z}, \frac{n^2+n-6}{n+3} \in \mathbb{Z}
\end{align*}


\subsubsection{Negating Quantifiers}

Everybody in this room was born before 2010 $\leftarrow$ Universal

Somebody in this room was born after 2010, or on 2010 $\leftarrow$ Existential

$\forall x \in S, P(x)$ is false when there is at least one $x \in S$ for which $P(x)$ is false.
\begin{align*}
    \neg(\forall x \in S, P(x)) &\equiv \exists x \in S, (\neg P(x)) \\
    \neg(\exists x \in S, P(x)) &\equiv \forall x \in S, (\neg P(x))
\end{align*}

We cannot just change all the signs since $P(x)$ might be complicated.

$\forall x \in \mathbb{R}, |x| < S$. Negation: $\exists x \in \mathbb{R}, |x| \ge S$

Someone in this room was born before 1990.
Everyone in this room was born after or during 1990 is the negation. 

$\exists x \in \mathbb{Q}, x^2 = S$. Negation: $\forall x \in \mathbb{Q}, x^2 \ne S$.

\subsection{Nested Quantifiers}

$\forall x \in \mathbb{R}, \forall y \in \mathbb{R}, x^3-y^3 = 1$ is false for every $x$ and every $y$. 

$\forall x \in \mathbb{R}, \exists y \in \mathbb{R}, x^3-y^3 = 1$ is true. $\exists$ is in the open statement

$\exists x \in \mathbb{R}, \exists y \in \mathbb{R}, x^3 - y^3 = 1$ is true. 

$\exists x \in \mathbb{R}, \forall y \in \mathbb{R}, x^3 - y^3 = 1$ is false. If $x$ was fixed, there is no way every $y$ will work. 

\section{Logical Analysis of Mathematical Statements}

\subsection{Logical Operators}

Statement represented by $A$. 


\begin{table}[!h]
    \centering
    \begin{tabular}{|c|c|} \hline 
        $A$ & $\neg A$\\ \hline 
        $T$ & $F$\\ \hline 
        $F$ & $T$\\ \hline
    \end{tabular}
\end{table}

\underline{Conjunction and Disjunction}

$A$ and $B$ $\equiv$ $A \wedge B$ is 

\begin{table}[!h]
    \centering
    \begin{tabular}{|c|c|c|} \hline 
        $A$ & $B$ & $A \wedge B$\\ \hline 
        $T$ & $T$ & $T$\\ \hline 
        $T$ & $F$ & $F$\\ \hline 
        $F$ & $T$ & $F$\\ \hline 
        $F$ & $F$ & $F$\\ \hline
    \end{tabular}
\end{table}

$\sqrt{2}$ is irrational and $3 > 2$ is true.

10 is even and $1=2$ is true.

$\forall x \in \mathbb{N}, (x > x - 1) \wedge (2x > x)$ is true.

$\forall x \in \mathbb{Z}, (x > x-1) \wedge (2x >x)$ is false.

$A$ or $B$ $\equiv$ $A \vee B$ is 

\begin{table}[!h]
    \centering
    \begin{tabular}{|c|c|c|} \hline 
        $A$ & $B$ & $A \vee B$\\ \hline 
        $T$ & $T$ & $T$\\ \hline 
        $T$ & $F$ & $T$\\ \hline 
        $F$ & $T$ & $T$\\ \hline 
        $F$ & $F$ & $F$\\ \hline
    \end{tabular}
\end{table}

$5 \le 6$ is true.

87 is a prime number of $14x=25$ has $x \in \mathbb{Z}$ is false.

16 is a perfect square or 15 is a multiple of 3 is true.


\underline{Logical Equivalence}

$A \equiv \neg(\neg A)$. $A$ is logically equivalent to not not $A$.

\underline{De Morgan's Laws}
\begin{align*}
    \neg (A \vee B) &\equiv (\neg A) \wedge (\neg B) \\
    \neg (A \wedge B) &\equiv (\neg A) \vee (\neg B)
\end{align*}


\begin{table}[!h]
    \centering
    \begin{tabular}{|c|c|c|c|c|c|c|} \hline 
         $A$& $B$ & $A \vee B$ & $\neg(A \vee B)$ & $\neg A$ & $\neg B$ & $(\neg A) \wedge (\neg B)$\\ \hline 
         $T$ & $T$ & $T$ & $F$ & $F$ & $F$ & $F$\\ \hline 
         $T$ &  $F$& $T$ & $F$ & $F$ & $T$ & $F$\\ \hline 
         $F$& $T$ & $T$ & $F$ & $T$ &  $F$& $F$\\ \hline 
         $F$& $F$ & $F$ & $T$ & $T$ & $T$ & $T$\\ \hline
    \end{tabular}
\end{table}

Example, show
\begin{align*}
    &\neg(A \wedge (\neg B \wedge C)) \equiv \neg (A \wedge C) \vee B \\
    &\neg(A \wedge(\neg B \wedge C)) \\
    &\equiv (\neg A) \vee \neg(\neg B \wedge C) \\
    &\equiv (\neg A) \vee (B \vee \neg C) \\
    &\equiv (\neg A) \vee (\neg C \vee B) \\
    &\equiv (\neg A \vee \neg C) \vee B \\
    &\equiv \neg(A \wedge C) \vee B
\end{align*}


\subsection{Implication}

"If $H$ then $C$", $H \implies C$

Equivalent to $(\neg H) \vee C$

$H$ = hypothesis, $C$ is conclusion

\begin{table}[!h]
    \centering
    \begin{tabular}{|c|c|c|} \hline 
        $H$ & $C$ & $H \implies C$\\ \hline 
        $T$ & $T$ & $T$\\ \hline 
        $T$ & $F$ & $F$\\ \hline 
        $F$ & $T$ & $T$\\ \hline 
        $F$ & $F$ & $T$\\ \hline
    \end{tabular}
\end{table}

$\sqrt{2}$ is irrational, $3^3 = 27 \leftarrow$ True.

$\sqrt{2}$ is irrational, $3^3 = 28 \leftarrow$ False.

$\sqrt{2}$ is rational, $3+4=6 \leftarrow$ True.

$\sqrt{2}$ is rational, $3+4=7 \leftarrow$ True.

For all real numbers $x$, if $x > 2, x^2 > 4 \leftarrow$ True.

For all real numbers $x$, if $x \ge 2, x^2 > 4 \leftarrow$ True.

$\forall k \in \mathbb{Z}$, if $k > 3$, then $2k+1 \ge 9$ is true.

$\forall k \in \mathbb{Z}$, if $k > 3$, then $2k+1 \ge 10$ is false.

$\forall k \in \mathbb{Z}$, if $k > 3$, then $2k + 1 \ge 8$ is true.

$\forall x \in \mathbb{R} (x \ge 7 \implies x + \frac{1}{x} \ge 2)$

For all $x \in \mathbb{R}$, if $x \ge 7$, then $x + \frac{1}{x} \ge 2$

$x \in \mathbb{R} \wedge x \ge y \implies x + \frac{1}{x} \ge 2$

$x + \frac{1}{x} \ge 2$ whenever $x \in \mathbb{R}$ and $x \ge 7$


\underline{Negation of Implication}
\begin{align*}
    \neg (H \implies C) \equiv \neg((\neg H) \vee C) \equiv (\neg(\neg H)) \wedge (\neg C)) \equiv H \wedge (\neg C)
\end{align*}

Negation of implication is not an implication. 

If 7 is a prime and $5 \le 6$, then 24 is a perfect square (false). 

7 is prime and $5 \le 6$ and 24 is not a perfect square (true).

Negation of implication is and. Hypothesis is not always first.

\underline{Implication Examples}

For all $a,b,x,\in \mathbb{R}$

\begin{enumerate}
    \item If $a < b$, then $a \le b$ (true)
    \item If $|x| = 3$, then $x^2 = 9$ (true)
\end{enumerate}


\subsection{Contrapositive and Converse}

\underline{Contrapositive}

The contrapositive of $A \implies B$ is the implication $\neg B \implies \neg A$

\begin{enumerate}
    \item If $a > b$, then $a \ge b$ (true)
    \item If $x^2 \ne 9$, then $|x| \ne 3$ (true)
\end{enumerate}

Logically equivalent with $A \implies B$

\underline{Converse}

The converse of $A \implies B$ is the implication $B \implies A$

\begin{enumerate}
    \item If $a \le b$, then $a < b$ (false)
    \item If $x^2=9$, then $|x| = 3$ (true)
\end{enumerate}

Not logically equivalent with $A \implies B$


\subsection{If and Only If}

Logical operator $\iff$

For all $x \in \mathbb{R}, |x| = 3$ iff $x^2 = 9$

True both ways. 

\begin{table}[!h]
    \centering
    \begin{tabular}{|c|c|c|} \hline 
         $A$ & $B$ & $A \iff B$\\ \hline 
         $T$& $T$ & $T$\\ \hline 
         $T$& $F$ & $F$\\ \hline 
         $F$& $T$ & $F$\\ \hline 
         $F$& $F$ & $T$\\ \hline
    \end{tabular}
\end{table}
$2+2 = 5$ iff $3 + 3 = 7$ is True


\section{Proving Mathematical Statements}

Prove:
\begin{align*}
    x^4 + x^2y + y^2 \ge 5x^2y-5y^2 \\
\end{align*}

Let $x,y \in \mathbb{R}$
\begin{align*}
    0 &\le (x^2 - 2y)^2 \\
    &= x^4 - 4x^2y + 4y^2 \\
    &= x^4 - 5x^2y + x^2y + 5y^2 + y^2
\end{align*}


Faulty logic: Prove $7=-7$ by squaring both sides

\subsection{Proving Universally Quantified Statements}

Proving $\forall x \in S, P(x)$

We can consider arbitrary $x \in S,$ and argue that $P(x)$ must be true (direct proof). 

\underline{Prove an identity}

Prove

$max\{x,y\} = \frac{x+y+|x-y|}{2}$ for all $x,y\in \mathbb{R}$

\underline{Case 1}: $x \ge y$. In this case $max\{x,y\} = x$. And $\frac{x+y+x-y}{2} = x$

\underline{Case 2}: $ x < y$. In this case $max\{x,y\} = y$. And $\frac{x+y+(-x+y)}{2} = y$

In both cases, LHS = RHS $\blacksquare$

\underline{Disprove Universally Quantified Statement}

$\forall x \in \mathbb{R}, (x^2 - 1)^2 \ge 0$

A counter example is $1 \in \mathbb{R}$.

Single example doesn't prove $\forall x \in S, P(x)$ is true.

Single counter example does prove $\forall x \in S, P(x)$ is false.

\subsection{Prove Existentially Quantified Statements}

There exists a perfect square $k$ such that $k^2 - \frac{31}{2}k = 8$.

Consider $k = 16$. Since $k=4^2, k$ is a perfect square. Also $k^2 - \frac{31}{2}k = 256 - 248 = 8$ completing the proof.

\underline{Disprove Existential Statement}

We will prove the negation is true.

"There exists a real number $x$ such that $\cos2x + \sin2x = 3$"

"For all real numbers $x$ such that $\cos2x + \sin2x \ne 3$"

$x \in \mathbb{R}$

Since $\cos2x, \sin2x \le 1$, then

$\cos2x + \sin2x \le 2 \blacksquare$


For all $k \in \mathbb{N}$, there exists $x \in \mathbb{R}$, such that $\log_kx^5 = \frac{1}{2}$

\underline{Proof}

Let $k \in \mathbb{N}$. Consider $x = k^{\frac{1}{10}}$. Clearly $x \in \mathbb{R}$. Moreover, $\log_kx^5 = \log_k(k^{\frac{1}{10}})^5 = \log_kk^{\frac{1}{2}} = \frac{1}{2}$


\subsection{Proving Implications}

If $m$ is an even integer, then $7m^2 + 4$ is an even integer. 

\underline{Proof}

Assume $m$ is an even integer.

That is $m = 2k$ for some integer $k \in \mathbb{Z}$

We must show $\exists \ell \in \mathbb{Z}, 7m^2 + 4 = 2\ell$

We have $7m^2 + 4 = 7(2k)^2 + 4 = 2(14k^2 + 2)$

Since $k \in \mathbb{Z}$, then $14k^2 + 2 \in \mathbb{Z}$. That is, picking $\ell = 14k^2 + 2$ completes the proof.

For all integers $k$, if $k^5$ is a perfect square, then $9k^19$ is a perfect square
\begin{align*}
    &\text{Let } k \in \mathbb{Z} \\
    &\text{Assume } k^2 \text{ is a prefect square} \\
    &\text{That is } k^5 = n^2 \text{for some } n \in \mathbb{Z} \\
    &\text{then } 9k^{19} = (9k^{14})k^5 \\
    &= (9k^{14})n^2 \\
    &= (3k^7)^2 n^2 \\
    &= (3k^7n)^2
\end{align*}

Since $k,n \in \mathbb{Z}$, then $3k^7n \in \mathbb{Z}$. Thus $9k^{19}$ is a perfect square. $\blacksquare$

\subsection{Divisibility of Integers}

An integer $m$ divides an integer $n$ if there exists an integer $k$ so that $n = km$.

We write $m \vert n$ is $m$ divides $n$

$7 \vert 56, 7\vert-56, 7\vert0, 0\vert0$

$7 \nmid 55, 0 \nmid 7$

$\frac{7}{56}$ is a number, $7 \vert 56$ is a statement.

\subsubsection{Transitivity of Divisibility}

For all $a,b,c \in \mathbb{Z}$ if $a \vert b$ and $b \vert c$ then $a \vert c$.

\underline{Proof}

Let $a,b,c \in \mathbb{Z}$. Assume $a \vert b$ and $b \vert c$ then $b = ak$ and $c = b\ell$ for some $k,\ell \in \mathbb{Z}$.


Substituting gives $c = (ak) \ell = (k \ell)a$

Notice that $k\ell \in \mathbb{Z}$ because $k,\ell \in \mathbb{Z}$. Thus $a \vert c$ by the definition of divisibility.

\subsubsection{Divisibility of Integer Combinations}

For all $a,b,c$ if $a \vert b$ and $a \vert c$ then $a | (bx + cy)$ for all integers $x,y$.

e.g. $a = 5, b = 10, c = 25$

DIC $\rightarrow 5 \vert (10x + 25y)$ for all $x,y,\in \mathbb{Z}$

\underline{Proof}

Let $a,b,c \in \mathbb{Z}$. Assume $a \vert b$ and $a \vert c$. Then $ak = b$ and $a\ell = c$ for some $k, \ell \in \mathbb{Z}$. Now $bk + cy = akx + a\ell y = a(kx + \ell y)$

Since $k, x, \ell, y \in \mathbb{Z}$, then $kx + \ell y \in \mathbb{Z}$.

\underline{Proposition}

For all $a,b,c \in \mathbb{Z}$ if $a \vert b$ or $a \vert c$, then $a \vert bc$

Note 

Let $P,Q$, and R be statement variables
\begin{align*}
    (P \vee Q) \implies R &\equiv (P \implies R) \wedge (Q \implies R)
\end{align*}

\underline{Proof}

Let $a,b,c \in \mathbb{Z}$

First we prove $a \vert b \implies a \vert bc$

So suppose $b = ak$ for some $k \in \mathbb{Z}$

Then $bc = (ak)c = a(kc)$

Since $k,c \in \mathbb{Z}$, then $kc \in \mathbb{Z}$. Hence $a \vert bc$

To complete this proof, we must show $a \vert c \implies a \vert bc$. The argument in this case is similar $\blacksquare$.

\underline{Another Example}

For all $a,b,c \in \mathbb{Z}$ if for all $x \in \mathbb{Z}, a \vert (bx + c)$ then $a \vert (b + c)$

\underline{Proof}

Let $a,b,c \in \mathbb{Z}$

Assume $\forall x \in \mathbb{Z}, a \vert (bx + c)$

Choosing $x=1$, gives $a \vert (b + c)$

This is not choosing a number for all integers $x$. We are assuming the hypothesis is correct.

For all $a,b,c,x \in \mathbb{Z}$ if $a \vert (bx + c)$, then $a \vert (b + c)$

This is false. Counter example

$3 \vert (2(3) + 3)$ and $3 \nmid (2 + 3)$

TD: $\forall a,b,c \in \mathbb{Z}, (a \vert b \wedge b \vert c) \implies a \vert c$

$11 \vert 55$ and $55 \vert n$, we know $11 \vert n$, by TD.


\subsection{Proof of Contrapositive}

\underline{Example}

For all integers $x$, if $x^2 + 4x-2$ is odd, then $x$ is odd. 

\underline{Proof}

Let $x \in \mathbb{Z}$. We will show the contrapositive is true.

Assume $x$ is even. That is $x = 2k$ for some integer $k$. Substitute to get 

$x^2 + 4x - 2 =4k^2 + 8k - 2 = 2(2k^2 + 4k-1)$

Since $k$ is an integer, then $2k^2 + 4k-1 \in \mathbb{Z}$. That is $x^2 + 4x-2$ is even $\blacksquare$.

\underline{Example}

If $a,b \in \mathbb{R}$. If $ab$ is irrational then $a$ is irrational or $b$ is irrational.

\underline{Proof}

Let $a,b \in \mathbb{R}$. We will use the contrapositive.

Assume $a = \frac{p}{q}$ and $b = \frac{r}{s}$ for some integers $p,q,r,s \in \mathbb{Z}$ where $q,s \ne 0$.

Then $ab = \frac{rp}{qs}$ moreover since $p,q,r,s \in \mathbb{Z}$ then $rp, qs \in \mathbb{Z}$. Also $qs \ne 0$. That is $ab$ is rational.


\underline{Example}

Let $x \in \mathbb{R}$. If $x^3 + 7x^2 < 9$, then $x < 1.1$.

\underline{Proof}

Let $x \in \mathbb{R}$. Suppose $x \ge 1.1$ then $x^3 + 7x^2 \ge (1.1)^3 + 7(1.1)^2 > 9.8 > 9$.

We get that $x^3 + 7x^2 \ge 9$. Therefore the contrapositive is true, proving the original statement is true as well.

\underline{Example}

Let $a,b,c \in \mathbb{Z}$

If $a \vert b$ then $b \nmid c$ or $a \vert c$.

\underline{Proof}

Let $a,b,c \in \mathbb{Z}$.

Using "elimination", assume $a \vert b$ and $b \vert c$. By TD $a \vert c$.

Why does this work?
\begin{align*}
    (A \implies (B \vee C)) \equiv A \wedge \neg B \implies C
\end{align*}

\subsection{Proof by Contradiction}

$A$ or $\neg A$ must always be false.

$A \wedge (\neg A)$ is always false, calling it true is a contradiction. 

We can prove that statement $P$ is true by, assuming $\neg P$ is true then based on this assumption, prove that both $Q$ and $\neg Q$ are true for some statement $P$. 

Prove that $\neg(\exists a, b \in \mathbb{Z}, 10a + 15b = 12)$

By way of contradiction (BWOC), assume that $10a + 15b = 12$ for some $a,b, \in \mathbb{Z}$. Then $5(2a + 3b) = 12.$ Since $2a + 3b \in \mathbb{Z}$, then $5 \vert 12$. However we know that $5 \nmid 12$. This is a contradiction, completing the proof.

Prove $\sqrt{2}$ is irrational. 

Assume it is rational, $\sqrt{2} \in \mathbb{Q}$.

$\sqrt{2} = \frac{a}{b}$ where $a,b$ are integers $> 0$.

Assume they are \underline{not} even. If they were even, $a = 2c$ and $b = 2d$ and thus $c < a$ and $d < b$.

$\frac{a}{b} = \frac{2c}{2d} = \frac{c}{d}$

$\frac{a}{b} = \sqrt{2}$

$a^2 = 2b^2$

$2 \vert a^2$, so $a^2$ is even.


Assume its odd

$a^2 = (2k+1)^2 = 4k^2 + 4k+1 = 2(2k^2+2k)+1$. $a$ must be even.

$\exists$ an integer $m$ such that $a = 2m$, 

$b^2 = 2m^2$. $b$ must be even then which is a contradiction. 

$\therefore \sqrt{2}$ is irrational.

$\neg(A \implies B) \equiv (A \wedge (\neg B))$

Proving $A \implies B$ is true by contradiction, we assume $A \implies B$ is false. $A$ is true, $B$ is false. If we can prove this is a contradiction, $A \implies B$ is true.


$\forall a,b,c \in \mathbb{Z}$ if $a \vert (b + c)$ and $a \nmid b$, then $a \nmid c$. 

For sake of contradiction, there exists integers $a,b,c$ such that $a \vert (b + c)$ and $a \nmid b$ and $a \vert c$. 

By DIC we have $a \vert [(1)(b+c) + (-1)c] = a \vert b$ contradiction. 

\subsection{Proving If and Only If Statements}

\underline{Example}

Let $x,y \in \mathbb{R}$ where $x,y \ge 0$. Then $x = y$ iff $\frac{x+y}{2} = \sqrt{xy}$

\underline{Proof}

Let $x,y \in \mathbb{R}$ where $x,y \ge 0$. 

We will prove this in both directions ($\rightarrow$)

Assume $x=y, \frac{y+y}{2} \rightarrow y \leftarrow \sqrt{yy}$.

($\leftarrow$) Assume $\frac{x+y}{2} = \sqrt{xy}$

$\implies x + y = 2\sqrt{xy}$\\
$\implies (x+y)^2 = 4xy$\\
$\implies x^2 - 2xy+y^2=0$\\
$\implies (x-y)^2 = 0$\\
$\implies x-y = 0$\\
$\implies x = y$


\section{Mathematical Induction}


\subsection{Notation for Summations, Products and Recurrences}

\underline{Summation Notation}
\begin{align*}
    \sum_{k=3}^{7}k^2 = 3^2 + 4^2 + 5^2 + 6^2 + 7^2 = 135
\end{align*}

\underline{Product Notation}
\begin{align*}
    \prod_{k=1}^{3}(5-k)! = 4! \cdot 3! \cdot 2! = 288
\end{align*}

\subsection{Proof by Induction}

Statement
\begin{align*}
    \sum_{i=1}^{n}i(i+1) = \frac{1}{3}n(n+1)(n+2) \quad \forall n \in \mathbb{N}
\end{align*}

\underline{Proof}

We will proceed by induction on $n$. 

\underline{Base Case}

We consider when $n=1$

Then 
\begin{align*}
    \sum_{i=1}^ni(i+1) = \sum_{i=1}^1i(i+1) = 1(1+1)=2
\end{align*}

And 
\begin{align*}
    \frac{1}{3}n(n+1)(n+2) = \frac{1}{3}(1)(2)(3) = 2
\end{align*}

That is, the statement is true when $n=1$.

\underline{Inductive Step}

Let $k$ be an arbitrary natural number. 

Assume 

$\sum_{i=1}^ki(i+1) = \frac{1}{3}k(k+1)(k+2)$

Consider when $n = k+1$

Then 
\begin{align*}
    \frac{1}{3}n(n+1)(n+2) = \frac{1}{3}(k+1)(k+2)(k+3)
\end{align*}

And
\begin{align*}
    \sum_{i=1}^ni(i+1) &= \sum_{i=1}^{k+1}i(i+1) \\
    &= (\sum_{i=1}^ki(i+1)) + (\sum_{i=k+1}^{k+1}i(i+1)) \\
    &= \frac{1}{3}k(k+1)(k+2) + (k+1)(k+2) \text{ by our inductive hypothesis} \\
    &= \frac{1}{3}k(k+1)(k+2) + \frac{3}{3}(k+1)(k+2) \\
    &= \frac{1}{3}(k+1)(k+2)(k+3)
\end{align*}

That is, the statement is true when $n = k+1$. Therefore by POMI, the proof is complete.

\underline{POMI}

Let $P(n)$ be a statement that depends on $n \in \mathbb{N}$. If statement 1 and 2 are true
\begin{enumerate}
    \item $P(1)$
    \item For all $k \in \mathbb{N}$, if $P(k)$, then $P(k+1)$
\end{enumerate}
Then statement 3 is true.
\begin{enumerate}
    \setcounter{enumi}{2}
    \item For all $n \in \mathbb{N}, P(n)$
\end{enumerate}

$P(1) \implies P(2) \implies P(3) \implies P(4)$

POMI doesn't have to start at 1. 


Let $P(n)$ be the open sentence

$6 \vert (2n^3 + 2n^2 + n)$

Prove $P(n)$ is true for all $n$. 

\underline{Base Case}
$P(1), 6 \vert 6 \checkmark$

Assume $P(k)$ is true

$6 \vert (2k^3 + 3k^2 + k)$

\underline{Inductive Step}
$6 \vert (2(k+1)^3 + 3(k+1)^2 + (k+1))$

$2(k^3+3k^2 + 3k+1) + 3(k^2 + 2k+1) + (k+1)$

$\underbrace{2k^3 + 3k^2 + k}_{6\text{ divides this}} + \underbrace{6k^2 + 6k + 6k +6}_{6 \text{ divides this}}$

6 divides the sum by DIC.

\subsection{Binomial Coefficients}

$\binom{5}{2} \implies 5C2 \implies $ "5 choose 2" $= \frac{5!}{3! \cdot 2!} = 10$

$\binom{n}{m} = \frac{n!}{(n-m)!m!}$

$\binom{n}{m} = 0$ when $m > n$. 

\underline{Pascals Identity}

\begin{align*}
    \binom{n}{m} = \binom{n-1}{m-1} + \binom{n-1}{m} \quad \text{ for all positive integers } n,m \text{ with } m < n.
\end{align*}

\underline{Binomial Theorem}

$(1+x)^4 = 1 + 4x + 6x^2 + 4x^3 + x^4$

BT1
\begin{align*}
    (1+x)^n = \sum_{m=0}^n\binom{n}{m}x^m  
\end{align*}

BT2
\begin{align*}
    (a+b)^n = \sum_{m=0}^n\binom{n}{m}a^{n-m}b^m  
\end{align*}

\underline{Practice}

Prove that for all integers $n \ge 0$, $\sum_{k=0}^n\binom{n}{k} = 2^n$

Let $x=1$ in BT1

$(1+1)^n = \sum_{k=0}^{n}\binom{n}{0}(1)^0$

What is the coefficient of $x^{18}$ in $(x^2-2x)^{12}$

By BT2
\begin{align*}
    (x^2-2x)^{12} &= \sum_{m=0}^{12}\binom{12}{m}(x^7)^{12-m}(-2x)^m \\
    &= \sum_{m=0}^{12}\binom{12}{m}(-2)^mx^{24-m}\\
    \text{Choosing } m&=6 \text{ gives the coefficient of } \binom{12}{6}(-2)^6 \\
    &= 59136
\end{align*}


\underline{Example}

Define $x_1=4, x_2 =68$ and $x_m = 2x_{m-1} + 15x_{m-2}$ for $m \ge 3$

Prove that $x_n = 2(-3)^n + 10 \cdot 5^{n-1}$ for all $n \in \mathbb{N}$.

Proof by Induction on $n$. 

Base Case: True when $n=1, n=2$

Inductive Step:

Let $k$ be an arbitrary natural number where $k \ge 2$. 

Let $P(n)$ be the open sentence.

Assume $P(1), P(2), P(3), \ldots, P(k)$ are all true. Then what happens to $k+1$?

Consider $n = k+1$

Then
\begin{align*}
    x_n = x_{k+1} &= 2x_k + 15x_{k-1} \\
    &= 2[2(-3)^k + 10 \cdot 5^{k-1}] + 15[2(-3)^{k-1}+10 \cdot 5^{k-2}] \\
    &= 4(-3)^4 + 30(-3)^{k-1} + 20 \cdot 5^{k-1} + 150 \cdot 5^{k-2} \\
    &= 4(-3)^k - 10(-3)^k + 4 \cdot 5^k + 6 \cdot 5^k \\
    &= -6(-3)^k + 10 \cdot 5^k \\
    &= 2(-3)^{k+1} + 10 \cdot 5^k
\end{align*}

Hence the proof is done by POSI. Difference between POMI and POSI is not base cases.

\subsection{Principal of Strong Induction}

Let $P(n)$ be a statement that depends on $n \in \mathbb{N}$. If 
\begin{enumerate}
    \item $P(1)$ is true, and 
    \item $\forall k \in \mathbb{N}, [(P(1) \wedge P(2) \wedge \ldots \wedge P(k)) \implies P(k+1)]$
\end{enumerate}

Example

Prove that $nm - 1$ breaks are needed to break an $n \times m$ chocolate bar into individual pieces.

Proof

$N = nm$. We will proceed by induction on $N$.

\underline{Base Case}

When $N = 1$, no breaks are needed. 

Since $N-1 = 0$, the statement is true for $N=1$.

\underline{Inductive Step}

Let $k \in \mathbb{N}$.

Suppose the statement is true when $N = 1, N = 2, N = 3, \ldots, N = k$. 

Consider $N = k+1$ and the first break. We are left with 2 smaller bars. Let $x$ and $y$ be the number of pieces in these smaller bars. 

Then $1 \le x, y \le k$. Also $x + y = N$. Breaking these two bars requires $(x-1)+(y-1) = N-2$ breaks by our IH. 

For the original bar, we require

$1 + N - 2 = N - 1$ breaks. By POSI this completes the proof.


\section{Sets}


\subsection{Introduction}

The number of elements in a set is cardinality. Denoted by $|S|$.

$S = \{1,2,4,6\}. |S| = 4$

$|\emptyset| = 0$ but $|\{\emptyset\}| = 1$

$\emptyset = \{\}$ empty set but $\ldots$

$\{\emptyset\}$ is not an empty set

\subsection{Set-Builder Notation}

Universal set $\mathcal{U}$ contains the objects we are concerned with (universe of discourse $\rightarrow$ universal set). 

Notation:

$\{x \in \mathcal{U}: P(x)\} = $ "The set of all $x$ in $\mathcal{U}$ such that $P(x)$ is true". 

$Q = \{x \in \mathbb{R}:x = \frac{a}{b}$ for some $a,b \in \mathbb{Z}, b \ne 0\}$

Set of positive factors of 12 $\{x \in \mathbb{N}: n \vert 12\}$

Set of even integers $\{x \in \mathbb{Z} : x = 2k, k \in \mathbb{Z}\}$

\underline{Set-Builder Notation Type 2}

$\{f(x): x \in \mathcal{U}\}$ "all objects in $\mathcal{U}$ of the form $f(x)$"

Even set of integers $\{2k: k \in \mathbb{Z}\}$

Perfect squares $\{x^2: x \in \mathbb{R}\}$

Multiples of 12 $\{12n: n \in \mathbb{Z}\}$

\underline{Set-Builder Notation Type 3}

$\{f(x): x \in \mathcal{U}, P(x)\}$ or $\{f(x): P(x), x \in \mathcal{U}\}$

Set consisting of all objects of the form $f(x)$ such that $x$ is an element of $\mathcal{U}$ and $P(x)$ is true.

$Q = \{\frac{a}{b}: a,b \in \mathbb{Z}, b \ne 0\}$

Integer powers of $2: \{2^k : k \in \mathbb{Z}, k \ge 0\}$

Perfect squares larger than $50: \{x^2: x^2 > 50, x \in \mathbb{Z}\}$

Multiples of $7: \{7x: x \in \mathbb{Z}\}$

Odd perfect squares: $\{x^2 : x^2 = 2k+1, k \in \mathbb{Z}\}$

\subsection{Set Operations}

Union of 2 sets $S$ \& $T, S \cup T$ is the set of all elements in either
\begin{align*}
    S \cup T = \{x: (x \in S) \vee (x \in T)\}
\end{align*}

e.g. $\{2k: k \in \mathbb{Z}\} \cup \{k \in \mathbb{Z}: 0 \le k \le 10\} = \{0,1,2,3,4,\ldots,10,12,14,\ldots\}$

Intersection of 2 sets $S$ \& $T, S \cap T$ is the set of elements in both
\begin{align*}
    S \cap T = \{x: (x \in S) \wedge (x \in T)\}
\end{align*}

Set Difference of 2 sets $S$ \& $T, S - T$ or $S \setminus T$ is the set of all elements in $S$ but not in $T$. 
\begin{align*}
    S \setminus T = \{x: (x \in S) \vee (x \notin T)\}
\end{align*}

The complement of a set $S, \overline{S}$ or $S^\complement$ is the set of elements in the universal set but not in $S$. 
\begin{align*}
    \overline{S} = \mathcal{U} - S = \{x \in \mathcal{U}: x \notin S\}
\end{align*}

(When $\mathcal{U} = \mathbb{Z}$) Let $S = \{x \in \mathbb{Z}: x \ge 0\}, \overline{S} = \{x \in \mathbb{Z}: x < 0\}$


\subsection{Subsets of a Set}

Two sets are disjoint when $S \cap T = \emptyset$.

Any set $S$ and its complement $\overline{S}$ are disjoint. 

Any set $S$ and $\emptyset$ are disjoint.

A set $S$ is a subset of set $T$ if every element of $S$ is an element of $T$. Denoted by: $S \subseteq T$. If $S$ is not a subset of $T$, that is denoted by $S \nsubseteq T$.

$\{2k: k \in \mathbb{Z}\} \subseteq \mathbb{Z}$\\
$\{2,5,6,8,10\} \nsubseteq \{2k: k \in \mathbb{Z}\}$\\
$\emptyset \subseteq S$ and $S \subseteq S$\\
$\mathbb{N} \subseteq \mathbb{Z}, \mathbb{Z} \subseteq \mathbb{Q}, \mathbb{Q} \subseteq \mathbb{R}$

A set $S$ is a proper set of $T$ if there is at least one element of $T$ that is not in $S$. ($S$ must be a subset). $S \subsetneq T$.

$A = \{2k: k \in \mathbb{Z}\}, B = \{2k+1: k \in \mathbb{Z}\}, C = A \cup B$

$A \subsetneq \mathbb{Z}, B \subsetneq \mathbb{Z}$

$C \subset \mathbb{Z}$ (not a proper subset since $C = \mathbb{Z}$)

$\{1,2,3\} \subset \{1,2,3,4\}$ and $\{1,2,3\} \subsetneq \{1,2,3,4\}$

All proper subsets are subsets

If $A \subset B \wedge B \subset A$, then $B = A$.

\subsection{Subsets, Set Equality, and Implications}

Given $S$ and $T$, prove $S \subseteq T$

Prove the implication $\forall x \in \mathcal{U}, (x \in S) \implies (x \in T)$

Example: Let $S = \{8m: m \in \mathbb{Z}\}$ and $T = \{2n: n \in \mathbb{Z}\}$. Show that $S \subseteq T$.

Proof: Let $x \in \mathbb{Z}$ and assume $x \in S$. Then $8m$ for $m \in \mathbb{Z}$. Then $x = 2(4m)$. $4m \in \mathbb{Z}$, set $n=4m$ and we can see $x = 2n$. Thus $x \in T$, $S \subseteq T$.

Let $A = \{n \in \mathbb{N}: 4 \vert (n-3)\}$ and $B=\{2k+1:k \in \mathbb{Z}\}.$ Prove $A \subseteq B$.

Let $x \in \mathbb{N}$ since $x \in A$. Then $4 \vert (x-3)$, such that $j \in \mathbb{Z}$
\begin{align*}
    4j &= x-3 \\
    x &= 4j + 3 \\
    &= 4j + 2 + 1 \\
    &= 2 \underbrace{(2j + 1)}_{\mathbb{Z}} + 1
\end{align*}

since $j \in \mathbb{Z}, 2j+1 \in \mathbb{Z}. k = 2j+1, x = 2k+1, x \in B$

Given $S$ \& $T$, prove $S = T$.

Prove $S \subseteq T$ and $T \subseteq S$.

Show $\forall x \in \mathcal{U}, (x \in S) \implies (x \in T) \wedge (x \in T) \implies (x \in S)$ or $(x \in S) \iff (x \in T)$

Let $S = \{1,-1,0\}$ and $T = \{x \in \mathbb{R}: x^3 = x\}$. Prove $S = T$

$\subseteq$ Let $x \in S$. Then $x=1,-1,0$. When $x=1, (1)^3 = 1 \ldots$. So $x \in S \implies x \in T$

$\supseteq$ Let $x \in T$. Then $x^3 = x$ or $x^3 - x = 0, x(x-1)(x+1) = 0$. $x$ must be $0,-1,$ or $1 \ldots x \in S$. $T \subseteq S$.

Since we have shown both $S \subseteq T$ and $S \supseteq T, S = T$.

Proving General Statements 

Prove $A \cap B \subseteq A$

Proof: Let $x \in A \cap B$, then $x \in A$ and $x \in B$. Thus $x \in A \cap B \implies x \in A$ so $A \cap B \subseteq A$.

Prove that $S = T$ if and only if $S \cap T = S \cup T$

($\rightarrow$) Assume $S = T$. Then $S \subseteq T$ and $T \subseteq S$. 

$\subseteq$ Let $x \in S \cap T$. Then $x \in S$ and $x \in T$ so $x \in S \cup T$

$\supseteq$ Let $x \in S \cup T$. Then $x \in S$ or $x \in T$. If $x \in S$, since $S \subseteq T$, then $x \in T$ and vice versa. 

Thus $x \in S \cup T, x \in S \cap T$.

($\leftarrow$) Assume $S \cap T = S \cup T$

$\subseteq$ Let $x \in S$. Then $x \in S \cup T \implies x \in S \cap T$ so $x \in T$.

$\supseteq$ Let $x \in T$. Then $x \in S \cup T \implies x \in S \cap T$ so $x \in S$.

We have shown it both ways so $S \subseteq T$ and $T \subseteq S, S = T$.


\section{The Greatest Common Divisor}

\underline{Bounds by Divisibility}

For all $a,b \in \mathbb{Z}$, if $b \vert a$ and $a \ne 0$, then $ b \le |a|$

Proof

Let $a,b \in \mathbb{Z}$

Assume $b \vert a$ and $a \ne 0$

Then there exists $q \in \mathbb{Z}$ such that $bq = a$.

From this we get $|bq| = |a|$

This tells us $|b||q| = |a|$

Since $a \ne 0$m then $q \ne 0$. 

Since $q \in \mathbb{Z}, q \ne 0$, then $|q| \ge 1$

Sub into equation to get $|b| \le |a|$

Since $b \le |b|$, so $b \le |a|$. 

\subsection{Division Algorithm}

For all $a \in \mathbb{Z}$ and for all $b \in \mathbb{N}$ there exists unique integers $q$ and $r$ such that 
\begin{align*}
    a = bq + r \quad \text{ where } 0 \le r < b
\end{align*}

\underline{Examples}
\begin{align*}
    a = 50, b &= 8 \quad 50 = 8 \cdot \underbrace{6}_{q} + \underbrace{2}_{r} \\
    a = 40, b &= 8 \quad 40 = 8 \cdot 5 + 0 \\
    a = -50, b &= 8 \quad -50 = 8 \cdot (-7) + 6
\end{align*}

\subsection{Greatest Common Divisor (GCD)}

Divisors of $84 : \pm 1, \pm 2, \pm 3, \pm 4, \pm 6, \pm 7, \pm 12, \pm 14, \pm 21, \pm 28, \pm 42, \pm 84$

Divisors of $60: \pm 1, \pm 2, \pm 3, \pm 4, \pm 5, \pm 6, \pm 10, \pm 12, \pm 15, \pm 20, \pm 30, \pm 60$

$gcd(84,60) = 12$

\underline{Formal Definition}

$Let a,b \in \mathbb{Z}$

When $a$ and $b$ are not both zero, we say an integer $d > 0$ is the \underline{greatest common divisor of $a$ and $b$}, and write $gcd(a,b)$ iff 
\begin{itemize}
    \item $d \vert a \wedge d \vert b$ 
    \item for all integers $c$, if $c \vert a$ and $c \vert b$ then $c \le d$
\end{itemize}

Otherwise, we say $gcd(0,0)=0$

\underline{Examples}
\begin{itemize}
    \item $gcd(84,60)=12$
    \item $gcd(-84,60)=12$
    \item $gcd(84,-60)=12$
    \item $gcd(-84,-60)=12$
    \item $gcd(84,0)=84$
    \item $gcd(-84,0)=84$
\end{itemize}

\underline{Fact}

For all $a,b \in \mathbb{Z}, gcd(3a + b, a) = gcd(a,b)$

\underline{Proof}

Let $a,b \in \mathbb{Z}$. Let $d = gcd(a,b)$

\underline{Case 1} $a=b=0$

In this case, by definition, $d=0$

Also $3a+b=0$ and $a=0$ in this case, thus $gcd(3a+b,a)=0$ as well.

\underline{Case 2} $a \ne 0$ or $b \ne 0$

Note that $3a + b \ne 0$ or $a \ne 0$ in this case as well. Since $d = gcd(a,b)$, we know $d > 0$ and $d \vert a$. We get $d \vert (3a + b)$ by DIC since we also know $d \vert b$. 

To complete the proof we let $c \in \mathbb{Z}$ and assume $c \vert (3a + b)$ and $c \vert a$

All we must show is $c \le d$. Using DIC again we get 

$c \vert [(3a+b)(1) + a(-3)]$

$c \vert b$

Hence by definition of $gcd(a,b) c \le d.$

\underline{GCD with Remainders (GCD w R)}

For all $a,b,q,r \in \mathbb{Z}$, if $a = bq + r$ then $gcd(a,b) = gcd(b,r)$

Example $86 = 20(7) - 54$

$gcd(86,20) = 2 \\
gcd(20,-54) = 2$

Alternative proof of our fact

Clearly $3a + b = 3a + b$

By GCD w R, $gcd(3a+b,a) = gcd(a,b)$

\underline{Euclidean Algorithm (EA)}

Process to compute $gcd(a,b)$ for $a,b \in \mathbb{N}$
\begin{align*}
    84 = 60(1) + 24 &\quad gcd(84,60) \\
    60 = 24(2) + \underline{12} &\quad = gcd(60,24) \\
    24 = 12(2) + 0 &\quad = gcd(24,12) \\
    &\quad gcd(12,0) = \underline{12}
\end{align*}

The last non-zero will be GCD since remainder is non-negative and $< b$.

Bigger example: Compute $gcd(1239,735)$
\begin{align*}
    1239 &= (735)(1) + 504 \\
    735 &= 504(1) + 231 \\
    504 &= 231(2) + 42 \\
    231 &= 42(4) + \underline{21} \\
    42 &= 21(2) + 0 \\
    &\implies gcd(1239,735) =21 
\end{align*}

\underline{Back Substitution}
\begin{align*}
    21 &= 231 + 42(-5) \\
    &= 231 + (-5)(504 + 231(-2)) \\
    &= 504(-5) + 231(11) \\
    &= 504(-5) + (11)(735-504) \\
    &= 735(11) + 504(-16) \\
    &= 735(11) + (-16)(1239-735) \\
    &= 1239(-16) + 735(27)
\end{align*}

\subsection{Certificate of Correctness and Bézout's Lemma}

For all $a,b,d \in \mathbb{Z}$ where $d \ge 0$. If $d \vert a$ and $d \vert b$ and there exists $s, t \in \mathbb{Z}$ such that $as + bt = d$ then $d = gcd(a,b)$.

Example

$d = 6, a = 30, b = 42$

$b \ge 0, 6 \vert 30, 6 \vert 42$

$6 = 30(3) + 42(-2)$

$\implies 6 = gcd(30,42)$

\underline{Bézout's Lemma}

For all integers $a,b \in \mathbb{Z}$, there exists $s,t \in \mathbb{Z}$ such that $as + bt = gcd(a,b)$

\underline{GCD w R}

$a = bq + r$ then $gcd(a,b) = gcd(b,r)$

\underline{GCD CT}

If $d \ge 0, d \vert a d \vert b$ and $s,t$ exists $as + bt = d$, then $d = gcd(a,b)$

\underline{BL}

If $d = gcd(a,b)$, there exists $x,y \in \mathbb{Z}$ such that $ax + by = d$

\underline{Example}

For all $n \in \mathbb{Z}, gcd(n,n+1) = 1$

\underline{Proof 1}

Since $n+1 = n(1) + 1$, GCD w R gives us

$gcd(n+1, n) = gcd(n,1)$. However 

$gcd(n,1) = 1$ because 1 is the only positive divisor of 1

\underline{Proof 2}

Since $(n+1)(1) + n(-1) = 1, 1 \ge 0$

$1 \vert n+1$ and $1 \vert n$, then $gcd(n+1, n) = 1$ by GCD CT. 

\underline{Proof 3}

Suppose $d \in \mathbb{Z}, d \vert (n+1)$ and $d \vert n$ then by DIC, $d \vert 1 [(n+1)(1)+n(-1) = 1]$ Thus $1$ is the only divisor, that is GCD = 1. 

\underline{Example}

Let $a,b,x,y \in \mathbb{Z}$, where $gcd(a,b) \ne 0$. If $ax + by = gcd(a,b)$ then $gcd(x,y) = 1$.

\underline{Proof}

Let $a,b,x,y \in \mathbb{Z}$. Assume $gcd(a,b) \ne 0$ and $ax + by = gcd(a,b)$

Division gives 

$(\frac{a}{gcd(a,b)})x + (\frac{b}{gcd(a,b)})y = 1$ since $gcd(a,b) \ne 0$

Since $\frac{a}{gcd(a,b)}, \frac{b}{gcd(a,b)} \in \mathbb{Z}$

Moreover $1 \ge 0, 1 \vert x and 1 \vert y$

Thus by GCD LT, $gcd(x,y)=1$

\underline{Example}

For all $a,b,c \in \mathbb{Z}$

If $gcd(a,c)=1$ then $gcd(ab,c) = gcd(b,c)$

\underline{Proof}

Let $a,b,c \in \mathbb{Z}$. Assume $gcd(a,c)=1$. Let $d = gcd(b,c)$

By BL, there are integers $x,y,s,t$ such that 

$ax + cy = 1$ and $bs + ct = d$

multiply to get 

$(ax + cy)(bs + ct) = d$

Thus 

$ab(xs) + c(axt + ybs + yct) = d$

Since $xs, axt + ybs + yct$ are integers, $d \ge 0$ (by definition), $d \vert c$ (by definition), $d \vert ab$, we get $d = gcd(ab,c)$ by GCD CT.

\subsection{Extended Euclidian Algorithm}

Solve $56x + 35y = gcd(56,35)$ for $x,y \in \mathbb{Z}$


\begin{table}[!h]
    \centering
    \begin{tabular}{c|c|c|c} 
         $x$&$y$& $r$ & $q$\\ \hline 
         1& 0 & 56 & $\leftarrow 56 = 35(1) + 21$\\ 
         0& 1 & 35 &$\vdots$\\
         1& -1 & 21 & 1\\ 
         -1& 2 & 14 & 1\\ 
         \underline{2}& \underline{-3} & \underline{7} & 1\\ 
         &  & 0 &2 \\
    \end{tabular}
\end{table}

Thus $gcd(36,35) = 7, x = 2, y = -3$

EEA with 408 and 170

\begin{table}[!h]
    \centering
    \begin{tabular}{c|c|c|c} 
         $x$&$y$& $r$ & $q$\\ \hline 
         1& 0 & 408 & $\leftarrow 408 = 170(2) + 68$\\ 
         0& 1 & 170 &$\vdots$\\
         2& -2 & 68 & 2\\ 
         -2& 5 & 34 & 2\\ 
         &  & 0 & \\
    \end{tabular}
\end{table}

Solve $-170x + 408y = d$ for $x,y \in \mathbb{Z}$ and $d = gcd(-170, 408)$

Order is irrelevant for gcd. 

From before $d = 34$ and $x = -5, y = -2$


\subsection{Further Properties of the Greatest Common Divisor}

\underline{Proof of CDD GCD (Common Divisor Divides)}

Let $a,b,c \in \mathbb{Z}$. Assume $c \vert a$ and $c \vert b$. 

By BL, $ax + by = gcd(a,b)$ for some $x,y \in \mathbb{Z}$

By DIC, $c \vert ax+by$. That is $c \vert gcd(a,b)$.

\underline{Definition}

Let $a,b \in \mathbb{Z}$

When $gcd(a,b) = 1$, we say $a$ and $b$ are coprime.

\underline{Coprimeness Characterization Theorem}

$a$ and $b$ are coprime iff there exists integers $s$ and $t$ with $as + bt = 1$.

Sketch of CCT Proof

$\implies$ BL

$\impliedby$ GCD CT

Exercise

Let $a,b,c \in \mathbb{Z}$.

If $gcd(a,b,c) = 1$, then $gcd(a,c) = 1$ and $gcd(b,c)=1$

a) Prove or disprove

Let $a,b,c \in \mathbb{Z}$. Assume $gcd(a,b)=1$.

By CCT, $(ab)s + ct = 1$ for some $s,t \in \mathbb{Z}$

Since $bs, t \in \mathbb{Z}, gcd(a,c) = 1$ by CCT

Since $as, t \in \mathbb{Z}, gcd(b,c) = 1$ by CCT

b) Prove or disprove the converse

If $gcd(a,c)$ and $gcd(b,c)$, then $gcd(ab,c)=1$

Let $a,b,c \in \mathbb{Z}$. Assume $gcd(a,c) = gcd(b,c)=1$.

By CCT, $as + ct = 1$ and $bx + cy = 1$ for some $s,t,x,y \in \mathbb{Z}$

Multiply to yield

$(as + ct)(bx + cy) = 1$

After expanding and rearranging, CCT gives us $gcd(a,b) = 1$ because $sx, asy + tbx + txy \in \mathbb{Z}$.

\underline{Division by GCD (DB GCD)}

If $gcd(a,b) = d \ne 0$ then $gcd(\frac{a}{d}, \frac{b}{d}) = 1$.

Let $a,b \in \mathbb{Z}$ such that $gcd(a,b) = d \ne 0$.

By BL, $ax + by = d$ for some $x,y \in \mathbb{Z}$.

Divide by d

$\frac{a}{d}x + \frac{b}{d}y = 1,$ since $d \ne 0$

Note $d \vert a$ and $b \vert d$ by definition of $d$, so $\frac{a}{d}, \frac{b}{d}$ are $\mathbb{Z}$. Thus $(\frac{a}{d}, \frac{b}{d}) =1 $ by CCT

\underline{Proof of Coprimeness and Divisibility (CAD)}

If $a,b$ and $c$ are integers and $c \vert ab$ and $gcd(a,c)=1$, then $c \vert b$.

\underline{Proof}

Let $a,b,c \in \mathbb{Z}$. 

Assume $c \vert ab$ and $gcd(a,c) = 1$

$ax + cy = 1$ by CCT for some $x,y \in \mathbb{Z}$

Multiply both sides by $b$ to get 

$abx + cby = b$

We know $c \vert c$ and we assumed $c \vert ab$ so by DIC, $c \vert [(ab) x + (c)by]$ (because $x, by \in \mathbb{Z}$).

That is, $c \vert b$

\underline{Note}

$\forall a,b,c \in \mathbb{Z}, (c \vert ab) \implies (c \vert a \vee c \vert b)$ is \underline{false}.


\subsection{Prime Numbers}

\underline{Prime Factorization}

Every integer greater than 1, can be written as the product of primes.

\underline{Proof}

Proceed by Strong Induction (can't use POMI) to prove that an integer $n > 1$ can always be written as a product of primes.

\underline{Base Case}

When $n=2$, $n$ by itself is a product of primes since 2 is prime.

\underline{Inductive Step}

Let $k$ be an arbitrary integer greater than 2. 

Assume $i$ can be written as the product of primes for all integers $i$ such that $2 \le i \le k$.

We will consider cases for $n = k+1$

When $k+1$ is prime, there is nothing to prove.

Otherwise, $k+1$ is composite. 

That is $k+1 = ab$ for some $a,b \in \mathbb{Z}$ satisfying $1 < a, b < k+1$

By our inductive hypothesis, $a$ and $b$ can each be written as the product of primes. Multiplying these products gives a product of primes equal to $k+1$. Hence the statement is true by POSI. 

\underline{Euclid's Theorem}

There are infinitely many primes.

\underline{Proof}

By way of contradiction, assume there are a finite number of primes. We will name them $p_1, p_2, \ldots, p_k$ for some $k \in \mathbb{N}$.

Consider $N = (p_1 \cdot p_2 \hdots p_k) + 1$

By PF, $p_i \vert N$ for some $i \in \{1,2,\ldots,k\}$

However, also $p_i \vert (p_1 \cdot p_2 \hdots p_k)$ by definition.

By DIC, we get $p_i \vert N - (p_1 \cdot p_2 \hdots p_k)$

That is, $p \vert 1$. This is a contradiction because 1 is the only positive divisor of 1.

\underline{Euclid's Lemma}

For all $a,b \in \mathbb{Z}$ and primes $p$, if $p \vert ab$, then $p \vert a$ or $p \vert b$. 

\underline{Proof}

Let $a,b \in \mathbb{Z}$. Let $p$ be prime. 

Assume $p \vert ab$ and $p \nmid a$ (elimination). 

Since the only positive divisors of $p$ are 1 and $p$, and $p \nmid a, gcd(a,p) =1$.

Thus $p \vert b$ by CAD. 


\subsection{Unique Factorization Theorem}

Every natural number $> 1$ can be written as a product of prime factors uniquely, apart from order.

\underline{Example}

Let $p$ be prime. Prove that $13p + 1$ is a perfect square iff $p = 11$. 

If $p = 11, 13(11) + 1 = 144 = 12^2 \checkmark$

Other direction:

$13p + 1 = k^2$

$13p = (k+1)(k-1)$

UFT $\rightarrow 13 = k+1$ or $13 = k-1$

$k = 12 \checkmark$ or $k = 14$ (wrong). 

\subsection{Prime Factorization and the Greatest Common Divisor}

If $a = p_1^{\alpha_1} \ldots p_k^{\alpha_k}$ and $b=p_1^{\beta_1}\ldots b=p_k^{\beta_k}$ where $p_1,p_2,\ldots,p_k$ are primes and all exponents are non-negative.
\begin{align*}
 gcd(a,b) = p_1^{\gamma_1} p_2^{\gamma_2} \ldots \text{ where } \gamma_i = min\{\alpha_i, \beta_1\} \text{ for } i \ldots k   
\end{align*}

Examples
\begin{align*}
    &gcd(13^2 \cdot 7^{100}, 16^3 \cdot 7^{44}) \\
    &gcd(7^{100}11^013^2, 7^{44}11^313^0) \\
    &= 7^{44} \cdot 11^0 \cdot 13^0 \\
    &= 7^{44}
\end{align*}
And
\begin{align*}
    &gcd(20000,30000) \\
    &gcd(2^55^4,2^43^15^4) \\
    &= 2^4 \cdot 5^4 \cdot 3^0 \\
    &= 2^4 \cdot 5^4 \\
    &= 10000
\end{align*}


\section{Linear Diophantine Equations}

\subsection{The Existence of Solutions in Two Variables}

Given $a,b,c \in \mathbb{Z}$, find $x,y \in \mathbb{Z}$ such that $ax + by = c$

\begin{itemize}
    \item Is there a solution? LDET 1
    \item If so, how can we find one? EEA
    \item And can we find all solutions? LDET 2
\end{itemize}

Examples of 
\begin{enumerate}
    \item $143x + 253y = 11$
    \item $143x + 253y = 155$
    \item $143x + 253y = 154$
\end{enumerate}

1) Use EEA

\begin{table}[!h]
    \centering
    \begin{tabular}{c|c|c|c} 
         $y$&$x$& $r$ & $q$\\ \hline 
         1& 0 & 253 & \\ 
         0& 1 & 143 &\\
         1& -1 & 110 & 1\\ 
         -1& 2 & 33 & 1\\ 
         4& -7 & 11 & 3\\ 
         -13 & 23 & 3 &0 \\
    \end{tabular}
\end{table}

Thus $\{(-7+23n, 4-13n):n\in \mathbb{Z}\}$

Thus $143(-7) + 253(4) = 11$, $(-7, 4)$ is a solution.

2) There is no solution because $x,y \in \mathbb{Z}$, $11 \vert (143x + 253y)$ but $11 \vert 155$ (not a multiple of $11$).

3) Multiply equation in 1) by $\frac{154}{11} = 14$ to get: 

$143(-98) + 153(56) = 154$

Other solutions to 1)?

Rewrite as $y = \frac{-13}{23}x + \frac{1}{23}$

LDET 1

Let $a,b \in \mathbb{Z}$ (both not zero) and let $d = gcd(a,b)$ the LDE $ax + by = c$ has a solution if and only if $d \vert c$. 

First, suppose there exists $x,y \in \mathbb{Z}$ such that $ax + by = c$.

We know $d \vert a$ and $d \vert b$ (by definition of gcd), so $d \vert c$ by DIC. 

Next we suppose $d \vert c$ to prove the other direction. 

By BL there exists $s,t \in \mathbb{Z}$ such that $as + bt = d$. 

Now we also know $dk = c$ for some integer $k$. Multiplying by $k$ gives

$a(bk) + b(tk) = dk = c$

Since $sk$ and $tk, \in \mathbb{Z}$, the proof is complete.

\subsection{Finding All Solutions in Two Variables}

LDET 2

Let $gcd(a,b) = d$ where $a \ne 0, b \ne 0$. 

If $(x,y) = (x_0, y_0)$ is one solution to the LDE $ax + by = c$, then the complete solution is 
\begin{align*}
    \{(x_0 + \frac{b}{d}n, y_0 - \frac{a}{d}n) : n \in \mathbb{Z}\}
\end{align*}

LDET 2 Example

We found that $(x,y) = (-7,4)$ was a particular solution to $143x + 253y = 11$. 

LDET 2 tells us the complete solution is 

$\{(-7 + \frac{253}{11}n, 4 - \frac{143}{11}n): n \in \mathbb{Z}\}$

$= \{(-7 + 23n, 4-13n): n \in \mathbb{Z}\}$

Examples of some solutions are:

$n = 0 \quad (-7, 4)$ \\
$n = 1 \quad (16, -9)$ \\
$n = -1 \quad (-30, 17)$

Exercise 

Solve the following LDEs: 

1) $28x + 35y = 60$

$7 \nmid 60$, no solutions. 

2) $343x + 259y = 658$
\begin{align*}
    343(-3) + 259(4) &= 7\\
    343(-3 \cdot 94) + 259(4 \cdot 94) &= 7 \cdot 94 \\
    343(282) + 259(376) &= 658 \\
    \{(-3 + 37n, 4+49n): n \in \mathbb{Z}\}
\end{align*}

\underline{LDET 2 Proof}

Let $a,b,c \in \mathbb{Z}$ where $d = gcd(a,b), a \ne 0$ and $b \ne 0$. 

Assume $ax_0 + by_0 = c$ for some $x_0, y_0 \in \mathbb{Z}$. 

Define $S = \{(x,y): ax + by = c$ and $x,y \in \mathbb{Z}\}$ and $T = \{(x_0 + \frac{b}{d}n, y_0-\frac{a}{d}n): n \in \mathbb{Z}\}$

Must show how $S = T (S \subseteq T, T \subseteq S)$

We begin by showing $T \subseteq S$. 

Let $n \in \mathbb{Z}$. 

We must show $(x_0 + \frac{b}{d}n, y_0-\frac{a}{d}n) \in S$.

To do this we substitute into $ax + by$ to get 
\begin{align*}
    a(x_0 + \frac{b}{d}n) + b(y_0-\frac{a}{d}n) = ax_0 + by_0 = c
\end{align*}

Indeed $T \subseteq S$. 

Now we must show $S \subseteq T$. 

Let $(x,y) \in S$. Then $ax + by = c$. 

We also know $ax_0 + by_0 = c$. 

Equating gives $ax-ax_0 = -by + by_0$

Thus $a(x-x_0) = -b(y-y_0) (\star)$

Since $d \ne 0$, we divide and get the following.

$\frac{a}{d}(x-x_0) = \frac{-b}{d}(y-y_0)$

This tells us $\frac{b}{d} \vert \frac{a}{d}(x-x_0)$

By DBGCD, $gcd(\frac{a}{d},\frac{b}{d})=1$. By CAD, we know $\frac{b}{d} \vert (x - x_0)$. Thus $\frac{b}{d} \vert (x-x_o)$. Thus $\frac{b}{d}n = x-x_0$ for some $n \in \mathbb{Z}$. 

That is $x=x_0 + \frac{b}{d}n$. Substitution into ($\star$) yields $y= y_0 - \frac{a}{d}n$. Thus $(x,y) \in T$. 

\underline{Exercise}

Find all $x,y \in \mathbb{Z}$ satisfying

$15x-24y = 9 \quad 0 \le x,y \le 20$. 

We will solve the LDE first.

Note that it is equivalent to 

$5x-8y=3$

By inspection, a solution (7,4). 

So by LDET 2, the complete solution is 

$x = -1 -8n$ and $y=-1-5n$ where $n \in \mathbb{Z}$

We also need 
\begin{align*}
    -1 -8n \ge 0 &\implies n \le -1 \\
    -1 -8n \le 20 &\implies n \ge -2 \\
    -1 -5n \ge 0 &\implies n \le -1 \\
    -1 -5n \le 20 &\implies n \ge -4
\end{align*}

Thus $n = -1$ or $n = -2$. 

Thus the final answer is $\{(7,4),(15,9)\}$


\section{Congruence and Modular Arithmetic}

\subsection{Congruence}

$-1$ is congruent to 7 modulo 8. 

\underline{Definition}

Let $a,b \in \mathbb{Z}$. Let $m \in \mathbb{N}$.

We say $a$ is congruent to $b$ module $m$ when

$m \vert (a-b)$.

We write 
\begin{align*}
    a \equiv b \pmod{m}
\end{align*}
Otherwise we write $a \not\equiv b \pmod{m}$. 

\underline{Examples}

$-1 \equiv 7 \pmod{8}$\\
$-1 \equiv -1 \pmod{8}$\\
$-1 \equiv 15 \pmod{8}$\\
$15 \equiv -1 \pmod{8}$\\
$15 \equiv 7 \pmod{8}$

Let $a,b \in \mathbb{Z}$. Let $m \in \mathbb{N}$
\begin{align*}
    &a \equiv b \pmod{m}\\
    &\iff m \vert (a - b)\\
    &\iff \exists k \in \mathbb{Z}, mk = a-b\\
    &\iff \exists k \in \mathbb{Z}, a = mk + b
\end{align*}

\subsection{Elementary Properties of Congruence}

Let $a,b,c \in \mathbb{Z}$. Let $m \in \mathbb{N}$.

Reflexive: $a \equiv a \pmod{m}$\\
Symmetric: $a \equiv b \pmod{m} \implies b \equiv a \pmod{m}$\\
Transitivity: $a \equiv b \pmod{m}$ and $b \equiv c \pmod{m} \implies a \equiv c \pmod{m}$

Proof of Reflexivity:

Since $a - a = 0$ and $m0 = 0$, we have $m \vert (a-a)$. That is $a \equiv a \pmod{m}$.

Proof of Symmetric:

Assume $a \equiv b \pmod{m}$

This means $mk = a-b$ for some $k \in \mathbb{Z}$. $m(-k) = b-a$. 

Since $-k \in \mathbb{Z}, m \vert (b-a)$. 

That is $b \equiv a \pmod{m}$. 

Proof of Transitivity:

Assume $a \equiv b \pmod{m}$ and $b \equiv c \pmod{m}$. 

$m \vert (a-b), m \vert (b-c)$. 

By DIC, $m \vert (a-c)$. 

That is $a \equiv c \pmod{m}$.

Proposition 2

If $a_1 \equiv b_1 \pmod{m}$ and $a_2 \equiv b_2 \pmod{m}$, then
\begin{enumerate}
    \item $a_1 + a_2 \equiv b_1 + b_2 \pmod{m}$
    \item $a_1 - a_2 \equiv b_1 - b_2 \pmod{m}$
    \item $a_1a_2 \equiv b_1b_2 \pmod{m}$
\end{enumerate}

Proof of 1.

$mk = a_1 - b_1 \quad m\ell = a_2 - b_2$
\begin{align*}
    a_1 + a_2 &= (mk + b_1) + (m\ell + b_2) \\
    &= m \underbrace{(k+\ell}_{\in \mathbb{Z}}) + b_1 + b_2
\end{align*}

Proof of 3. 
\begin{align*}
    a_1a_2 &= (mk + b_1) + (m\ell + b_2) \\
    &= (b_1 \cdot b_2) + m\underbrace{(\ldots)}_{\text{some integer}}
\end{align*}

\underline{CAM} (Generalization of Proposition 2)

For all positive integers $n$, for all integers $a_1 \ldots a_n$ and $b_1 \ldots b_n$, if $a_i \equiv b_i \pmod{m}$ for all $1 \le i \le n$ then 
\begin{align*}
    a_1 + a_2 + \ldots + a_n &\equiv b_1 + b_2 \ldots + b_n \pmod{m} \\
    a_1 a_2 \ldots a_n &\equiv b_1b_2 \ldots b_n \pmod{m}
\end{align*}

\underline{Congruence of Power}

For all positive integers $n$ and $a,b \in \mathbb{Z}$. 

$a \equiv b \pmod{m} \implies a^n \equiv b^n \pmod{m}$.

Question: Does 7 divide $5^9 + 62^{2000} - 14$

Is $5^9 + 62^{2000} - 14 \equiv 0 \pmod{7}$?

We will "reduce modulo 7"
\begin{align*}
    -14 \equiv 0 \pmod{7} \\
    \implies 5^9 + 62^{2000}-14 &\equiv 5^9 + 62^{2000} + 0 \pmod{7} \\
    &\equiv 5^9 + (-1)^{2000} \pmod{7} \leftarrow \text{ by CP} \\
    &\equiv 5^9 + 1 \pmod{7} \\
    &\equiv (-2)^9 + 1 \pmod{7} \\
    &\equiv (-2)^3(-2)^3(-2)^3 + 1 \pmod{7}\\
    &\equiv (-1)(-1)(-1) + 1 \\
    &\equiv 0 \pmod{7}
\end{align*}

\underline{Congruence and Division}

Examples

Let $a,b,c \in \mathbb{Z}$. Let $m \in \mathbb{N}$. 

If $ac \equiv bc \pmod{m}$ and $gcd(c,m)=1$ then $a \equiv b \pmod{m}$.

\underline{Examples}

1) \\
$3 \equiv 24 \pmod{7}$\\
$1 \equiv 8 \pmod{7}$

2)\\
$3 \equiv 27 \pmod{6}$\\
$1 \not\equiv 9 \pmod{6}$

\underline{Exercise}

Does 72 divide $4(-66)^{2022} + 800$

By CAR, CAM and CP:
\begin{align*}
    4(-66)^{2022} + 800 &\equiv 2(-6)^2 2(-11)^2 (-66)^{2020} + 800\\
    &\equiv 0 + 8 \pmod{72} \\
    &\equiv 8 \pmod{72}
\end{align*}

But $8 \not\equiv 0 \pmod{72}$

Thus by CER, our number is not congruent to 0 modulo 72. Thus it does not. 

\underline{Proof of CD}

Let $a,b,c \in \mathbb{Z}$. Let $m \in \mathbb{N}$.

Assume $ac \equiv bc \pmod{m}$ and $gcd(c,m)=1$.

Then $m \vert (ac - bc)$ or equivalently $m \vert c(a-b)$. 

By CAD, $m \vert (a-b)$. That is, $a \equiv b \pmod{m}$.

\subsection{Congruence and Remainders}

\underline{Congruent iff Same Remainder (CISR) and Congruent to Remainder (CTR) Examples}

1) What is the remainder when $x = 77^{100}(999)-6^{83}$ is divided by 4. 

We will find $r$ such that $0 \le r < 4$ and $x \equiv r \pmod{4}$. By CTR, this will be our answer. 

By CER, CAM, and CP:
\begin{align*}
    x &\equiv 1^{100}(-1) - 36 \cdot 6^{81} \pmod{4} \\
    x &\equiv -1 \pmod{4} \\
    &\equiv 3 \pmod{4}
\end{align*}

The answer is 3. 

2) What is the last digit (units) of $x = 5^{32}3^{10}+9^{22}$

The answer will be $r$ such that $x \equiv r \pmod{10}$ and $0 \le r < 10$ (By (TR)). 

By CER, CAM, and CP
\begin{align*}
    x &\equiv (5^2)^{16} (3^2)^5 + (-1)^{22} \pmod{10} \\
    &\equiv (5^2)^8(-1)^5 + 1 \pmod{10} \\
    &\equiv (5^2)^4(-1)+1 \\
    &\equiv -5 + 1 \pmod{10} \\
    &\equiv 6 \pmod{10}
\end{align*}
The answer is 6.

\underline{Proof of CISR}

Let $a,b \in \mathbb{Z}$. Let $m \in \mathbb{N}$. 

By DA,

$a = mq_a + r_a, \quad 0 \le r_a < m$\\
$b = mq_b + r_b, \quad 0 \le r_b < m$

Then, $a-b = m(q_a-q_b)+(r_a-r_b)$ 

where $-m < r_a - r_b < m$

Now we assume $r_a = r_b$.

Thus, $m \vert (a-b)$ by our equation for $a-b$. That is $a \equiv b \pmod{m}$. 

Next, we assume $a \equiv b \pmod{m}$. 

Then $mk = a-b$ for some $k \in \mathbb{Z}$. 

Substituting and rearranging gives,

$m(k-q_a + q_b) = r_a - r_b$

So $m \vert (r_a - r_b)$ since $k - q_a + q_b \in \mathbb{Z}$. Thus $r_a - r_b = 0$ by our inequality for $r_a - r_b$. We get $r_a = r_b$, completing the proof. 

\underline{CTR}

For all $a,b$ with $0 \le b < m$, $a \equiv b \pmod{m}$ iff $a$ has remainder $b$ when divided by $m$. 

$m \vert (a -b)$ if $a = mr + b$

\underline{Divisibility Tests}

Let $n \ge 0$ be an integer. Then we can write.

$n = d_k10^k + d_{k-1}10^{k-1} + \ldots d_110 + d_0$ for digits $d_k, d_{k-1}, \ldots, d_1, d_0$

What about 3?

Since $10 \equiv 1 \pmod{3}$. $n \equiv d_k + d_{k-1} + \ldots + d_1 + d_0 \pmod{3}$

Thus, by CER

$n \equiv 0 \pmod{3}$ iff $d_k + d_{k-1} + \ldots + d_1 + d_0 \equiv 0 \pmod{3}$.

9?

$10 \equiv 1 \pmod{9}$ so we can deduce that $n$ is divisibly by 9 iff the sum of its digits are divisible by $n$. 

e.g. 4456217395

$4+4+5+6+2+1+7+3+9+5=46$. 46 is not divisible by 9, the number is not divisible by 9. 

11?

$8217993$

$8-2+1-7+9-9+3 = 3$

$10 \equiv -1 \pmod{11}$

\subsection{Linear Congruences}

Let $m \in \mathbb{N}$. 

Let $a,c \in \mathbb{Z}$ where $a \ne 0$. 

Find all $x \in \mathbb{Z}$ such that 
\begin{align*}
    ax \equiv c \pmod{m}
\end{align*}

\begin{itemize}
    \item Is there a solution?
    \item If so can we find one?
    \item If so can we find them all?
\end{itemize}

\underline{Example}

Solve $4x \equiv 5 \pmod{8}$
\begin{align*}
    &\iff 8 \vert (4x-5) \\
    &\iff 8k = 4x - 5 \text{ for some } k \in \mathbb{Z} \\
    &\iff 4x - 8k = 5 \text{ for some } k \in \mathbb{Z} \\
    &\iff 4x = 8y = 5 \text { for some } y \in \mathbb{Z}
\end{align*}

Linear Diophantine $\implies gcd(4,8) = 4$. $4 \nmid 5$. $\therefore$ no solution, $\therefore$ no $x-$values.

$5x \equiv 3 \pmod{7}$

Rewrite
\begin{align*}
    5x + 7y = 3 \implies &x \in \{2 + 7n: n \in \mathbb{Z}\}\\
    &gcd(5,7) = 1 \quad 1 \vert 3 \checkmark \\
\end{align*}

Answer in congruence is $x \equiv 2 \pmod{7}$.

By CTR, every integer is congruent to $\{0,1,2,3,4,5,6\}$.

Try all of them and see which one works. 

By CER, CAM, if $x_0$ is a solution, $x \equiv x_0 \pmod{7}$ are solutions.

GCD is the number of solutions in the set $\{0,1,2,\ldots\}$
\begin{align*}
    2x &\equiv 4 \pmod{6}\\
    2(0) &\not\equiv 4 \pmod{6}\\
    &\vdots\\
    2(2) &\equiv 4 \pmod{6}\\
    &\vdots\\
    2(5) &\equiv 4 \pmod{6}
\end{align*}

Complete solution is $ x \equiv 2,5 \pmod{6}$. 

Using LDE's we get $\{2+3n:n \in \mathbb{Z}\}$.

Complete solution is $x \equiv 2 \pmod{3}$. 

$x \equiv 2,5 \pmod{6}$ and $x \equiv 2 \pmod{4}$ represent the exact same set of integers. 


\underline{Linear Congruence Theorem (LCT)}

Complete solution $\{x \in \mathbb{Z} : x \equiv x_0 \pmod{\frac{m}{d}}\}$ equivalently,
\begin{align*}
 \{x \in \mathbb{Z}: x \equiv x_0, \underbrace{x_0 + \frac{m}{d}, x_0 + 2\frac{m}{d}, \ldots, x_0+(d-1)\frac{m}{d}\}}_{d \text{ number of solutions}} 
\end{align*}

Informally, LCT tells us there 
\begin{itemize}
    \item is one solution modulo $\frac{m}{d}$ or 
    \item $d$ solutions modulo $m$
\end{itemize}

Solve $9x \equiv 6 \pmod{15}$

$d = gcd(9,15) = 3, 3 \vert 6 \checkmark$

$\{x \in \mathbb{Z}: x \equiv 4 \pmod{5}\}$

\subsection{Congruence Classes and Modular Arithmetic}

\underline{Definition}

Let $m \in \mathbb{N}$. Let $a \in \mathbb{Z}$. 

The congruence class of $a$ modulo $m$ is 
\begin{align*}
    [a] = \{x \in \mathbb{Z}: x \equiv a \pmod{m}\}
\end{align*}

\underline{Example}

Let $m=5$

The congruence class of 3 modulo 5 is: 
\begin{align*}
    [3] &= \{x \in \mathbb{Z}: x \equiv 3 \pmod{5}\} \\
    &= \{\ldots, -12, -7, -2, 3, 8, 13, 18, 23,\ldots\} \text{ infinite set of integers}
\end{align*}

\begin{itemize}
    \item $[3]$ is an infinite set
    \item $[3] = [23] = [-7]$ (both subsets of each other)
    \item $[3]$ is our most common representative from this set because $0 \le 3 \le 5$
\end{itemize}

\underline{Operations}

Let $m \in \mathbb{N}$. Let $a,b \in \mathbb{Z}$. We define 
\begin{align*}
    [a] + [b] = [a + b]\\
    [a][b] = [ab]
\end{align*}

\underline{Examples} $(m=5)$


\begin{table}[!h]
    \centering
    \begin{tabular}{c|ccccc}  
        $+$ & [0] & [1] & [2] & [3] & [4]\\ \hline 
        $[0]$ & [0] & [1] & [2] & [3] & [4]\\  
        $[1]$ & [1] & [2] & [3] & [4] & [0]\\
        $[2]$ & [2] & [3] & [4] & [0] & [1]\\
        $[3]$ & [3] & [4] & [0] & [1] & [2]\\
        $[4]$ & [4] & [0] & [1] & [2] & [3]\\
    \end{tabular}
\end{table}

\begin{table}[!h]
    \centering
    \begin{tabular}{c|ccccc}  
        $\times$ & [0] & [1] & [2] & [3] & [4]\\ \hline 
        $[0]$ & [0] & [0] & [0] & [0] & [0]\\  
        $[1]$ & [0] & [1] & [2] & [3] & [4]\\
        $[2]$ & [0] & [2] & [4] & [1] & [3]\\
        $[3]$ & [0] & [3] & [1] & [4] & [2]\\
        $[4]$ & [0] & [4] & [3] & [2] & [1]\\
    \end{tabular}
\end{table}

\underline{Note}

Addition is well-defined

$[8] + [31] = [39] = [4]$

$[-7]+[16] = [9] = [4]$

Multiplication is as well. 

\underline{Definition}

Let $m \in \mathbb{N}$. The integers modulo $m$ are
\begin{align*}
    \mathbb{Z}_m &= \{[0],[1],[2],\ldots,[m-1]\} \quad |\mathbb{Z}_m| = m \text{ finite}\\
    &= \{[x]:x \in \mathbb{Z}\}
\end{align*}
\begin{align*}
    &a \equiv b \pmod{m} \iff m \vert (a-b) \iff \exists k \in \mathbb{Z}, a-b = km \iff \exists k \in \mathbb{Z}, a = km + b\\
    &\iff a \text{ and } b \text{ have the same remainder when divided by } m \iff [a] = [b] \text{ in } \mathbb{Z}_m
\end{align*}

Let $[a] = \mathbb{Z}_n$ where $m \in \mathbb{N}$. 

$[0]$ is the additive identity $[a] + [0] = [a]$\\
$[1]$ is the multiplication identity $[a][1] = [a]$\\
$[-a]$ is the additive inverse of $[a] \implies [a]+[-a]=[0]$

Multiplicative inverse of $[a]$ (if exists) is an elem $[b]$ such that $[a][b] = [b][a] = [1]$ and we write $[b]=[a]^{-1}$.

\underline{Examples}

In $\mathbb{Z}_{12}$ does $[5]^{-1}$ exist? Does $[6]^{-1}$ exist?

$[5][x]=[1]$

 $[x] = [5]$ is a solution, so $[5]^{-1} = [5]$

 $[6][x] = [1]$. Only 12 combinations, none where $6x \equiv 1 \pmod{12}$. 

 \underline{Modular Arithmetic Solution}

 Let $gcd(a,m) = d \ne 0$. 

 The equation $[a][x]=[c]$ in $\mathbb{Z}_n$ has a solution iff $d \vert c$. 

 If $[x] = [x_0]$ is one solution, then there are $d$ solutions given by, 
 \begin{align*}
     \{[x_0],[x_0 + \frac{m}{d}], [x_0 + 2\frac{m}{d}],\ldots,[x_0 + (d-1)\frac{m}{d}]\}
 \end{align*}

Review 

$\mathbb{Z}_{10}, [3]=[13]=[23]=[-17]$

In $\mathbb{Z}_{10}$, solve

1) $[12][x] + [3] = [8]$\\
$[2][x]=[5]$ has no solution. 

2) $[15][x]+[7] = [12]$\\
$[5][x]=[5]$. $gcd(5,10) = 5 \implies$ 5 solutions. $\frac{10}{5}=2$, spanned by 2 $\downarrow$

$[1],[3],[5],[7],[9]$

3) $[9][x] + [1] = [8]$

$[9][x]=[7]$. $gcd(9,10) = 1 \implies 1$ solution.

$x=3, 3\cdot 9 = 27, 27-7 = \underline{20}$.


\underline{Inverses in $\mathbb{Z}_m$ (INV $\mathbb{Z}_m$)}

Let $a \in \mathbb{Z}$ with $0 \le a \le m-1$. $[a] \in \mathbb{Z}_m$ has a multiplicative inverse iff $gcd(a,m) = 1$. Multiplicative inverse is unique. 

\underline{Inverses in $\mathbb{Z}_p$ (INV $\mathbb{Z}_p$)}

For all prime numbers $p$ and $[a] \in \mathbb{Z}_p$ have a unique multiplicative inverse. 

\subsection{Fermat's Little Theorem (F$\ell$T)}

Let $p$ be prime. Let $a \in \mathbb{Z}$. 

If $p \nmid a$, then $a^{p-1} \equiv 1 \pmod{p}$. 

\underline{Examples}

$4^6 \equiv 1 \pmod{7} \quad 39^6 \equiv 1 \pmod{7}$

$13^2 \equiv 1 \pmod{7}$ but not by F$\ell$T. 

\underline{Exercise}

What is the remainder when $7^{92}$ is divided by 11?

Since 11 is prime and $11 \nmid 7$, $7^{10} \equiv 1 \pmod{11}$. 
\begin{align*}
    7^{92} \equiv (7^{10})^9 \cdot 7^2 \equiv 1^9 \cdot 7^2 \equiv 49 \equiv 5 \pmod{11}
\end{align*}

By CAM, CER, CP. Thus, the remainder is 5.

\underline{Notes}

We can write $a^{p-1} \equiv 1 \pmod{p}$ as $[a^{p-1}] = [1]$ in $\mathbb{Z}_p$. In this case $[a]^{-1} = [a^{p-2}]$

Idea of Proof of F$\ell$T

Let $a = 4$ and $p = 7$.

$\{[4],[2 \cdot 4], [3 \cdot 4], [4 \cdot 4], [5 \cdot 4], [6 \cdot 4]\}$

$= \{[4],[1],[5],[2],[6],[3]\}$

No zero, all distinct. 

\underline{Corollary to F$\ell$T}

Let $p$ be prime. Let $a \in \mathbb{Z}$. 

Then $a^p \equiv a \pmod{p}$

\underline{Proof}

Let $p$ be prime. Let $a \in \mathbb{Z}$. We will use cases. 

When $p \nmid a$, by F$\ell$T, $a^{p-1} \equiv 1 \pmod{p}$. Multiplying gives $a^p \equiv a \pmod{p}$ by CAM. 

When $p \vert a$, $a \equiv 0 \pmod{p}$. Thus $a^p \equiv 0 \pmod{p}$ by CP. Thus $a^p \equiv a \pmod{p}$ by CER. 

The statement is true in all cases. $\blacksquare$

\underline{Exercise}

What is the remainder when $8^{(9^7)}$ is divided by 11. 
\begin{align*}
    9^7 &\equiv -1 \pmod{10}\\
    &\equiv 9 \pmod{10}\\
    8^{9^{7}} \equiv 8^{10q + r} &\equiv (8^{10})^q8^r \equiv 8^r \pmod{11}
\end{align*}

\underline{Simultaneous Congruences Examples}

Solve $x \equiv 2 \pmod{13}, x \equiv 17 \pmod{29}$. If moduli are coprime, always get one solution.

Rewrite the second statement as $x = 17 + 29k$ where $k \in \mathbb{Z}$. 

Thus we want to find all $k$ satisfying:
\begin{align*}
    &17+29j \equiv 2 \pmod{13}\\
    &\iff 29k \equiv 11 \pmod{13} \\
    &\iff 3k \equiv 11 \pmod{13} \\
    &\iff k \equiv 8 \pmod{13} \\
    &\iff k = 8 + 13\ell \text{ for some } \ell \in \mathbb{Z} 
\end{align*}
Sub to get 
\begin{align*}
    x &= 17 + 29(8 + 13\ell) \\
    x &= 17 + 29 \cdot 8 + 29 \cdot 13\ell \\
    x &= 249 + 377\ell
\end{align*}

The solution is $x \equiv 249 \pmod{377}$

\subsection{Chinese Remainder Theorem}

Suppose $gcd(m_1, m_2) = 1$ and $a_1, a_2 \in \mathbb{Z}$

There is a unique solution module $m_1m_2$ to the system
\begin{align*}
    x &\equiv a_1 \pmod{m_1}\\
    x &\equiv a_2 \pmod{m_2}
\end{align*}

That is, once we have one solution $x = x_0$, CRT also tells us the full solution is $x \equiv x_0 \pmod{m_1m_2}$

\underline{Generalized CRT}

If $m_1, m_2, \ldots, m_k \in \mathbb{N}$ and $gcd(m_i, m_j)=1$ then for any integers there exists a solution to simultaneous congruences. 
\begin{align*}
    n &\equiv a_1 \pmod{m_1} \\
    &\vdots \\
    n &\equiv a_k \pmod{m_k}
\end{align*}

The complete solution is $n \equiv n_0 \pmod{m_1m_2\ldots m_k}$

Exercises

$x \equiv 4 \pmod{6}, x \equiv 2 \pmod{8}$. 

Rewrite the second equation as $x = 2 + 8k$ where $k \in \mathbb{Z}$. Sub into the first equation to get 
\begin{align*}
    2 + 8k &\equiv 4 \pmod{6}\\
    8k &\equiv 2 \pmod{6} \\
    2k &\equiv 2 \pmod{6}
\end{align*}

Since 1 is a solution, the full solution is $k \equiv 1 \pmod{3}$ by LCT. 

Rewrite as $k = 1 + 3\ell$ where $\ell \in \mathbb{Z}$. Sub to get $x = 2+ 8(1 + 3\ell), x = 10 + 24\ell$.

Final answer is $x \equiv 10 \pmod{24}$. 

\subsection{Splitting the Modulus}

Let $m_1$ and $m_2$ be coprime positive integers. For any two integers $x$ and $a$,
\begin{align*}
    x \equiv a \pmod{m_1}, x \equiv a \pmod{m_2} \iff x \equiv a \pmod{m_1m_2}
\end{align*}

\underline{Exercise}

What is the units digit of $8^{(9^{7})}$?

\underline{Rough}
\begin{align*}
    8^{(9^{7})} &\equiv r \pmod{10}\\
    r &\equiv 8^{(9^{7})} \pmod{2}\\
    r &\equiv 8^{(9^{7})} \pmod{5} \\
    r &\equiv 0 \pmod{2} \\
    8^{(9^{7})} &\equiv 3^{(9^{7})} \pmod{5}\\
    9 &\equiv 1 \pmod{4}\\
    \therefore 9^7 &\equiv 1 \pmod{4}\\
    \therefore 9^7 &\equiv 4\ell + 1 \text{ for some } \ell \in \mathbb{Z}
\end{align*}

So we get 
\begin{align*}
    8^{(9^{7})} \equiv 3^{4k+1} \equiv (3^4)^k \cdot 3 \equiv 1^k 3 \equiv 3 \pmod{5}
\end{align*}

To complete the problem, we solve
\begin{align*}
    r &\equiv 0 \pmod{2}\\
    r &\equiv 3 \pmod{5} \\
    r &\equiv 8 \pmod{10}
\end{align*}

$8^{(9^{7})} \equiv r \pmod{11}, 8^{10} \equiv 1 \pmod{11}$ by F$\ell$T


\section{The RSA Public-Key Encryption Scheme}

Cool history lesson about \href{https://uwaterloo.ca/magazine/spring-2015/features/keeping-secrets}{William Tutte}

Message $\to$ encrypt $to$ transmit cipher $to$ decrypt $to$ message

Math functions (easy to encrypt), hard to decrypt (invert) without info. 

\underline{RSA Scheme}

\underline{Setup} (Bob)
\begin{enumerate}
    \item Randomly choose two large, distinct primes $p$ and $q$ and let $n = pq$
    \item Select arbitrary integer $e$ such that $gcd(e,(p-1)(q-1)) = 1$ and $1 < e < (p-1)(q-1)$
    \item Solve $ed \equiv 1 \pmod{(p-1)(q-1)}$ for an integer $d$ where $1 < d < (p-1)(q-1)$
    \item Publish the public key $(e,n)$
    \item Keep the private key $(d,n)$ secret, and the primes $p$ and $q$
\end{enumerate}

\underline{Encryption} (Alice does the following to send a message as ciphertext to Bob)
\begin{enumerate}
    \item Obtain a copy of Bob's public key $(e,n)$
    \item Construct the message $M$, an integer such that $0 \le M < n$
    \item Encrypt $M$ as the ciphertext $C$, given by $C \equiv M^e \pmod{n}$ where $0 \le C < n$
    \item Send $C$ to Bob
\end{enumerate}

\underline{Decryption} (Bob does the following to decrypt)
\begin{enumerate}
    \item Use the private key $(d,n)$ to decrypt the ciphertext $C$ as the received message $R$, given by $R \equiv C^d \pmod{n}$ where $0 \le R < n$
    \item Claim: $R = M$
\end{enumerate}

\underline{Setup}

$p = 2, q = 11, n = 22$

$\phi(n) = 10 (1 \times 10)$

$e = 3 \quad gcd(3,10) = 1$

$3d \equiv 1 \pmod{10} \leftarrow ed \equiv 1 \pmod{\phi(n)}$ where $0 < d < \phi(n)$. $d = 7$.

Public key $(e,n) \implies (3,22)$.

Private key $(d,n) \implies (7,22)$.

\underline{Encryption}

Generate message $M$ where $0 \le M < n$

$M = 8$
\begin{align*}
    C &\equiv 8^3 \pmod{22} \quad{0 \le C < n}\\
    &\equiv (-2) \cdot 8 \pmod{22}\\
    &\equiv 6 \pmod{22}
\end{align*}

\underline{Decryption}
\begin{align*}
    R &\equiv 6^7 \pmod{22} \quad{0 \le R < n}\\
    &\equiv (36)^3 6 \pmod{22}\\
    &\equiv 14^3 \cdot 6 \pmod{22}\\
    &\equiv 84 \cdot 2^2 \cdot 7^2 \pmod{22}\\
    &\equiv (-4) \cdot 6 \cdot 7 \pmod{22}\\
    &\equiv 8 \pmod{22}
\end{align*}

8 is the original message that Alice wanted to send.

\underline{Exercise}

Let $p=11, q=13, e = 23$
\begin{itemize}
    \item public key?
    \item private key?
    \item if $M=13$ what is $C$?
\end{itemize}

Public key: $(c,n) \to (23,143)$

Private key: solve $23d \equiv 1 \pmod{10 \cdot 12}$, $d \equiv 47$
\begin{align*}
    C \equiv 13^{23} &\pmod{143}\\
    \equiv 13^{16}&13^413^213^1 \pmod{143}\\
    13^2 &\equiv 169 \equiv 26 \pmod{123}\\
    13^4 &\equiv 26^2 \equiv \ldots \\
    &\vdots 
\end{align*}
Square and multiply, then use SMT if you know $p$ and $q$. 


\section{Complex Numbers}

\subsection{Standard Form}

\underline{Complex Numbers}

$\mathbb{N} \subsetneq \mathbb{Z} \subsetneq \mathbb{Q} \subsetneq \mathbb{R} \subsetneq \mathbb{C}$

\underline{Examples}

\begin{itemize}
    \item $2 + 3i \leftarrow$ standard form $\quad \mathbb{C} = \{x + yi: x,y \in \mathbb{R}\}$
    \item $\frac{1}{2} + (-\sqrt{2})i$
    \item $0 + 0i = 0$
    \item $1 + 1i = 1 + i$
\end{itemize}

For $z = x + yi \in \mathbb{C}$, we call $x$ the real part and $y$ the imaginary part. 

$Re(z)$ and $Im(z)$

$z = w$ means $Re(z)=Re(w)$ and $Im(z) = Im(w)$

$z = 7 + 0i = 7 \implies \mathbb{R} \subsetneq \mathbb{C} \implies z$ is purely real 

$z = 7i \implies$ purely imaginary

\underline{Arithmetic}

Addition:

$(a + bi)+(c+di)=(a+c)+(b+d)i$\\
$(2+3i)+(1+2i) = 3 + 5i$

Multiplication:

$(a+bi)\cdot(c+di) = (ac-bd)+(ad+bc)i$\\
$(2+3i)\cdot(5+4i) = ((2\cdot 5)-(3\cdot 4)) + ((2\cdot 4)+(3 \cdot 5))i = -2 + 23i$\\
$(0 + 1i)\cdot(0+1i) = -1 + 0i$\\
$i^2 = -1$

Informally we can treat elements of $\mathbb{C}$ as "normal" algebraic expressions where $i^2 = -1$ and when we do that "everything works". 

0 is the additive identity in $\mathbb{C}$.\\
$-z$ is the additive inverse of $z$ in $\mathbb{C}$.

\underline{Subtraction}

Let $w,z \in \mathbb{C}$. We define

$z - w = z +(-1 + 0i)w$

1 is the multiplicative identity in $\mathbb{C}$.\\
$\frac{a-bi}{a^2+b^2}$ is the unique multiplicative inverse of $a+bi\ne 0$

\underline{Division}
\begin{align*}
    \frac{3+4i}{1+2i} &= (3+4i)(1+2i)^{-1}\\
    &= (3+4i)(\frac{1-2i}{5}) \\
    &= (3+4i)(\frac{1}{5}-\frac{2}{5}i) \\
    &= (\frac{3}{5} + \frac{8}{5}) - \frac{2}{5}i\\
    &= \frac{11}{5} - \frac{2}{5}i
\end{align*}

Why is $(1+2i)^{-1} = \frac{1-2i}{5}.$
\begin{align*}
    &\text{Let } (1+2i)^{-1} = x + yi \text{ where }x,y\in \mathbb{R}\\
    &\text{Then } (1+2i)(x+yi) = 1 + 0i\\
    &= (x-2y)+(y+2x)i = 1 + 0i\\
    &x-2y = 1\\
    &\underbrace{y+2x = 0}_{\underbrace{x = \frac{1}{5}, y=-\frac{2}{5}}_{\text{multiplicative inverse}}}
\end{align*}

Alternatively
\begin{align*}
    \frac{3+4i}{1+2i} \cdot \frac{1-2i}{1-2i} &= \frac{(3+4i)(1-2i)}{5}\\
    &= 11 - 2i\\
    &= \frac{11}{5} - \frac{2}{5}i
\end{align*}

\underline{Properties of Complex Arithmetic (PCA)}

Let $u,v,z \in \mathbb{C}$ with $z = x + yi$
\begin{align*}
    &(u + v) + z = u + (v + z)\\
    &u + v = v + u\\
    &z + 0 = z \text{ where } 0 = 0 + 0i\\
    &z + (-z) = 0 \text{ where } -z = -x - yi\\
    &(uv)w = u(vw)\\
    &z \cdot 1 = z \text{ where } 1 = 1 + 0i\\
    &z \ne 0 \implies zz^{-1} = 1 \text{ where } z^{-1} = \frac{x-xi}{x^2 + y^2} \\
    &z(u+v) = zu + zv
\end{align*}

Proof that multiplicative inverses are unique in $\mathbb{C}$.

Let $z \in \mathbb{C}$ where $z \ne 0$.

Suppose $u \cdot z = 1$ and $v \cdot z = 1$ for $u,v \in \mathbb{C}$. 

Then $uz = vz$

Thus
\begin{align*}
    (uz)u &= (vz)u \\
    \implies u(zu) &= v(zu) \text{ by PCA 5}\\
    u &= v \quad \blacksquare
\end{align*}

\subsection{Conjugate and Modulus}

Warm-up 

$\frac{(1-2i)-(3+4i)}{5-6i}$

$=\frac{-2-6i}{5-6i} \cdot \frac{5+6i}{5+6i}$

$i^{2022} = -1$ since $(i^2)^{1011}$

$6x^3 + (1 + 3\sqrt{2}i)z^2 - (11-2\sqrt{2}i)z - 6 = 0$. Let $r \in \mathbb{R}$.
\begin{align*}
    &6r^3 + (1 + 3\sqrt{2}i)r^2 - (11 - 2\sqrt{2}i)r-6 = 0 + 0i \\
    &6r^3 + r^2 - 11r - 6 = 0 \quad \text{a}\\
    &3\sqrt{2}r^2 + 2\sqrt{2}r = 0 \quad \quad \text{b}\\
    &\text{b } \implies \underbrace{r = 0}_{\neg\checkmark}, \underbrace{r = -\frac{2}{3}}_{\checkmark}
\end{align*}

\underline{Definition}

Let $z = a + bi$ be a complex number in standard form

The \underline{complex conjugate} of $z$ is $\overline{z} = a-bi$

\underline{Examples}

$5 + 6i = 5-6i \quad \overline{5-6i} = 5+6i$


\underline{Properties of Complex Conjugate (PCJ)}

Let $z,w \in \mathbb{C}$. Then,
\begin{enumerate}
    \item $\overline{\overline{z}} = z$
    \item $\overline{z + w} = \overline{z} + \overline{w}$
    \item z + $\overline{z}$ = $2Re(z)$; $\quad z - \overline{z} = 2Im(z)i$
    \item $\overline{zw} = \overline{z} \cdot \overline{w}$
    \item $z \ne 0 \implies \overline{z^{-1}} = \overline{z}^{-1}$
\end{enumerate}

$1 - 4$ can be proved by using standard form and showing $LHS = RHS$. 

Proof of $5$. 

Suppose $z \in \mathbb{C}$ where $z \ne 0$.

Therefore $z^{-1}$ exists and $zz^{-1} = 1$ by PCA. 

We get $\overline{zz^{-1}} = \overline{1}$.

Thus $\overline{z}\overline{z^{-1}} = 1$. That is, $\overline{z^{-1}} = \overline{z}^{-1}$

\underline{Exercise}

Solve $z^2 = i\overline{z}$

Rough work
\begin{align*}
    (a + bi)^2 &= i(a-bi)\\
    a^2 - b^2 + 2abi &= b + ia\\
    a^2 - b^2 &= b\\
    2ab &= a
\end{align*}

When $a = 0, b = 0, b = i$.

When $a \ne 0, b = \frac{1}{2}, a = \frac{\sqrt{3}}{2}$, or, $a = -\frac{\sqrt{3}}{2}, b = \frac{1}{2}$.

Thus there are 4 solutions. 


\underline{Modulus}

Let $z = x + yi \in \mathbb{C}$. 

The modulus of $z$ is $|x + yi| = \sqrt{x^2 + y^2}$. 

\underline{Examples}

$|5 + 6i| = \sqrt{5^2 + 6^2} = \sqrt{61}$\\
$|5-6i| = \sqrt{61}$\\
$|135|=135$\\
$|-135|=135$

\underline{Properties of Modulus}

$|z| = 0$ iff $z = 0$\\
$|\overline{z}| = |z|$\\
$z \cdot \overline{z} = |z|^2$\\
$|zw| = |z||w|$\\
if $z \ne 0$, then $|z^{-1}| = |z|^{-1}$

Proof of the fourth statement above.

Let $z,w \in \mathbb{C}$. 

Consider 
\begin{align*}
    |zw|^2 &= (zw)(\overline{zw})\\
    &= zw(\overline{z}\overline{w})\\
    &= (z\overline{z})(w\overline{w})\\
    &= |z|^2|w|^2\\
    &= (|z||w|)^2
\end{align*}

Since the modulus of every complex number is a non-negative real number, we get 
\begin{align*}
    |zw| = |z||w| \quad \blacksquare
\end{align*}

\subsection{Complex Plane and Polar Form}

\underline{Complex Plane}

Imaginary axis is $y-$axis, real axis is $x-$axis.

$\overline{z}$ is the reflection of $z$ in the real axis. 

$|z|$ is the distance from $z$ to the origin $(\sqrt{x^2+y^2})$

$z + w$ is considered to be vector addition. 

\underline{Polar Form}

Standard form: $3 + 3i$\\
Cartesian Coordinates: $(3,3)$\\
Polar Coordinates: $(3\sqrt{2},\frac{\pi}{4})$\\
Polar Form: $3\sqrt{2}cis(\frac{\pi}{4}) \downarrow$

$3\sqrt{2}(\cos(\frac{\pi}{4}) + i\sin(\frac{\pi}{4})) = $ Standard Form

\underline{Definition}

The polar form of a complex number $z$ is 
\begin{align*}
    z = r(\cos\theta + i\sin\theta)
\end{align*}
where  $r = |z|$ and $\theta$ (an argument) is an angle measured counter-clockwise from the real axis.

\underline{Note}

Polar form is not unique (add multiples of 2$\pi$).

\underline{Examples}

Convert to standard form
$cis(\frac{\pi}{2})$\\
$r=1, |z| = 1$\\
$=i$

$2cis(\frac{3\pi}{4})$\\
$r=2, |z|=2$\\
$= -\sqrt{2} +\sqrt{2}i$

Convert from standard form

$\frac{1}{\sqrt{2}} - \frac{i}{\sqrt{2}}$

$(r,\theta) = (1,(\sqrt{\frac{1}{\sqrt{2}}^2 + \frac{1}{\sqrt{2}}^2}))$\\
$\theta = \frac{7\pi}{4}$\\
$=cis(\frac{7\pi}{4})$\\

$\sqrt{6}+\sqrt{2}i$\\
$r = \sqrt{8} = 2\sqrt{2}$\\
$\cos\theta = \frac{\sqrt{6}}{2\sqrt{2}}, \sin\theta = \frac{\sqrt{2}}{2\sqrt{2}}$\\
$\cos\theta = \frac{\sqrt{6}}{2\sqrt{2}}, \sin\theta = \frac{\sqrt{2}}{2\sqrt{2}}$\\
$=2\sqrt{2}cis(\frac{\pi}{6})$\\

$cis(\frac{15\pi}{6})$ in standard form.\\
$cis(\frac{15\pi}{6}) = cis(\frac{3\pi}{6}) = \frac{\pi}{2} = 1(0 + 1i) = i$

Write $-3\sqrt{2} + 3\sqrt{6}i$ in polar form.

$r^2 = 72, r = 6\sqrt{2}$.

$\cos\theta = \frac{-3\sqrt{2}}{6\sqrt{2}} = -\frac{1}{2}$\\
$\sin\theta = \frac{3\sqrt{6}}{6\sqrt{2}} = \frac{\sqrt{3}}{2}$

Thus $\theta = \frac{2\pi}{3}$

$6\sqrt{2}cis(\frac{2\pi}{3})$

\underline{Polar Multiplication of Complex Numbers}
\begin{align*}
    z_1z_2 = r_1r_2(\cos(\theta_1 + \theta_2) + i\sin(\theta_1 + \theta_2))
\end{align*}

\subsection{De Moivre's Theorem (DMT)}

For all $n \in \mathbb{Z}$ and $\theta \in \mathbb{R}$\\
\begin{align*}
    (\cos\theta + i\sin\theta)^n = \cos(n\theta) + i\sin(n\theta)
\end{align*}

\underline{Proof of Polar Multiplication in $\mathbb{C}$ (PM$\mathbb{C}$)}

Multiply in standard form and use trig identities.

\underline{Proof of DMT}

When $n \ge 0$, this is induction\\
When $n < 0$, we can translate to the previous case. 

Using rules for $cos(-x)$ and $sin(-x)$.

\underline{DMT Examples}

Write $(cis\frac{3\pi}{4})^{-100}$ in standard form.
\begin{align*}
    = cis(\frac{-300\pi}{4}) &= cis(-75\pi) \\
    &= cis(\pi) \\
    &= -1
\end{align*}

Write $(\sqrt{3}-i)^{10}$ in standard form
\begin{align*}
    (\sqrt{3}-i)^{10} &= (2cis\frac{11\pi}{6})^{10}\\
    &= 2^{10}cis(\frac{55\pi}{3}) \\
    &= 2^{10}cis(\frac{1}{2} + \frac{\sqrt{3}}{2}i)\\
    &= 512 + 512\sqrt{3}i
\end{align*}

\underline{Note}

Multiplying by $i$ corresponds to rotating $90^\circ$

\subsection{Complex $n$-th Roots Theorem (CNRT)}

$N^{\text{th}}$ Root Examples

Solve $z^6 = - 64$\\
Let $z = rcis\theta$ in polar form.

In polar form, $-64 = 64cis(\pi)$\\

Equating gives that
\begin{align*}
    (rcis\theta)^6 = 64cis(\pi)\\
    \implies r^6cis6\theta = 64cis(\pi)
\end{align*}

Since $r \in \mathbb{R}$ and $r \ge 0$, we get $r = 2$.

Also $\theta = \frac{\pi + 2\pi k}{6}$ where $k \in \mathbb{Z}$.

We get $2cis \frac{\pi}{6},2cis \frac{3\pi}{6}, 2cis\frac{5\pi}{6}, 2cis\frac{7\pi}{6}, 2cis\frac{9\pi}{6}, 2cis\frac{11\pi}{6}$

Roots of Unity

Solve $z^8 = 1$

$i, \frac{-1}{\sqrt{2}} + \frac{i}{\sqrt{2}}, -1, \frac{-1}{\sqrt{2}} - \frac{i}{\sqrt{2}}, -i, \frac{1}{\sqrt{2}} - \frac{i}{\sqrt{2}}, 1, \frac{1}{\sqrt{2}} + \frac{i}{\sqrt{2}}$

\subsection{Square Roots and the Quadratic Formula}

\underline{Quadratic Formula}

For all $a,b,c \in \mathbb{C}, a \ne 0$, the solutions to $az^2 + bz + c = 0$ are,
\begin{align*}
    \frac{-b\pm w}{2a} \quad \text{where } w^2 = b^2 - 4ac
\end{align*}

\section{Polynomials}

\subsection{Introduction}

\underline{Fields}

All non-zero numbers have a multiplicative inverse.

$ab = 0$ iff $a = 0$ or $b =0$

$\mathbb{Q}, \mathbb{R}, \mathbb{C}, \mathbb{Z}_p$ when $p$ is prime. 

\subsection{Arithmetic of Polynomials}

\underline{Polynomials}

No negative exponents, no fractional exponents. 
\begin{align*}
    a_nx^n + a_{n-1}x^{n-1}+\ldots + a_1x+a_o \quad \text{ is a polynomial over } \mathbb{F}.
\end{align*}

when $n \ge 0 \in \mathbb{Z}, a_n,a_{n-1} \in \mathbb{F}$.

Terminology/Notation

$iz^3 + (2+3i)z + \pi, z$ is indeterminate.

\begin{itemize}
    \item complex polynomial (not real)
    \item degree is $3$
    \item cubic polynomial
    \item in $\mathbb{C}[z]$
    \item $f(x)=g(x)$ means corresponding coefficients are equal
    \item polynomial equation (if there was an equal sign). Solution to that is a root.
\end{itemize}

\underline{Degree of a Product}
\begin{align*}
    degf(x)g(x) = degf(x) + degg(x)
\end{align*}

\underline{Division Algorithm for Polynomials}

If $f(x),g(x) \in \mathbb{F}[x]$, then $\exists q(x), p(x) \in \mathbb{F}[z]$ such that $f(x) =q(x)g(x) + r(x)$ where $r(x)$ is the 0 polynomial or $deg(r(x)) < deg(g(x))$

If $r$ is 0, $g(x) \vert f(x)$

Polynomial Arithmetic

Let $g(z) = z + (i + 1)$ and $q(z) = iz^2 + 4z - (1-i)$. Compute $q(z)g(z)$.

Find the $q$ and $r$ where

$f(z) = iz^3 + (i+3)z^2 + (5i+3)z + (2i-2)$\\
$g(z) = z + (i + 1)$

$\begin{array}{r}
iz^2 + 4z + (i -1)\phantom{)}   \\
z + (1 + i){\overline{\smash{\big)}\,iz^3 + (i+3)z^2 + (5i + 3)z + (2i - 2)\phantom{)}}}\\
\underline{-~\phantom{(}(iz^3+(-1 + i)z^2)\phantom{-b)}}\\
4z^2 + (5i + 3)\phantom{)}\\ 
\underline{-~\phantom{()}(4z^2 + (4 + 4i)z)}\\ 
\vdots\\
2i
\end{array}$


Yields $q(z) = iz^2 + yz + (i -1)$\\
$r(z) = 2i$

Check

$f(z) = g(z)q(z) - r(z)$

\underline{Exercise 3}

Prove $(x-1) \nmid (x^2 + 1)$

BWOC suppose $(x-1) \vert (x^2 + 1)$ in $\mathbb{R}[x]$.

Then by DP we have

$x^2 + 1 = (x-1)(ax + b)$ 

for some $a,b \in \mathbb{R}$ and $a \ne 0$.

If they are equal, coefficients must be the same. 

Comparing coefficients:

$1= a, 0 = b-a, 1 = -b$

Second and third above $\implies b - a = -2$


\subsection{Roots of Complex Polynomials and the Fundamental Theorem of Algebra}

\underline{Remainder Theorem (RT)}

For all fields $\mathbb{F}$, all polynomials $f(x) \in \mathbb{F}[x]$, and all $c \in \mathbb{F}$, the remainder polynomial when $f(x)$ is divided by $x-c$ is the constant polynomial $f(c)$.

\underline{Proof}

Let $f(x) \in \mathbb{F}[x]$ where $\mathbb{F}$ is a field. Let $c \in \mathbb{F}$.

By DAP, 

$f(x) = r(x-c)q(x) + r(x)$ for unique $g(x),r(x) \in \mathbb{F}[x]$ where $r(x)$ is the zero polynomial or $deg(r(x)) = 0$.

Regardless, $r(x) = r_0$ for some $r_0 \in \mathbb{F}$. 

Alas, $f(x) = (c-c)q(c) + r_0 = r_0$

\underline{Takeaway}

Finding roots corresponds to finding linear factors. 

\underline{Fundamental Theorem of Algebra (FTA)}

Every complex polynomial of complex degrees has a root. 

\underline{Complex Polynomials of Degree $n$ Have $n$ Roots (CPN) Proof Discovery}

Induction on $n$ degrees. 

Base Case

$az + b, a \ne 0$

$a(z - (-\frac{b}{a}))$

If $f(z)$ has degree $k+1$

By FTA, $f(z)$ has a root. Name it $c_{k+1}$.

Then $f(z) = g(z)(z-c_{k+1})$

\underline{Multiplicity}

The multiplicity of root $c$ of a polynomial $f(x)$ is the largest possible integer $k$ such that $(x-c)^k$ is a factor of $F(x)$.

\underline{Reducible and Irreducible Polynomial}

Polynomial in $F[x]$ of positive degree is a reducible polynomial in $F[x]$ when it can be written as the product of 2 polynomials of positive degree. 

Otherwise we say that the polynomial is irreducible in $P[x]$. 



$x^2 + 1$ is irreducible in $R[x]$

BWOC suppose $x^2 + 1$ is the product of $(ax+b)(cx+d)$ where $a,b,c,d \in \mathbb{R}$. Then compare coefficients. 

Prove that $x^4+2x^2 + 1$ has no roots in $\mathbb{R}$ but is reducible. 

$x^4+2x^2+1$ 

$(x^2+1)(x^2+1)$

Prove factors don't have roots to prove no roots (lots of ways to show no roots)

Write $x^2+1$ as a product of irreducible factors in $\mathbb{C}[x]$
\begin{align*}
    x^2+1 = (x-i)(x+i)
\end{align*}

Write $x^4+2x+1$ as a product of irreducible factors
\begin{align*}
    x^4+2x^2+1 = (x-i)^2(x+i)^2
\end{align*}

Factor $ix^3 + (3-i)x^2 + (-3-2i)x - 6$ as a product of linear factors.

Hint $-1$ is a root

$\begin{array}{r}
ix^2 + (3-2i)x -6\phantom{)}   \\
x+1{\overline{\smash{\big)}\,ix^3 + (3-i)x^2 + (-3-2i)x - 6\phantom{)}}}\\
\underline{-~\phantom{(}(ix^3+ix^2)\phantom{-b)}}\\
(3-2i)x^2 + (-3-2i)x\phantom{)}\\ 
\underline{-~\phantom{()}(3-2i)x^2 + (3-2i)x}\\ 
\vdots\\
0
\end{array}$

The roots of this quotient are $\frac{(-3-2i)\pm w}{2i}$ where $w^2 = (3-2i)^2 + 24i$ by QF. 

Let $wa + bi$ where $a,b \in \mathbb{R}$

Then $a^2 - b^2 = 5, 2ab = 12, a =3, b =2$

So the roots are $\frac{(-3-2i)\pm 3 + 2i}{2i}$.

That is
\begin{align*}
    &\frac{(-3-2i)+3+2i}{2i}\\
    &= \frac{4i}{2i}\\
    &= 2
\end{align*}

and 
\begin{align*}
    &\frac{(-3-2i)-3+2i}{2i}\\
    &= \frac{-6}{2i}\\
    &= 3i
\end{align*}

Roots are $-1,2,3$

Hence the final answer is 
\begin{align*}
    i(x+1)(x-2)(x-3i)
\end{align*}

Write $x^4-5x^3 +16x^2-9x-13$ as a product of irreducible polynomials given that $2-3i$ is a root. 

\subsection{Real Polynomials and Conjugate Roots Theorem}

$f(x)$, if $z \in \mathbb{C}$ and $f(z) = 0$, then $f(\overline{z}) = 0$. Depends on the fields.

By CJRT, $2+3i$ is also a root. Thus, $(x-(2-3i))(x-(2+3i))$ is a factor. 

This quadratic factor equals, $x^2-4x+13$

Now we use long division to yield, $x^2 - x - 1$

By QF, the roots of $x^2 - x -1$ are $\frac{1\pm \sqrt{5}}{2}$

Therefore, 
\begin{align*}
    (x-(2-3i))(x-(2+3i))(x-\frac{1+\sqrt{5}}{2})(x-\frac{1-\sqrt{5}}{2}) \text{ over } \mathbb{C}.
\end{align*}

or 
\begin{align*}
 x^2 - 4x + 13)(x - \frac{1+\sqrt{5}}{2})(x+\frac{1-\sqrt{5}}{2}) \in \mathbb{R}   
\end{align*}

or
\begin{align*}
    (x^2 - 4x + 13)(x^2 - x - 1) \in \mathbb{Q}
\end{align*}

\underline{Real Quadratic Factors}

If $f(c) = 0$ for some $c \in \mathbb{C}$ with $Im(C) \ne 0$, $\exists$ real quadratic irreducible polynomial $g(x)$ and real polynomial $q(x)$ such that $f(x) = g(x)q(x)$

\underline{Real Factors of Real Polynomials}

Every non-constant with real coefficients can be written as a product of real linear and quadratic factors.

Proof of CJRT

Let $f(x) = a_nx^n + a_{n-1}x^{n-1} + \ldots + a_1x + a_0$. Where $a_n, a_{n-1}, \ldots, a_0 \in \mathbb{R}$. 

Let $z \in \mathbb{C}$ and assume $f(z) = 0$

Now we get,
\begin{align*}
    f(\overline{z}) &= a_n(\overline{z})^n + a_{n-1}(\overline{z})^{n-1} + \ldots + a_1 \overline{z} + a_0 \\
    &= a_n(\overline{z^n})+a_{n-1}(\overline{z^{n-1}}) + \ldots + a_1 \overline{z} + a_0 \text{ by PCJ}\\
    &= \overline{a_n}(\overline{z^n}) + \overline{a_{n-1}}(\overline{z^{n-1}}) + \overline{a_1}\overline{z} + \overline{a_0} \\
    &= \overline{a_n z^n + a_{n-1}z^{n-1} + \ldots + a_1 z + a_0} \text{ by PCJ}\\
    &= \overline{0} = 0 \quad \blacksquare
\end{align*}




\end{document} 